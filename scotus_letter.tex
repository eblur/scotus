\documentclass[letterpaper,11pt]{letter}

\usepackage{multicol}

%% Page dimensions for 1 inch margins
\setlength{\textwidth}{6.5in}
\setlength{\textheight}{9in}
\setlength{\topmargin}{-0in}
\setlength{\oddsidemargin}{0in}
\setlength{\evensidemargin}{0in} 
\setlength{\headheight}{0in}
\setlength{\headsep}{0in} 
\setlength{\hoffset}{0in}
\setlength{\voffset}{0in}

\begin{document}
\pagestyle{plain}
\pagenumbering{arabic}
%\pagenumbering{gobble}

%\newcommand \justice {Chief Justice John G. Roberts}
%\newcommand \justice {Justice Antonin Scalia}
%\newcommand \justice {Justice Anthony Kennedy}
%\newcommand \justice {Justice Clarence Thomas}
%\newcommand \justice {Justice Ruth Bader Ginsberg}
\newcommand \justice {Justice Stephen Breyer}
%\newcommand \justice {Justice Samuel Alito}
%\newcommand \justice {Justice Sonia Sotomayor}


\today \\ \\
\justice \\
Supreme Court of the United States \\
1 First Street, NE \\
Washington, DC 20543

{\bf Dear \justice},

We are writing to you today as professional physicists and astrophysicists to respond to comments made by Justices in the course of oral arguments of Fisher vs. University of Texas which occurred on Wednesday, December 9, 2015. First, we strongly repudiate the line of questioning from Justice Antonin Scalia based on the discredited Mismatch Theory [1]. Secondly, we are particularly called to address the question from Chief Justice John Roberts about the value of promoting equity and inclusion in our own field, physics. 

We share the outrage and dismay already expressed by many other groups and individual scientists over the comments of Justice Scalia, which appear to endorse the claim made in the amicus curiae brief of Heriot and Kirsanow, that affirmative action prevents black people from becoming scientists. We take this opportunity to strongly rebuke this claim and offer a rebuttal.

We object to the use of STEM (science, technology, engineering, and math) fields as a paper tiger in the debate over affirmative action. We as professional scientists are in strong support of affirmative action policies. As we work continuously to educate ourselves about the obstacles facing students of color, we see, now more than ever, a need for action.

We are working very hard to solve the ongoing problem of the lack of underrepresented minorities in the professional community of physicists and astronomers. In spite of the misguided claims of Heriot and Kirsanow, that ``gaps in academic credentials are imposing serious educational disadvantages on... minority students, especially in the areas of science and engineering,'' science is not an endeavor which should depend on the credentials of the scientist. Rather, a good scientist is one who does good science. We hope to push our community towards equity and inclusion so that the community of scientists more closely matches the makeup of humankind, because the process of scientific discovery is a human endeavor that benefits from removing prejudice against any race, ethnicity, or gender. Indeed, science relies heavily on consensus about acceptable results as well as future research directions, making diversity among scientists a crucial aspect of objective, bias-free science [2, 3]. Affirmative action programs that aim to bring the numbers of minority students to more proportional levels are an important ingredient in our ongoing work. Blaming affirmative action for our community's lack of progress in this regard is not only wrong, it is plainly ignorant of what we as scientists have determined must be done to reform our pedagogical and social structures to achieve the long-delayed goal of desegregation.

Affirmative action is just one part of a larger set of actions needed to achieve social justice within our STEM and education fields. In their brief, Heriot and Kirsanow claim that affirmative action causes fewer minority students to enter technical fields because their completion rates are low. Unlike Heriot and Kirsanow, we are scientists and science educators who are keenly aware that merely adding students to a pipeline is not enough to correct for the imbalance of power. The experience of a minority student in STEM is often much different from that of a white student in STEM [4]. Minority students attending primarily white institutions commonly face racism, biases, and a lack of mentoring. Meanwhile, white students unfairly benefit psychologically from being overrepresented [5]. We argue that it is the social experience of minority students that is more likely to make them drop out, rather than a lack of ability.

Before Justice Scalia's remarks on black scientists, Chief Justice Roberts asked, ``what unique perspective does a minority student bring to physics class?'' and ``What [are] the benefits of diversity? in that situation?'' Before addressing these questions directly, we note that it is important to call attention to questions that weren't asked by the justices, such as, ``What unique perspectives do white students bring to a physics class?'' and ``What are the benefits of homogeneity in that situation?'' We reject the premise that the presence of minority students and the existence of diversity need to be justified, but meanwhile segregation in physics is tacitly accepted as normal or good. Instead, we embrace the assumption that minority physics students are brilliant [6] and ask, ``Why does physics education routinely fail brilliant minority students?''

This is what we see when we look at a minority student in a majority-white physics class: determination and an ability to overcome obstacles and work hard in stressful environments. We see this because we know that many students from minority backgrounds are subjected to social and political stress from institutionalized racism (past and present), a history of economic oppression, and societal abuse from both micro-aggressions and subtle racism. We believe that it is these qualities that make minority students able to succeed as physics researchers.

The implication that physics or ``hard sciences'' are somehow divorced from the social realities of racism in our society is completely fallacious. The exclusion of people from physics solely on the basis of the color of their skin is an outrageous outcome that ought to be a top priority for rectification. The rhetorical pretense that including everyone in physics class is somehow irrelevant to the practice of physics ignores the fact that we have learned and discovered all the amazing facts about the universe through working together in a community. The benefits of inclusivity and equity are the same for physics as they are for every other aspect of our world.

The purpose of seeking out talented and otherwise overlooked minority students to fill physics classrooms is to offset the institutionalized imbalance of power and preference that has traditionally gone and continues to go towards white students. Minority students in a classroom are not there to be at the service of enhancing the experience of white students. 

We ask that you take these considerations seriously in your deliberations and join us physicists and astrophysicists in the work of achieving full integration and removing the pernicious vestiges of racism and white supremacy from our world.

Sincerely, 

{\bf The {\sl Equity \& Inclusion in Physics \& Astronomy} group and supporters}

\vspace{0.25in}

\large{{\bf References}}
{\small 

\noindent [1] Harris, Cheryl I., and William C. Kidder. ``The Black student mismatch myth in legal education: The systemic flaws in Richard Sander's affirmative action study.'' Journal of Blacks in Higher Education (2004): 102-105.

\noindent [2] Bug, Amy. ``Has Feminism Changed Physics?'' Signs: Gender and Science: New Issues 28.3 (2003): 881-899.

\noindent [3] Whitten, Barbara.``(Baby) Steps Towards Feminist Physics.'' Journal of Women and Minorities in Science and Engineering 18.2 (2012): 115-134.

\noindent [4] McGee, Ebony O., and Danny B. Martin. ````You Would Not Believe What I Have to Go Through to Prove My Intellectual Value!'' Stereotype Management Among Academically Successful Black Mathematics and Engineering Students.'' American Educational Research Journal 48.6 (2011): 1347-1389.

\noindent [5] Bandura, Albert. ``Perceived self-efficacy in cognitive development and functioning.'' Educational psychologist 28.2 (1993): 117-148.

\noindent [6] Leonard, Jacquelyn, and Martin, Danny B. (Eds.). The Brilliance of Black Children in Mathematics: Beyond the Numbers and Toward New Discourse. Charlotte, NC: Information Age Publishers. (2013)
}

\vspace{0.25in}
\large{{\bf For more information, contact:}}\\
Lia~Racquel~Corrales, {\sl Postdoctoral researcher at MIT}, lia@space.mit.edu, (617)-258-8119 \\
Joshua~Paul~Tan, {\sl Pontificia Universidad Cat\'{o}lica de Chile}, joshuapaultan@gmail.com \\
Alessondra~Springmann, {\sl University of Arizona}, asteroid@alum.mit.edu

\vspace{0.25in}

\large{{\bf Co-signers}} {\small {\sl in random order}}
\begin{multicols}{2}
{\footnotesize

\noindent
1.~{\bf David Schmeltzer}, Professor of Condensed Matter Theory, {\sl CCNY} \\
2.~{\bf Nicholas Timmons}, {\sl University of California, Irvine} \\
3.~{\bf Mark Cryer}, Associate Professor, {\sl Hamilton College} \\
4.~{\bf Justine Jang}, Student, {\sl MIT} \\
5.~{\bf Hunter G. Close}, Associate Professor, {\sl Texas State University} \\
6.~{\bf Kanene Ubesie}, MD, {\sl Dallas, Texas} \\
7.~{\bf Jennifer Sims} \\
8.~{\bf Wendy Cutler}, Ms, {\sl SAG/AFTRA} \\
9.~{\bf Mark Neyrinck}, PhD, {\sl Johns Hopkins University} \\
10.~{\bf Tyler Cohen}, {\sl Stony Brook University} \\
11.~{\bf Elena Higuchi}, MD Candidate, {\sl University of California San Diego School of Medicine} \\
12.~{\bf Joelle Murray}, Professor, {\sl Linfield College} \\
13.~{\bf Kidada Gilbertt-Lexis}, MD, MPH, {\sl MLD Pathology} \\
14.~{\bf Olle Heinonen}, Materials Scientist \\
15.~{\bf Ian Guinn}, MS/PhD student, {\sl University of Washington} \\
16.~{\bf Jack Wimberley}, PhD candidate, {\sl University of Maryland, College Park} \\
17.~{\bf Crystal Bailey}, Careers Program Manager, {\sl American Physical Society} \\
18.~{\bf Bryan Gaensler}, Professor, {\sl University of Toronto} \\
19.~{\bf Benjamin P. Kellman}, PhD, {\sl UC, San Diego} \\
20.~{\bf Dan Burns}, Mr., {\sl Los Gatos High School} \\
21.~{\bf Charles Shapiro}, PhD, {\sl Jet Propulsion Laboratory} \\
22.~{\bf Randall Bostick}, PhD, {\sl USAF} \\
23.~{\bf Kedarnath Sangam}, PhD, {\sl retired} \\
24.~{\bf Erika Hamden}, PhD, {\sl Caltech} \\
25.~{\bf Peter Gray Friedman}, PhD, Project Scientist on NASA GALEX satellite mission, {\sl Caltech, retired} \\
26.~{\bf Benjamen Paul Bycroft}, MS, {\sl University of Southern California} \\
27.~{\bf John S. Watson Jr.}, Natural Resource Protection Professional, {\sl D\&R Greenway Land Trust} \\
28.~{\bf Susan Singley}, Ms., {\sl Education} \\
29.~{\bf Jordan Sickle}, Academic Hourly, {\sl University of Illinois at Urbana-Champaign} \\
30.~{\bf Koissaba B.R. OLe}, PhD (ABD), {\sl Clemson University} \\
31.~{\bf Dawn Erb}, Professor, {\sl University of Wisconsin Milwaukee} \\
32.~{\bf Aleksey Generozov}, {\sl Columbia University } \\
33.~{\bf Jeffrey Grover}, PhD, {\sl MIT} \\
34.~{\bf Devon McCarthy}, Mr.  Devon McCarthy MS chemistry and forensic science, {\sl University of Massachusetts } \\
35.~{\bf Brenda Williams}, Elementary education, Athletic coaching, Paralegal, US postal service, {\sl USPS} \\
36.~{\bf Kenneth D. Cornett}, PhD, Principal Scientist, {\sl Veeder-Root Company} \\
37.~{\bf Jacqueline Doyle}, MS, {\sl Florida International University} \\
38.~{\bf Rachael Beaton}, PhD, {\sl Carnegie Observatories} \\
39.~{\bf Sam Zeller}, PhD, {\sl Fermi National Accelerator Laboratory} \\
40.~{\bf Andrea Egan}, MS in Physics, {\sl Software Engineer} \\
41.~{\bf Jefferson Strait}, Professor of Physics, {\sl Williams College} \\
42.~{\bf Gregory Rudnick}, Associate Professor and Director of Graduate Studies, {\sl University of Kansas} \\
43.~{\bf Alfred Porras, Jr.}, {\sl Retired HS teacher} \\
44.~{\bf Dick Marti}, MS, retired biologist, USDA, {\sl Retired from USDA} \\
45.~{\bf Malaika McKee}, PhD, {\sl University of Illinois Champaign Urbana} \\
46.~{\bf Martin Forsythe}, PhD \\
47.~{\bf Laron Johnson}, MD, MPP, {\sl Washington Adventist Hospital } \\
48.~{\bf Sharon Marcus}, PhD; Professor of English and Dean of Humanities, {\sl Columbia University} \\
49.~{\bf Robert Zisk}, M Ed., PhD Science Education, {\sl Rutgers University} \\
50.~{\bf Emily Petroff}, {\sl Swinburne University of Technology} \\
51.~{\bf Katelyn Allers}, Professor, {\sl Bucknell University} \\
52.~{\bf Colin Hayes}, M.S., MEd, {\sl Northside College Preparatory High School} \\
53.~{\bf Tunde Giwa}, CTO, {\sl The Juilliard School} \\
54.~{\bf Sabine Lammers}, Associate Professor, {\sl Indiana University} \\
55.~{\bf William Brown}, PhD, Research Scientist \\
56.~{\bf Patrick Crumley}, PhD, {\sl University of Amsterdam} \\
57.~{\bf Maryrose Barrios}, Physics Graduate Student, {\sl Georgia Institute of Technology } \\
58.~{\bf Christopher Roberts}, PhD Canidate, {\sl University of Massachestts - Lowell} \\
59.~{\bf Sammy Little}, Ms., {\sl Self employed} \\
60.~{\bf Susan E Ramlo}, Professor, {\sl The University of Akron} \\
61.~{\bf Paul Gallagher}, PhD candidate, Mathematics, {\sl MIT} \\
62.~{\bf Nadeen White}, MD \\
63.~{\bf Erica Batson}, BS- Biology, {\sl Pharmaceutical Research } \\
64.~{\bf Peter Shaffery}, Student, {\sl University of Colorado, Boulder} \\
65.~{\bf Anne Metevier}, PhD, {\sl Sonoma State University} \\
66.~{\bf James Correia Jr}, PhD, {\sl University of Oklahoma} \\
67.~{\bf Laura Lopez}, Professor, {\sl The Ohio State University} \\
68.~{\bf Joseph Gaalaas}, BS Physics, MS Engineering, {\sl Bose Corporation} \\
69.~{\bf Benjamin Ripman}, Senior Accelerator Operator, {\sl SLAC National Accelerator Laboratory} \\
70.~{\bf Adrienne Wootters}, PhD, {\sl Massachusetts College of Liberal Arts} \\
71.~{\bf Jim Bosch}, PhD, {\sl Princeton University} \\
72.~{\bf Kenneth Bloom}, Associate Professor, {\sl Department of Physics and Astronomy, University of Nebraska-Lincoln} \\
73.~{\bf Gabriel Chan}, Undergraduate, {\sl Yeh Lab at UCLA} \\
74.~{\bf Jennifer Piscionere}, PhD, {\sl Swinburne University of Technology} \\
75.~{\bf Berenice Garcia}, Physics Undergrad Studet, {\sl UC Berkeley} \\
76.~{\bf Benjamin L. Alterman}, B.A., {\sl University of Michigan } \\
77.~{\bf David Teichman}, Concerned Citizen \\
78.~{\bf Olga A. V\'{a}squez}, Associate Professor, {\sl University of California San Diego} \\
79.~{\bf Zachary Constan}, PhD, {\sl Michigan State University } \\
80.~{\bf Janice V. Williams}, M.Ed., {\sl The University of Texas at Austin} \\
81.~{\bf Suchitra Sebastian}, Associate Professor of Physics, {\sl University of Cambridge} \\
82.~{\bf Thomas Devlin}, Professor, Rutgers (Emeritus), {\sl U. Pennsylvania} \\
83.~{\bf Scott Gianelli}, PhD, {\sl Hofstra University} \\
84.~{\bf Travis Merritt}, PhD, {\sl Virginia Tech} \\
85.~{\bf Willard Williams Jr.,}, Registered Architect, {\sl Rutledge Maul Architects} \\
86.~{\bf Charles La Rosa}, Educator, {\sl Brattleboro Union High School} \\
87.~{\bf Frederick C Miller}, MD, {\sl Private Practice in Philadelphia PA} \\
88.~{\bf Ram\'{o}n S. Barthelemy}, PhD, {\sl S\&T Policy Fellow} \\
89.~{\bf Edgar Ibarra}, Undergraduate, {\sl University of California Berkeley } \\
90.~{\bf Matt Young}, {\sl Boston University} \\
91.~{\bf Steven J Sweeney}, PhD, Assistant Professor of Physics, {\sl DeSales University} \\
92.~{\bf Elizabeth Wayne}, PhD, {\sl University of North Carolina at Chapel Hill} \\
93.~{\bf Colette Salyk}, PhD, {\sl Vassar College} \\
94.~{\bf Elizabeth Bajema}, PhD candidate, {\sl Northwestern University} \\
95.~{\bf Valerie Connaughton}, PhD, {\sl Universities Space Research Association} \\
96.~{\bf Warren Focke}, PhD, {\sl SLAC National Accelerator Laboratory} \\
97.~{\bf Michael Strauss}, Professor, {\sl Princeton University} \\
98.~{\bf Joseph E. McEwen}, PhD candidate, {\sl The Ohio State University } \\
99.~{\bf Zoya Vallari}, PhD, {\sl Stony Brook University} \\
100.~{\bf David Talaga}, Professor, {\sl Montclair State University } \\
101.~{\bf Michelle Francl}, Professor of Chemistry, {\sl Bryn Mawr College} \\
102.~{\bf Taylor Tobin}, {\sl University of Illinois} \\
103.~{\bf Nathan Schine}, PhD candidate, {\sl University of Chicago} \\
104.~{\bf Tanya Berger-Wolf}, PhD, Professor, {\sl University of Illinois at Chicago} \\
105.~{\bf David Klassen}, Professor, PhD, {\sl Rowan University} \\
106.~{\bf Bryan Hilbert}, MS, {\sl Space Telescope Science Institute} \\
107.~{\bf John Unverferth}, Grad Student, {\sl Montana State University} \\
108.~{\bf Barbara L. Whitten}, Professor, {\sl Colorado College} \\
109.~{\bf Robin McCown}, MS, {\sl Boise State University} \\
110.~{\bf Jeffrey Jorge}, MEd, {\sl edX} \\
111.~{\bf Martin Weinhous}, PhD (Physics) \\
112.~{\bf Dr. Vincent Sulkosky}, {\sl University of Virginia} \\
113.~{\bf Melissa Dancy}, PhD, {\sl University of Colorado} \\
114.~{\bf Christopher Rose}, Professor of Engineering,  Associate Dean of the Faculty, {\sl Brown University} \\
115.~{\bf Shanadeen Begay}, PhD, Lecturer, Post-Doc, {\sl Northeastern University} \\
116.~{\bf Daniel Outerbridge} \\
117.~{\bf Rebecca Lindell}, Professor \\
118.~{\bf Elliot Lipeles}, Professor, {\sl University of Pennsylvania} \\
119.~{\bf Lauren Tompkins}, Assistant Professor, {\sl Stanford University} \\
120.~{\bf Amy L. Magnus}, Research Assistant Professor of Engineering Physics, {\sl Air Force Institute of Technology} \\
121.~{\bf Daniel Harlow}, PhD, {\sl Harvard University} \\
122.~{\bf Sarah Demers}, Associate Professor, {\sl Yale University} \\
123.~{\bf Isaac Hall}, PhD, {\sl STAR Geophysics} \\
124.~{\bf David Dethmers}, MA, MPP \\
125.~{\bf Alicia Aarnio}, Assistant Research Scientist, {\sl University of Michigan} \\
126.~{\bf Jennifer Hathaway}, Fine Artist \\
127.~{\bf Luke Walker Goetzke}, PhD, {\sl Columbia University} \\
128.~{\bf Jason Scott Hamilton}, PhD, {\sl Elite Sports Psychology,LLC} \\
129.~{\bf D. Curtis Showers}, Deacon, {\sl From our home.} \\
130.~{\bf Carmen A. Pantoja}, PhD, {\sl University of Puerto Rico} \\
131.~{\bf Lin Yang}, {\sl Harvey Mudd College} \\
132.~{\bf Jennifer Sobeck}, PhD, {\sl University of Virginia} \\
133.~{\bf Luanna Soledad Gomez}, PhD, Physics, University of Washington (2001) \\
134.~{\bf Elizabeth H. Simmons}, University Distinguished Professor and Dean, {\sl Michigan State University } \\
135.~{\bf Robert Blackwell}, PhD, {\sl University of Colorado} \\
136.~{\bf Stephen A. Zelazny}, PhD, {\sl Stony Brook University } \\
137.~{\bf Bob Moss}, Faculty Emeritus, {\sl UC San Diego} \\
138.~{\bf Amy Sullivan}, PhD, {\sl University of Colorado at Boulder} \\
139.~{\bf Samuel Martin}, High School Physics/Astronomy Teacher \\
140.~{\bf Geoff Stanley}, Graduate Student, {\sl Stanford Biophysics} \\
141.~{\bf Joey Neilsen}, PhD, {\sl MIT} \\
142.~{\bf Toby Burmett}, emeritus professor of physics, {\sl University of Washington} \\
143.~{\bf James M. Valles, Jr.}, Professor, {\sl Brown University} \\
144.~{\bf Emily Freeland}, PhD, {\sl Stockholm University} \\
145.~{\bf Leo Stein}, PhD, {\sl California Institute of Technology (Caltech)} \\
146.~{\bf Joseph Barranco}, Professor, {\sl San Francisco State University} \\
147.~{\bf Isabella Johansson}, Undergrad, {\sl Columbia University} \\
148.~{\bf Harry Levine}, PhD Student, {\sl Harvard} \\
149.~{\bf Masao Sako}, Professor, {\sl University of Pennsylvania} \\
150.~{\bf David Kaplan}, Assoc. Professor of Physics, {\sl University of Wisconsin-Milwaukee} \\
151.~{\bf Justin Edmondson}, PhD, {\sl University of Michigan} \\
152.~{\bf Loretta B. DeLoggio}, Esq., {\sl Durham, NC} \\
153.~{\bf Fergal Mullally}, Phd., {\sl SETI/NASA Ames} \\
154.~{\bf Don Anthony Lloyd}, PhD, {\sl Institute for Defense Analyses} \\
155.~{\bf Donna Riley}, Professor of Engineering Education, {\sl Virginia Tech} \\
156.~{\bf Rachel Paterno-Mahler}, PhD, {\sl University of Michigan} \\
157.~{\bf Lisa E. Wills}, PhD, Educational Psychology, Research,  \& Measurement, {\sl National Aeronautics and Space Administration } \\
158.~{\bf Stacy Palen}, Full Professor, Physics, {\sl Weber State University} \\
159.~{\bf Geoffrey A. Cordell}, PhD, Professor Emeritus, {\sl University of Illinois} \\
160.~{\bf Moritz Günther}, PhD, {\sl MIT} \\
161.~{\bf Mackenzie Barton-Rowledge}, Graduate Student, {\sl University of Washington} \\
162.~{\bf Peter Steinberg}, PhD, Physicist, {\sl Brookhaven National Laboratory} \\
163.~{\bf Jonathan Friedman}, Professor of Physics, {\sl Amherst College} \\
164.~{\bf Robert H. Romer}, Professor, {\sl Amherst College} \\
165.~{\bf Daniel Woolridge}, MD Candidate, {\sl UC San Diego} \\
166.~{\bf Beverly Guy Sheftall}, Professor, {\sl Spelman College} \\
167.~{\bf Linda Strubbe}, Postdoctoral Fellow, {\sl University of British Columbia} \\
168.~{\bf Jacob Jencson}, PhD Student, {\sl California Institute of Technology} \\
169.~{\bf Mason Porter}, Prof., {\sl University of Oxford} \\
170.~{\bf Javier Gonzalez-Rocha}, PhD Student, {\sl Virginia Tech} \\
171.~{\bf Joel P. Baumgart, PhD}, Fulbright Scholar, {\sl Cornell University} \\
172.~{\bf James Gerald}, Assistant Professor, {\sl Delta State University} \\
173.~{\bf Jia-an Yan}, Assistant Professor, {\sl Towson University} \\
174.~{\bf Katie Auchettl}, PhD, {\sl The Ohio State University} \\
175.~{\bf Omar Eaton-Mart\'{i}nez} \\
176.~{\bf Kevin Satterfield}, Attorney, {\sl Law Firm} \\
177.~{\bf Ian G Thompson}, Facilities Engineer Sr., {\sl Lockheed Martin Space Systems} \\
178.~{\bf Scott Seagroves}, MS, {\sl The College of St. Scholastica} \\
179.~{\bf Charles Loelius}, {\sl Michigan State University} \\
180.~{\bf Ashlee Wilkins}, PhD candidate, {\sl University of Maryland, Department of Astronomy} \\
181.~{\bf Eva Richerson} \\
182.~{\bf Genevi\'{E}ve de Messi\'{E}res}, PhD \\
183.~{\bf David Robert Bergman}, PhD, {\sl Exact Solution Scientific Consulting LLC } \\
184.~{\bf Andrew Morrison}, Professor, {\sl Joliet Junior College} \\
185.~{\bf Norm Cunningham}, Lay person., {\sl In a confused and troubled world.} \\
186.~{\bf Philip Bull}, DPhil FRAS, {\sl JPL/Caltech} \\
187.~{\bf Anita Bowman}, MS Bioinformatics, {\sl University of Pennsylvania } \\
188.~{\bf Ian Dell'Antonio}, PhD, {\sl Brown University} \\
189.~{\bf Elizabeth Wailes}, PhD, {\sl Charit\'{e}} \\
190.~{\bf Lauren Hetherington}, PhD candidate, {\sl University of Toronto} \\
191.~{\bf Michael Macmillan}, MS, {\sl MIT} \\
192.~{\bf Amara Graps}, PhD, {\sl Planetary Science Institute} \\
193.~{\bf Maurice Wilson}, B.Sc. Student, {\sl Embry-Riddle Aeronautical University} \\
194.~{\bf Robert B Lynch}, PhD Professor, {\sl U.C. Davis} \\
195.~{\bf Lina Woo}, Advisory Council Member, {\sl Conference on Asian Pacific American Leadership,  CAPAL.org} \\
196.~{\bf Laura Watkins}, PhD, {\sl Space Telescope Science Institute} \\
197.~{\bf Tran Chau}, MA in Education, {\sl Arcadia University} \\
198.~{\bf Thomas Carroll}, PhD, {\sl Ursinus College} \\
199.~{\bf Andreas Faisst}, Postdoc \\
200.~{\bf Scott Milam}, Teacher, {\sl Plymouth High School} \\
201.~{\bf David Dodsworth}, Physics PhD Student, {\sl Brandeis University} \\
202.~{\bf Francesca Sanders} \\
203.~{\bf Judith S Denison}, MS, Physics, Univ of Colorado, {\sl Retired} \\
204.~{\bf Niescja Turner}, Zilker Professor of Physics, {\sl Trinity University} \\
205.~{\bf Darren T. Garnier}, PhD, {\sl Newton Scientific, Inc.} \\
206.~{\bf Enrique Suarez}, MS, {\sl CU Boulder} \\
207.~{\bf James Stankevitz}, Physics teacher, {\sl Wheaton Warrenville South High School} \\
208.~{\bf Mary Rauner}, PhD, {\sl WestEd} \\
209.~{\bf Julie A Rathbun}, Professor, {\sl University of Redlands } \\
210.~{\bf Salvatore Rappoccio}, PhD, Assistant Professor of Physics, {\sl University at Buffalo, State University of New York} \\
211.~{\bf Kevin Baptista}, {\sl Accenture} \\
212.~{\bf Ken Muir}, PhD, {\sl AppState} \\
213.~{\bf Aaron Lopez}, BS, {\sl UC Santa Cruz} \\
214.~{\bf Megan Saylor}, PhD, {\sl Vanderbilt University} \\
215.~{\bf Grace Telford}, MS, {\sl University of Washington} \\
216.~{\bf Maria Beck}, PhD, {\sl University of North Texas, retired} \\
217.~{\bf Dana Rice}, DrPH, {\sl Wayne State University School of Medicine} \\
218.~{\bf Valerie M. Reynolds}, MSJ, MBA, {\sl University of Chicago} \\
219.~{\bf Timothy D. Morton}, PhD, {\sl Princeton} \\
220.~{\bf Patrick Morgan}, Physics Outreach Coordinator, {\sl Michigan State University} \\
221.~{\bf Mary Wallingford}, PhD, {\sl University of Washington} \\
222.~{\bf D. Matthew Coleman}, Portfolio Analyst, MSc, MBA, {\sl Hedge Fund} \\
223.~{\bf Meredith Reitz}, PhD, {\sl US Geological Survey } \\
224.~{\bf Laurel Anderson}, Graduate student, {\sl Harvard University} \\
225.~{\bf Zuri McClelland}, Graduate Student, {\sl Illinois Institute of Technology} \\
226.~{\bf Dima Kamalov}, MS, {\sl SFSU} \\
227.~{\bf Ashley Baker}, PhD Student, {\sl University of Pennsylvania} \\
228.~{\bf David M. Cole}, PhD \\
229.~{\bf Brandon Bozek}, PhD, {\sl University of Texas, Austin} \\
230.~{\bf Jessica Kirkpatrick}, PhD, {\sl Hired} \\
231.~{\bf Ellen Buettner}, Daughter of a physicist who worked on the space shuttle, {\sl Keller Williams} \\
232.~{\bf Kate Baker}, MS., {\sl Retired} \\
233.~{\bf Lauren Woolsey}, PhD candidate, {\sl Harvard University} \\
234.~{\bf Karl Martinez}, BS, MA Physics - The University of Texas at Austin, {\sl HS2 Solutions} \\
235.~{\bf Melei Kibrom} \\
236.~{\bf Eric Hauck}, MS Entomology, {\sl Woodland High School} \\
237.~{\bf Eliza Stefaniw}, Physics educator and attorney \\
238.~{\bf Samuel A. Yearout}, Radiologic Technologist, {\sl Saint Mary's Medical Center, San Francisco } \\
239.~{\bf Doug Swanson}, MS \\
240.~{\bf Kaitlyn Hood}, PhD candidate, {\sl UCLA} \\
241.~{\bf Warner W. Johnston}, Life Senior Member IEEE and past NY Section Chair, {\sl Retired} \\
242.~{\bf Tshaka Cunningham}, Ph.D., Molecular Biology, {\sl George Mason University} \\
243.~{\bf Daniel Reinholz}, PhD, {\sl University of Colorado Boulder} \\
244.~{\bf Bala Krishnamoorthy}, PhD, Associate Professor of Mathematics, {\sl Washington State University} \\
245.~{\bf Robert Minchin}, PhD, {\sl Arecibo Observatory} \\
246.~{\bf Anna Frebel}, Professor, {\sl MIT} \\
247.~{\bf Joseph Megel}, Professor, {\sl University of North Carolina at Chapel Hill} \\
248.~{\bf Diane Blackwood}, PhD, {\sl MSU} \\
249.~{\bf Nimarta Chowdhary}, Student Researcher, {\sl University of Maryland} \\
250.~{\bf Abhay Shastry}, PhD, {\sl University of Arizona} \\
251.~{\bf Julie Biteen}, Professor, {\sl University of Michigan} \\
252.~{\bf Dominique Segura-Cox}, {\sl University of Illinois} \\
253.~{\bf Gianfausto dell'Antonio}, professor, {\sl SISSA-ISAS} \\
254.~{\bf Bryan Ramson}, PhD Candidate, {\sl University of Michigan/Fermilab} \\
255.~{\bf Kedward Reyes} \\
256.~{\bf Peter Wittich}, Associate Professor, {\sl Cornell University} \\
257.~{\bf Jason Harris}, Business Intelligence Senior Analyst, {\sl The Hartford (Dartmouth College graduate) } \\
258.~{\bf Rachel Mandelbaum}, Professor, {\sl Carnegie Mellon University} \\
259.~{\bf Stuart Raby}, Professor of Physics, {\sl The Ohio State University } \\
260.~{\bf Celeste Melamed}, PhD Student, {\sl Colorado School of Mines} \\
261.~{\bf Amber Johnson}, MS, {\sl University of Maryland} \\
262.~{\bf Piers Coleman}, Professor, {\sl Rutgers, the State University of New Jersey} \\
263.~{\bf Nancy Vincent Zinke}, {\sl retired RN/BSN/PHN} \\
264.~{\bf Stefano Profumo}, Professor, {\sl University of California, Santa Cruz } \\
265.~{\bf Jessica Dwyer}, Professional, {\sl Health Care } \\
266.~{\bf M. Lisa Manning}, Professor, {\sl Syracuse University } \\
267.~{\bf Amy Wareham}, Student, {\sl BS Biology} \\
268.~{\bf Sandra}, Teacher, {\sl Best Steps Family Child Care, Lakewood,CA} \\
269.~{\bf Jenn Dubois}, Recruiter, {\sl Managment Consulting Firm } \\
270.~{\bf Joseph Catanzarite}, Data Scientist / Astronomer, {\sl SETI Institute } \\
271.~{\bf Robert Bluhm}, Professor of Physics, {\sl Colby College} \\
272.~{\bf Marcelo Jaime}, PhD, {\sl Los Alamos National Laboratory} \\
273.~{\bf Claudia Lagos}, Assistant Professor, {\sl University of Western Australia} \\
274.~{\bf Kerstin Perez}, Professor, {\sl Haverford College} \\
275.~{\bf Don Howard}, Professor, {\sl University of Notre Dame} \\
276.~{\bf Scott Schneider}, PhD, {\sl Lawrence Technological University} \\
277.~{\bf Eric Church}, PhD, {\sl Pacific Northwest National Laboratory} \\
278.~{\bf Kirsten Tamayo}, JD Candidate, {\sl CUNY School of Law} \\
279.~{\bf Krishan Aghi}, PhD student, {\sl University of California, Berkeley} \\
280.~{\bf Robert Donohoe}, Executive Chef \\
281.~{\bf Mafmudije Selimi}, PhD, {\sl University of Wisconsin - Madison} \\
282.~{\bf Jack Streete}, PhD, {\sl Rhodes College (retired)} \\
283.~{\bf Paul Schulze}, PhD Physics, {\sl Professor Emeritus Abilene ChristianUniverstity} \\
284.~{\bf Eric Hooper}, PhD, {\sl University of Wisconsin-Madison} \\
285.~{\bf Katharine Guiles Ellis}, MS, MA, {\sl Boulder Valley School District} \\
286.~{\bf Rachel E Scherr}, PhD, {\sl Seattle Pacific University} \\
287.~{\bf Owen Marshall}, Technical Specialist, BS Physics - Ohio State University, {\sl Fermilab} \\
288.~{\bf Dafna Wu}, Nurse Practitioner, {\sl SF Dept of Public Health} \\
289.~{\bf Arlene Chasek}, Ms., {\sl Rutgers University} \\
290.~{\bf Larry Gladney}, Professor and Associate Dean of Natural Sciences, {\sl University of Pennsylvania} \\
291.~{\bf Jeffrey Filippini}, Assistant Professor of Physics and Astronomy, {\sl University of Illinois at Urbana-Champaign} \\
292.~{\bf Michael Wen Sen Su}, Professor, {\sl Pratt Institute School of Undergraduate Architecture} \\
293.~{\bf Chandriker Dass}, PhD \\
294.~{\bf Aatish Bhatia}, PhD; Associate Director, Council on Science \& Technology, {\sl Princeton University} \\
295.~{\bf Klaus-Michael Aye}, PhD, {\sl LASP, CU Boulder} \\
296.~{\bf Brokk Toggerson}, PhD, Lecturer, {\sl UMass, Amherst} \\
297.~{\bf Larry R. Nittler}, PhD, {\sl Carnegie Institution} \\
298.~{\bf John R. Royer}, PhD, {\sl NIST} \\
299.~{\bf Alicia Prieto Langarica}, PhD, {\sl Youngstown State University} \\
300.~{\bf Ward Lopes}, PhD \\
301.~{\bf Kathryn Devine}, PhD, {\sl The College of Idaho} \\
302.~{\bf Jorge Noronha}, Professor, {\sl University of Sao Paulo} \\
303.~{\bf Waheed Bhuyan}, BSEE, {\sl Primerica } \\
304.~{\bf Robert Fairclough}, PhD, {\sl UC Davis} \\
305.~{\bf D. P. Siddons}, Dr., {\sl Brookhaven Nat. Lab.} \\
306.~{\bf Douglas T.}, MS \\
307.~{\bf Edward Schwieterman}, MS, {\sl University of Washington} \\
308.~{\bf Daniel Bowring}, PhD, {\sl Fermi National Accelerator Laboratory} \\
309.~{\bf Joshua W Knight}, PhD, {\sl retired} \\
310.~{\bf Tim Andeen}, Assistant Professor, {\sl University of Texas at Austin} \\
311.~{\bf Valerie Hood} \\
312.~{\bf Walter E. Daniels}, PhD in Physics, {\sl University of Maryland (Retired)} \\
313.~{\bf Ted Brzinski}, Post Doctoral Research Associate, {\sl NC State University, Department of Physics} \\
314.~{\bf Kendra Letchworth Weaver}, PhD, {\sl Argonne National Laboratory} \\
315.~{\bf John V. Koger}, Psychophysiology Research Equipment Manager, {\sl University of Wisconsin-Madison} \\
316.~{\bf Nial Tanvir}, Professor, {\sl University of Leicester} \\
317.~{\bf Sarah Schorr}, B.A. Neuroscience \\
318.~{\bf Jennifer Cano}, PhD, {\sl Princeton Center for Theoretical Science} \\
319.~{\bf James D. White}, Professor, {\sl Juniata College} \\
320.~{\bf Carolyn Brinkworth}, PhD, {\sl National Center for Atmospheric Research} \\
321.~{\bf Jessica Metcalfe}, PhD, {\sl Argonne National Laboratory} \\
322.~{\bf Steve Feller}, Professor of Physics, {\sl Coe College} \\
323.~{\bf Lewis Maday-Travis}, Science Teacher, {\sl Seattle Academy of Arts and Sciences} \\
324.~{\bf Zel Fowler}, Doctoral Student, {\sl University of Phoenix} \\
325.~{\bf Julia Salevan}, PhD candidate, {\sl Yale University} \\
326.~{\bf Nicholas B. Schade}, PhD, {\sl University of Chicago} \\
327.~{\bf Glenn Corser}, Senior Principal Scientist, Retired, {\sl BAE Systems} \\
328.~{\bf Anika Joseph-Smith}, PhD, {\sl Grifols Diagnostic Systems} \\
329.~{\bf Denisha McPherson}, BS Industrial Engineering, {\sl Rensselaer Polytechnic Institute} \\
330.~{\bf Addonis Terrell}, Student, {\sl University Of Chicago Medicine} \\
331.~{\bf Catherine Murphy}, {\sl The Pennsylvania State University} \\
332.~{\bf Katharina Vollmayr-Lee}, Professor, {\sl Bucknell University} \\
333.~{\bf Joshua Nollenberg}, PhD, {\sl SUNY College at Oneonta} \\
334.~{\bf Henry Prager}, Graduate Student, {\sl New Mexico Institute of Mining and Technology (New Mexico Tech)} \\
335.~{\bf Lincoln Ribeiro}, PhD, {\sl Universidade Federal de Campina Grande, Brazil} \\
336.~{\bf Elliot Teichman}, High School Physics Teacher, {\sl Newark, NJ.} \\
337.~{\bf Kay Emmert}, Lecturer, {\sl University of Illinois Urbana-Champaign } \\
338.~{\bf Melissa Eblen-Zayas}, Associate Professor of Physics, {\sl Carleton College} \\
339.~{\bf Leah Huk Griffith}, Astrophysics graduate student, PhD candidate, {\sl University of Denver} \\
340.~{\bf Anita Allyn}, Professor, {\sl The College of New Jersey } \\
341.~{\bf Melanie Galloway}, MS, {\sl University of Minnesota} \\
342.~{\bf Krista Tweed}, PhD, {\sl HBM} \\
343.~{\bf Hannah Jang-Condell}, Assistant Professor, {\sl University of Wyoming} \\
344.~{\bf Rebecca Daskalova}, {\sl The Ohio State University} \\
345.~{\bf Judith A. Osborn}, PhD, {\sl retired from college teaching} \\
346.~{\bf Teresa Burns}, PhD and Professor \\
347.~{\bf Anne Vazquez}, PhD \\
348.~{\bf Jerome Fung}, Lecturer in Physics, {\sl Wellesley College} \\
349.~{\bf Daniel M. Drydeb}, Graduate Researcher, {\sl UC Davis} \\
350.~{\bf Martha Haynes}, Goldwin Smith Professor of Astronomy, {\sl Cornell University} \\
351.~{\bf Lisa Oglesby}, M. Ed., {\sl Retired} \\
352.~{\bf Marshall Perrin}, Ph.D., Staff Astronomer, {\sl Space Telescope Science Institute} \\
353.~{\bf N. D. B. Connolly}, Associate Professor, {\sl Johns Hopkins University} \\
354.~{\bf Frank Delph}, BA History University of Texas at San Antonio, {\sl retired} \\
355.~{\bf Shawn Kenner}, PhD \\
356.~{\bf Brian Siana}, Professor of Physics and Astronomy, {\sl University of California, Riverside} \\
357.~{\bf Andrew Magyar}, PhD, {\sl Penn State} \\
358.~{\bf Kristina Carswell}, MD, assistant clinical professor, {\sl University of Florida } \\
359.~{\bf Jeff Bary}, Associate Professor, {\sl Colgate} \\
360.~{\bf Eric A. Oches}, Ph.D., Professor, {\sl Bentley University, Waltham, Massachusetts} \\
361.~{\bf Linda Anderson}, High School Physics Teacher, {\sl Everett Public Schools} \\
362.~{\bf Deepak Iyer}, Professor, {\sl Bucknell University} \\
363.~{\bf Rose Perea}, PhD candidate, {\sl Vanderbilt University} \\
364.~{\bf Sidney Vetens}, {\sl Reed College} \\
365.~{\bf Yonel Pierre}, PhD, {\sl Superior Court of New Jersey} \\
366.~{\bf Abigail C Shockley}, PhD \\
367.~{\bf LaRoy Brandt}, PhD, Assistant Professor of Biology, {\sl Truman State University} \\
368.~{\bf Kyle Ferguson}, MS, {\sl University of Michigan} \\
369.~{\bf Christine Waigl}, Dipl.-Phys., {\sl University of Alaska Fairbanks} \\
370.~{\bf Juliet Ashall}, Associate Researcher, {\sl Icahn School of Medicine at Mount Sinai} \\
371.~{\bf Andrea Cardinal}, MFA, {\sl University of Michigan} \\
372.~{\bf Tatsu Monkman}, Student, {\sl Pomona College} \\
373.~{\bf Alexander L. Rudolph}, PhD, Professor of Physics and Astronomy, {\sl Cal Poly Pomona} \\
374.~{\bf Sumner Starrfield}, Regents Professor, {\sl Arizona State University} \\
375.~{\bf Victoria Chia}, MPH, MD Candidate, {\sl University of California San Diego} \\
376.~{\bf Lauren Barth-Cohen}, Research Assistant Professor, {\sl University of Miami} \\
377.~{\bf Hannah Levy} \\
378.~{\bf Sharon Sessions}, PhD, {\sl New Mexico Tech} \\
379.~{\bf Sepehr Vakil}, PhD candidate, Education in Mathematics, Science, and Technology, M.S., UCLA, {\sl UC Berkeley} \\
380.~{\bf Michael G Pacifico}, BA, MT(ASCP), {\sl Retired} \\
381.~{\bf Julie Mitchell}, Professor, {\sl University of Wisconsin - Madison} \\
382.~{\bf Bradford A. Barker}, {\sl University of California, Berkeley} \\
383.~{\bf Sean Larsen}, PhD Mathematics, {\sl Portland State University} \\
384.~{\bf Elizabeth Dresselhaus}, Undergraduate Student, {\sl University of Pennsylvania} \\
385.~{\bf Arthur B. Powell}, PhD, {\sl Rutgers University} \\
386.~{\bf Christopher A. Cline}, Ph.D, {\sl Westminster College (Salt Lake City)} \\
387.~{\bf Trina L. Coleman}, PhD, {\sl Hampton University} \\
388.~{\bf Alex Dzinbal}, NA/ME Engineering Student, {\sl Webb Institute} \\
389.~{\bf Raissa Landor}, Ms., {\sl Wright College} \\
390.~{\bf Alison Annon}, {\sl General Dynamics} \\
391.~{\bf Michael D. Barnes}, Professor, {\sl University of Massachusetts Amherst} \\
392.~{\bf Arjendu K. Pattanayak}, Professor of Physics, {\sl Carleton College} \\
393.~{\bf Jennifer Kirby}, MBA, {\sl Merrill Lynch} \\
394.~{\bf Charles Windolf}, Applied Math/CS Undergrad, {\sl Brown University} \\
395.~{\bf Richelle Teeling-Smith}, PhD, {\sl Marion Technical College} \\
396.~{\bf Emil Chuck}, PhD, Director of Admissions, {\sl School of Dental Medicine, Case Western Reserve University} \\
397.~{\bf Rebekah S Cross}, Research Assistant, {\sl University of Arizona, Department of Physics} \\
398.~{\bf Andrea French}, JD, RN, MS Informatics student (decided against a physics engineering degree in 1989 due to racism), {\sl Lockheed Martin} \\
399.~{\bf Dr. Vicky Scowcroft}, {\sl Carnegie Observatories} \\
400.~{\bf John Hewitt}, Assistant Professor, {\sl University of North Florida} \\
401.~{\bf Valerie Horsley}, PhD, {\sl Yale University} \\
402.~{\bf Carl Ferkinhoff}, Assistant Professor, {\sl Winona State University} \\
403.~{\bf Alex Hagen}, MS, {\sl Pennsylvania State University} \\
404.~{\bf Thomas Rimlinger}, Mr., {\sl University of MD, College Park} \\
405.~{\bf Charlezetta Stokes}, Graduate Student, {\sl Howard University} \\
406.~{\bf Samson Black}, PhD, Mathematics, {\sl Pacific Northwest College of Art} \\
407.~{\bf Angela White}, Student, {\sl University of Pennsylvania} \\
408.~{\bf Cameron Hummels}, National Science Foundation Postdoctoral Fellow, {\sl Caltech} \\
409.~{\bf Clayton F. Gardinier}, PhD, {\sl Unemployed} \\
410.~{\bf Maritza Y Duran}, MSW candidate, {\sl University of Michigan} \\
411.~{\bf Siegfried Bleher}, PhD, {\sl Fairmont State University} \\
412.~{\bf Zoe Ann Brown}, PhD, {\sl Retired} \\
413.~{\bf Moses Rifkin}, High School Physics Teacher, {\sl University Prep, Seattle, WA} \\
414.~{\bf Michael Moum}, Civil Engineer and concerned citizen \\
415.~{\bf Melody Schiaffino}, PhD, MPH, {\sl SDSU} \\
416.~{\bf Benjamin Shank}, PhD (Physics), {\sl Thermotron Industries} \\
417.~{\bf Beth Reid}, PhD \\
418.~{\bf Carie Cardamone}, Assistant Professor of Astronomy, {\sl Wheelock College} \\
419.~{\bf Bryan J. Field}, Assistant Professor of Physics, {\sl Farmingdale State College} \\
420.~{\bf Allison Ho}, Student, {\sl Pomona College} \\
421.~{\bf Katherine E. Browne}, Professor, {\sl Colorado State Unversity} \\
422.~{\bf Gary William Hallford}, Student, {\sl San Francisco State University } \\
423.~{\bf Carol Marsh}, MSW, {\sl Retired} \\
424.~{\bf John A McClelland}, PhD, {\sl Retired, University of Richmond} \\
425.~{\bf John Johnson}, Professor, {\sl Harvard} \\
426.~{\bf Samuel Cohen}, Associate Professor, {\sl University of Oxford} \\
427.~{\bf Stephen Sadler}, PhD \\
428.~{\bf Ronald Leibach}, Graduate Student, {\sl Case Western Reserve University} \\
429.~{\bf Brendan Miller}, Assistant Professor, {\sl College of St. Scholastica} \\
430.~{\bf Mehr Un Nisa}, {\sl University of Rochester} \\
431.~{\bf Amy Gonzalez}, MS, {\sl Field of Education Technology (Physics)} \\
432.~{\bf Florence van Tulder}, MSc candidate, {\sl Oregon State University} \\
433.~{\bf Andrew Eichmann}, Research Scientist, {\sl NASA Goddard Space Flight Center} \\
434.~{\bf Keith Clay}, PhD, {\sl Green River College} \\
435.~{\bf Aungshuman Zaman}, {\sl Stony Brook University} \\
436.~{\bf Kevin Flaherty}, PhD, {\sl Wesleyan University} \\
437.~{\bf Eliza Kempton}, Professor, {\sl Grinnell College} \\
438.~{\bf Jillian Bellovary}, Helen Postdoctoral Fellow, {\sl American Museum of Natural History} \\
439.~{\bf Heather Bloemhard}, {\sl AAS} \\
440.~{\bf Borislava Bekker}, PhD, {\sl Hartnell College} \\
441.~{\bf John G Laramee} \\
442.~{\bf Cary Eggerling}, BA, {\sl In my mind} \\
443.~{\bf Jenna Walrath}, PhD candidate, {\sl University of Michigan} \\
444.~{\bf Obioma Ohia}, PhD, {\sl Maryland} \\
445.~{\bf DeShawna Williams}, QMHS, {\sl The Children's Home of Cincinnati } \\
446.~{\bf Shibu Mathukutty}, MS \\
447.~{\bf Tony Sevold}, BSc, {\sl Metabolic Foundation} \\
448.~{\bf Joshua H. Gordis}, Professor \\
449.~{\bf Miriam Simpson}, Professor, {\sl Cuyamaca College} \\
450.~{\bf Alan C. Brodrick}, Artist, {\sl Lancaster , Co. Pa.} \\
451.~{\bf Zahra Khan}, Student, {\sl Rutgers University} \\
452.~{\bf Brenda Newman}, MD, PhD, {\sl University of Washington } \\
453.~{\bf Adam Coogan}, Graduate student, {\sl UC Santa Cruz} \\
454.~{\bf Elizabeth Goldschmidt}, PhD, {\sl National Institute of Standards and Technology} \\
455.~{\bf Aneesha Badrinarayan} \\
456.~{\bf Justin Storbeck}, PhD, {\sl Queens College, CUNY} \\
457.~{\bf Ms Frances J Poodry}, Director of Physics, {\sl Vernier Software \& Technology} \\
458.~{\bf Humberto Gilmer}, MS Student, {\sl The Ohio State University} \\
459.~{\bf Geoffrey Blake}, Professor, {\sl California Institute of Technology} \\
460.~{\bf Jennifer L. Madison}, PhD \\
461.~{\bf Robin Bjorkquist}, Graduate Student, {\sl Cornell University} \\
462.~{\bf George Furbish}, PhD Student, {\sl City University of New York} \\
463.~{\bf Edgar Hernandez}, Masters of Social Work, {\sl Government} \\
464.~{\bf Angela Little}, PhD, {\sl Michigan State University} \\
465.~{\bf Neil Snepp}, MS, {\sl CGG} \\
466.~{\bf Casey Berger}, Graduate Student, {\sl The University of North Carolina at Chapel Hill} \\
467.~{\bf Max Liboiron}, PhD, {\sl Memorial University of Newfoundland } \\
468.~{\bf Malik S. Henfield}, PhD, {\sl University of San Francisco} \\
469.~{\bf Aleksandra Safonova}, Research Assistant, {\sl University of Arizona} \\
470.~{\bf Teresa Brecht}, MS, {\sl Yale University } \\
471.~{\bf Margaret Gardel}, Professor, {\sl University of Chicago} \\
472.~{\bf David Ferguson}, Physicist, {\sl Northrop Grumman Advanced Technology Laboratory} \\
473.~{\bf Cynthia Schwartzberg Edlow}, MA, CP., {\sl Poet, Certified Paralegal, Emerita.} \\
474.~{\bf Jeremy Ahouse}, PhD, Biophysics \& Immunology \\
475.~{\bf Ammar Husain}, {\sl UC Berkeley} \\
476.~{\bf David Pooley}, Professor, {\sl Trinity University} \\
477.~{\bf Eleanor Cawthon}, STEM PhD student, {\sl University of California, Berkeley} \\
478.~{\bf Risa Wechsler}, Professor, {\sl Stanford University} \\
479.~{\bf William J. Fischer}, PhD, {\sl NASA Goddard Space Flight Center} \\
480.~{\bf Linda Laws}, Ms., {\sl University of Pennsylvania } \\
481.~{\bf Susan D Kost}, PhD, {\sl Cleveland Clinic Foundation} \\
482.~{\bf Kerkira Stockton}, Graduate Students in Nuclear Physics, {\sl University of Washington} \\
483.~{\bf Heather-Lyn Haley}, PhD, {\sl UMass Medical School} \\
484.~{\bf Sharlene Denos}, PhD, {\sl University of Illinois} \\
485.~{\bf Jose Ricardo Correa}, Ph.D, {\sl Austin Community College} \\
486.~{\bf David Brookes}, PhD, Physics, {\sl California State University Chico } \\
487.~{\bf Eileen S. Vergino}, MIT '77, Geophysics, {\sl Retired, Lawrence Livermore National Laboratory (38 years)} \\
488.~{\bf Marlee Chong}, {\sl Harvard University} \\
489.~{\bf James T. Laverty}, PhD, {\sl Michigan State University} \\
490.~{\bf Karyn Watters}, MD \\
491.~{\bf Felix Alfonso}, PhD, {\sl Stanford University} \\
492.~{\bf Mark Gurwell}, Astrophysicist, {\sl Harvard-Smithsonian Center for Astrophysics} \\
493.~{\bf Maricedes Acosta-Martinez}, PhD, Assistant Professor, {\sl Stony Brook University} \\
494.~{\bf Robert H.Bragg}, Professor Emeritus, {\sl Universityof California,Berkeley} \\
495.~{\bf Michael Chini}, Assistant Professor, {\sl University of Central Florida } \\
496.~{\bf Jennifer Weston}, PhD Candidate, {\sl Columbia University} \\
497.~{\bf Tatiana Rodriguez}, PhD \\
498.~{\bf Walter Tangarife}, PhD, {\sl UT Austin and Tel Aviv University} \\
499.~{\bf Sarah Schlotter}, {\sl Harvard University} \\
500.~{\bf Colleen Nyeggen}, High School Physics Teacher, {\sl Lick-Wilmerding High School, San Francisco} \\
501.~{\bf Michael Gutperle}, Professor, {\sl UCLA} \\
502.~{\bf Benjamin Beamer}, MS, {\sl Stony Brook University} \\
503.~{\bf Joel S. Pendery}, PhD, {\sl University of Wisconsin-Madison} \\
504.~{\bf Benjamin Horowitz}, {\sl UC Berkeley} \\
505.~{\bf Aaron Noble}, MS (Stanford), NBCT, {\sl Mercer Island High School, Mercer Island, WA} \\
506.~{\bf Anton Kazakov}, Student, {\sl University of Pittsburgh} \\
507.~{\bf Klaus Pontoppidan}, PhD, {\sl Space Telescope Science Institute} \\
508.~{\bf Marc Edwards}, PhD, {\sl Johns Hopkins} \\
509.~{\bf Rahul Oka}, PhD, {\sl University of Notre Dame} \\
510.~{\bf Danny Kim}, student, {\sl UC Berkeley} \\
511.~{\bf Maxwell Porter}, PhD student, {\sl UT Austin} \\
512.~{\bf Jocelyn Read}, Asst Professor, {\sl California State University Fullerton} \\
513.~{\bf Thomas von Foerster}, PhD, {\sl Retired, American Institute of Physics} \\
514.~{\bf Ron Simons}, BA, Computer Science, Columbia Univ \\
515.~{\bf Helene Flohic}, Professor, {\sl University of the Pacific} \\
516.~{\bf Vincent Wrice}, Professor, {\sl Union County College} \\
517.~{\bf Angela Speck}, Professor, {\sl University of Missouri} \\
518.~{\bf Tamara F Isaacs-Smith}, MS, Research Associate IV, {\sl Auburn University} \\
519.~{\bf Larry Strickler}, Educator, {\sl Iowa education system retired} \\
520.~{\bf Benjamin Stuhl}, PhD, {\sl Space Dynamics Laboratory} \\
521.~{\bf Kristi Bradford}, {\sl Arizona State University} \\
522.~{\bf Anna Zaniewski}, PhD, {\sl Arizona State University} \\
523.~{\bf Steven Drapcho}, PhD Student, {\sl UC Berkeley} \\
524.~{\bf Chi Ho Chan}, PhD candidate, {\sl Johns Hopkins University} \\
525.~{\bf Nikolaos Vergos}, Dipl., {\sl The University of Texas at Austin} \\
526.~{\bf Tage}, {\sl Chicago Symphony Orchestra} \\
527.~{\bf Gurtina Besla}, Professor, {\sl U. Arizona} \\
528.~{\bf Philip Smith}, MS, {\sl Lone Star College} \\
529.~{\bf Gerhardt Meurer}, Professor, {\sl U. Western Australia (former Johns Hopkins University)} \\
530.~{\bf Michael Falk}, Professor of Physics, Materials Science and Engineering, and Mechanical Engineering, {\sl Johns Hopkins University } \\
531.~{\bf Christina C. Williams}, PhD, {\sl University of Arizona} \\
532.~{\bf Christine Rittenhouse}, BS Physics, MA Teaching Science, {\sl Liberty High School (Frisco, TX)} \\
533.~{\bf Ben Keller}, PhD Candidate, {\sl McMaster University} \\
534.~{\bf Patrick McCarter}, MD, PhD candidate, {\sl The University of North Carolina at Chapel Hill} \\
535.~{\bf Eli Parke}, Dr., {\sl University of California, Los Angeles} \\
536.~{\bf Robin Kent}, MS, {\sl HDR, Inc.} \\
537.~{\bf Kevin Liao}, Student of Computer Science, {\sl Haverford College} \\
538.~{\bf Stephen Swartz}, PhD \\
539.~{\bf Dennis Lo}, PhD, {\sl County College of Morris} \\
540.~{\bf Takamitsu Tanaka}, PhD, {\sl Stony Brook University} \\
541.~{\bf Kolja Kauder}, PhD, {\sl Wayne State} \\
542.~{\bf Peter Schnatz}, {\sl The City College of New York} \\
543.~{\bf Ethan Contini-Field}, Physics Educator: B.A. Physics, CWRU.; MEd UMass Boston, {\sl Harvard University} \\
544.~{\bf Helen Jackson}, PhD, {\sl Wright State University } \\
545.~{\bf Douglas Cowen}, Professor of Physics, {\sl Penn State} \\
546.~{\bf Mary V. Shaw}, MS, {\sl retired} \\
547.~{\bf Maria Luisa Pineda}, PhD, {\sl CEO of Envisagenics, Inc. Cold Spring Harbor Laboratory } \\
548.~{\bf Gregory Mosby, Jr.}, {\sl University of Wisconsin-Madison} \\
549.~{\bf James Aguirre}, Associate Professor, {\sl University of Pennsylvania} \\
550.~{\bf Michael Richman}, PhD, {\sl Drexel} \\
551.~{\bf Chelsea Harris}, MA Astronomy \& Astrophysics, {\sl University of California Berkeley} \\
552.~{\bf Nicole J Moore}, PhD, {\sl Gonzaga University } \\
553.~{\bf Danae Polychroni}, PhD \\
554.~{\bf Abdelrahman Jarada}, PhD, {\sl the university of arizona} \\
555.~{\bf Catherine Klauss}, {\sl JILA} \\
556.~{\bf Melodie Nguyen} \\
557.~{\bf Dr. Thomas Brantseg}, {\sl Iowa State University} \\
558.~{\bf James Rhoads}, Professor, {\sl Arizona State University} \\
559.~{\bf Ashley Villar}, PhD Student, {\sl Harvard} \\
560.~{\bf Emily L. Rice, PhD}, Assistant Professor, {\sl College of Staten Island, City University of New York} \\
561.~{\bf Audra K. Hernandez}, PhD, Research Scientist, {\sl University of Wisconsin-Madison} \\
562.~{\bf Kate Fitzgerald Groby}, LCSW, {\sl Private Practice} \\
563.~{\bf Anthony Oliver}, MT (ASCP)CM, {\sl Blue Ridge Regonal Hospital} \\
564.~{\bf Ellen Mink}, Professor, {\sl KCTCS} \\
565.~{\bf Susan Fischer Wilhelm}, PhD, M.S.E, {\sl State of California, Energu Commission} \\
566.~{\bf Ayana Arce}, Assistant Professor of Physics, {\sl Duke University} \\
567.~{\bf Kamen Todorov}, PhD, {\sl ETH Zurich} \\
568.~{\bf Tuguldur Sukhbold}, PhD candidate, {\sl University of California Santa Cruz} \\
569.~{\bf Adam Smercina}, PhD Student, {\sl University of Michigan} \\
570.~{\bf Duncan Farrah}, PhD, {\sl Virginia Tech} \\
571.~{\bf Bruce Barrett Banner}, Mathematics and Science Teacher (BS Mathematics) \\
572.~{\bf Bhima Nitta}, PhD, {\sl Power Guru} \\
573.~{\bf TR Hummer}, Professor, {\sl Arizona State University} \\
574.~{\bf Brendesha Tynes}, PhD, {\sl University of Southern California} \\
575.~{\bf Robert E. E. Maltby IV} \\
576.~{\bf Gina M. Quan}, {\sl University of Maryland, College Park} \\
577.~{\bf Randy Skaggs}, Fine Arts Chairman, {\sl St. Joseph School} \\
578.~{\bf Miranda}, MS, Computer Science \\
579.~{\bf Linda Jeschofnig}, Founder \& Chairman, {\sl Hands-On Labs, Inc.} \\
580.~{\bf David McAvity}, PhD, {\sl The Evergreen State College} \\
581.~{\bf Philip Phillips}, Professor, {\sl University of Illinois-Urbana-Champaign} \\
582.~{\bf Liz McDowell}, PhD, {\sl Kalamazoo College} \\
583.~{\bf Tanay Bhandarkar}, Undergraduate Student, {\sl Columbia University } \\
584.~{\bf Lawrence A. Husick}, Adjunct Faculty, {\sl The Johns Hopkins University Whiting School of Engineering} \\
585.~{\bf Patrick Rodriguez}, {\sl Fedex Courier} \\
586.~{\bf Jamie Orr}, PhD, {\sl California Community Colleges Chancellor's Office} \\
587.~{\bf Jacquelyn Chini}, Professor, {\sl University of Central Florida } \\
588.~{\bf Carolyn Sutton}, Director, {\sl Ruby's Academy for Health Occupations} \\
589.~{\bf Megan Bang}, Associate Professor, {\sl University of Washington} \\
590.~{\bf Walter Nogay}, MEd, physics teacher, {\sl Fairfax Ccounty Public Schools} \\
591.~{\bf Rachel Bezanson}, PhD, {\sl University of Arizona} \\
592.~{\bf Michael Murray}, Professor, {\sl University of Kansas} \\
593.~{\bf Sandra Perez}, Minister, {\sl NYC Mennonite Oversight Minister} \\
594.~{\bf Nancy E. Soule}, Ms, {\sl RSU 68/Maine} \\
595.~{\bf Gwendolyn Pettway}, PhD \\
596.~{\bf Maria Christoff}, MA, {\sl University of Detroit Mercy} \\
597.~{\bf Meenakshi Narain}, Professor, {\sl Brown University} \\
598.~{\bf Matthew Stevans}, MS, {\sl Univ. of Texas - Austin} \\
599.~{\bf Howard L. Smith}, PhD, {\sl University of Texas San Antonio } \\
600.~{\bf Saul Aryeh Kohn}, MPhys, {\sl University of Pennsylvania} \\
601.~{\bf Wesley Gohn}, PhD, {\sl University of Kentucky} \\
602.~{\bf Kevin Fischer}, PhD, {\sl Stanford University} \\
603.~{\bf Gerard Marandino}, MS, Astronomy, U MD, {\sl Retired} \\
604.~{\bf Scott Sahlman}, MS, {\sl NASA} \\
605.~{\bf Danika Painter}, PhD, {\sl Health Canada} \\
606.~{\bf John Gizis}, Professor, {\sl University of Delaware} \\
607.~{\bf Michaela Levin}, PhD in Neuroscience, {\sl Harvard Boston Children's Hospital} \\
608.~{\bf Jennifer Barnes}, Graduate Student, {\sl UC Berkeley} \\
609.~{\bf Jonathan Williams}, {\sl Natural History Museum of Los Angeles} \\
610.~{\bf Dawn Anderson}, Teacher, {\sl Public schools} \\
611.~{\bf Elizabeth Golovatski}, PhD, Assistant Professor of Physics, {\sl Central College} \\
612.~{\bf Ev Lunning}, Asst. Prof. Theater Arts, {\sl St. Edward's University} \\
613.~{\bf Brian Lewis}, J.D., {\sl Self Employed} \\
614.~{\bf Sergio Ballestrero}, Research Associate, {\sl CERN, Switzerland \& University of Johannesburg, South Africa} \\
615.~{\bf Peter Gao}, PhD candidate, {\sl Caltech} \\
616.~{\bf Haris A. Durrani}, B.S./MPhil, {\sl Columbia University/University of Cambridge} \\
617.~{\bf Jason Kowalski}, PhD, {\sl UW Parkside} \\
618.~{\bf Margaret A. Jacobson}, MD, {\sl Peacehealth St. Joseph Medical Center} \\
619.~{\bf Kari Jackson}, Master Student, {\sl UIUC} \\
620.~{\bf Wegene}, MS in Electrical Engineering and Computer Science, {\sl MIT, Microsoft} \\
621.~{\bf Allison Towner}, Ms., {\sl University of Virginia} \\
622.~{\bf Shohini Ghose}, Associate Professor, {\sl Wilfrid Laurier University} \\
623.~{\bf Wangui Hymes}, student, {\sl Spelman} \\
624.~{\bf Kyle Godbey}, Graduate Student, {\sl Vanderbilt University} \\
625.~{\bf Krystle McLaughlin}, PhD, Professor of Practice, {\sl Lehigh University } \\
626.~{\bf James Tucci}, PhD Candidate, {\sl Purdue University} \\
627.~{\bf Christopher Morrison}, PhD, {\sl Argelander-Institut für Astronomie} \\
628.~{\bf Julian Aptowitz}, HS Physics Teacher, {\sl Half Hollow Hills HS West } \\
629.~{\bf Adam Odeh}, Student, {\sl University of Arizona} \\
630.~{\bf Stephanie Daniels}, MEd, {\sl Texas State University} \\
631.~{\bf Howard L. Davidson}, PhD, {\sl Facebook } \\
632.~{\bf Melissa Graham}, PhD, {\sl UC Berkeley} \\
633.~{\bf Darryll Pines}, PhD, Professor and Dean of Engineering, {\sl University of Maryland} \\
634.~{\bf Julianna Moats}, MS, {\sl WSP | Parsons Brinckerhoff } \\
635.~{\bf Tom McClintock}, {\sl University of Arizona} \\
636.~{\bf Daniel Ucko}, PhD, {\sl American Physical Society} \\
637.~{\bf Denise Sands Baez}, Ms., {\sl Lehman College/CUNY} \\
638.~{\bf Guy G. Marcus}, Graduate Research Fellow, {\sl Institute for Quantum Matter, Johns Hopkins University} \\
639.~{\bf Stanford Amos}, Mr., {\sl SOS Consulting } \\
640.~{\bf Ashley J. Ross}, PhD, {\sl Ohio State University} \\
641.~{\bf Alexandre Sousa}, Assistant Professor of Physics, {\sl University of Cincinnati} \\
642.~{\bf Tamara Y. Washington}, MSEE, North Carolina A\&T State University, {\sl Engineer} \\
643.~{\bf Jean Turner}, Professor, {\sl UCLA Physics \& Astronomy} \\
644.~{\bf Trina L. Coleman}, PhD, {\sl Hampton University } \\
645.~{\bf Thomas O. Walton}, Doctoral Student, {\sl University of Washington} \\
646.~{\bf Robert Bowden}, Professor Emeritus, {\sl Virginia Tech} \\
647.~{\bf Joshua Parker}, PhD, {\sl US Army Corps of Engineers} \\
648.~{\bf Nathan A. Quarderer}, Science \& Math Instructor, {\sl Northeast Iowa Community College} \\
649.~{\bf Sean Moran}, PhD \\
650.~{\bf Megan Marshall}, Graduate Student, {\sl University of Maryland} \\
651.~{\bf Aaron Parsons}, Professor, {\sl University of California, Berkeley} \\
652.~{\bf Andrea Maccio}, Associate professor of physics, {\sl New york university} \\
653.~{\bf Joelle Labastide}, PhD, {\sl UMass Amherst} \\
654.~{\bf Sean Johnson}, PhD candidate, {\sl The University of Chicago} \\
655.~{\bf Jeanette Bushnell}, PhD, {\sl University of Washington. Seattle} \\
656.~{\bf Melinda Soares-Furtado}, {\sl Princeton University} \\
657.~{\bf Marce Abare}, MD, MPH, {\sl Albert Einstein/Montefiore Medical center} \\
658.~{\bf David Gierlach}, Musician \\
659.~{\bf Kimberly Palladino}, PhD, {\sl University of Wisconsin-Madison} \\
660.~{\bf Tim Atherton}, Assistant Professor of Physics, {\sl Tufts University } \\
661.~{\bf Marvin E. Schechter}, Attorney, {\sl Solo Practioner} \\
662.~{\bf Yadira Bribiesca}, PostBacc Student, {\sl LAUSD} \\
663.~{\bf William Mayer}, Mr, {\sl City College of New York} \\
664.~{\bf Moumita Das}, Assistant Professor, {\sl Rochester Institute of Technology} \\
665.~{\bf Kurt Brendlinger}, MS, {\sl University of Pennsylvania} \\
666.~{\bf Duncan Tate}, Professor, {\sl Colby College} \\
667.~{\bf Kaushala Bandara}, PhD \\
668.~{\bf Benjamin P. Brau}, Associate Profesor, {\sl University of Massachusetts, Amherst} \\
669.~{\bf Shanna Brown} \\
670.~{\bf Tina Le}, PhD, {\sl First Rate, Inc.} \\
671.~{\bf Rita Falbel}, {\sl Retired from Weill Cornell Medical College} \\
672.~{\bf Gareth Roberg-Clark}, Graduate Assistant, {\sl University of Maryland, College Park} \\
673.~{\bf Sumitro Huff}, MiT, {\sl Rainier Scholars} \\
674.~{\bf Milo Korman}, PhD candidate, {\sl Case Western Reserve University} \\
675.~{\bf Musa Campbell}, Medical Technologist (ASCP), {\sl Univ of Cincinnati Medical Center} \\
676.~{\bf David Neilson}, PhD, {\sl Bell Labs} \\
677.~{\bf K. Vlasoff} \\
678.~{\bf Elizabeth Agee}, BS Physics, {\sl University of Michigan} \\
679.~{\bf Richard Owens}, PhD, {\sl Issara institute} \\
680.~{\bf Laura Gray}, PhD, {\sl Princeton University} \\
681.~{\bf Daniel Stolarski}, PhD, {\sl CERN} \\
682.~{\bf Hallie Trauger}, MS Physics, MEd Secondary Science, {\sl University of Illinois at Chicago} \\
683.~{\bf Aaron Santos}, PhD, {\sl Simpson College} \\
684.~{\bf Jon Jenkons}, PhD, {\sl NASA Ames Research Center} \\
685.~{\bf Seth Zenz}, PhD, {\sl Imperial College London} \\
686.~{\bf Edmond Cheung}, PhD, {\sl Kavli IPMU} \\
687.~{\bf Tere Williams}, MS Biomedical Science, PhD candidate, {\sl Albert Einstein College of Medicine} \\
688.~{\bf Prash Sesh}, MS, {\sl UC Los Angeles} \\
689.~{\bf John E. Sohl}, Full Professor of Physics, {\sl Weber State University} \\
690.~{\bf Penelope Foran}, Observer, {\sl In my neighborhood.} \\
691.~{\bf Olivia D}, {\sl Target Corporation } \\
692.~{\bf Randall Kamien}, Professor, {\sl University of Pennsylvania } \\
693.~{\bf James H Price}, Professor, {\sl Texas Wesleyan University} \\
694.~{\bf Ian Roederer}, PhD, {\sl University of Michigan} \\
695.~{\bf Georgianna Kates}, MD, {\sl University Hospitals of Cleveland} \\
696.~{\bf David Tsang}, PhD, {\sl University of Maryland} \\
697.~{\bf Allyn French, PhD}, Professor, {\sl Lakeland College} \\
698.~{\bf Pamela Knox}, MS, {\sl University of Georgia} \\
699.~{\bf Edward Krohne}, MS, {\sl University of North Texas Department of Mathematics} \\
700.~{\bf Ira Holston}, Math and CS Teacher, MS Geology, {\sl Berkeley High School} \\
701.~{\bf Karen L. Thomas}, MS, {\sl Principal, William W. Bodine High School for International Affairs} \\
702.~{\bf Alison Bauer}, Elementary school teacher, {\sl Fairfax County Public Schools} \\
703.~{\bf Carolynn Moore}, PhD, {\sl U.S. Army} \\
704.~{\bf Mary E King}, MA, {\sl University of Washington} \\
705.~{\bf Michele Limon}, Research Associate, {\sl Columbia University} \\
706.~{\bf Daniel Holzman-Tweed}, M.S., {\sl North Shore LIJ} \\
707.~{\bf Joshua Bridger}, High School Physics/Astronomy teacher \\
708.~{\bf Michael J.Nolan}, Professor of Physics, PhD, {\sl Millersville University} \\
709.~{\bf Gautham Narayan}, PhD, Postdoctoral Associate, {\sl National Optical Astronomy Observatory \& The University of Arizona} \\
710.~{\bf Richard Hallstein}, Instructor, {\sl Michigan State University} \\
711.~{\bf Austin Reid}, MS, {\sl North Carolina State University} \\
712.~{\bf Frank Morton-Park}, MA \\
713.~{\bf Philip Choi}, Associate Professor of Physics and Astronomy, {\sl Pomona College} \\
714.~{\bf John C. Hood II}, {\sl Vanderbilt University} \\
715.~{\bf Jacob Wacker}, PhD Physics, {\sl Quora, Stanford University} \\
716.~{\bf Lee Walsh}, PhD candidate, {\sl University of Massachusetts} \\
717.~{\bf Nick McGreivy}, Physics major and aspiring physicist, {\sl University of Pennsylvania} \\
718.~{\bf Thad Szabo}, Assistant Professor, {\sl Cerritos College} \\
719.~{\bf Charemi A Jones}, PhD candidate, {\sl City of Chicago} \\
720.~{\bf M. Cather Simpson}, Associate Professor, {\sl The University of Auckland} \\
721.~{\bf Niels Damrauer}, Associate Professor (Physical Chemistry not physics but same issues), {\sl University of Colorado Boulder} \\
722.~{\bf Jennifer Silverman}, Mathematics Teacher and Inventor, {\sl ProRadian Protractors} \\
723.~{\bf Orion Sauter}, {\sl University of Michigan} \\
724.~{\bf David Atkinson}, Graduate Student, {\sl UMass Amherst} \\
725.~{\bf Maarten Golterman}, professor and chair, {\sl San Francisco State University} \\
726.~{\bf R.Wijewardhana}, PhD, {\sl Univ of Cincinnati} \\
727.~{\bf Chad Bieber}, PhD \\
728.~{\bf Andrew Svesko}, MS, PhD candidate, {\sl Arizona State University} \\
729.~{\bf Paul T. Winfield}, PhD, {\sl WOFN, Non-Profit} \\
730.~{\bf Omar Khatib}, PhD, {\sl CU Boulder} \\
731.~{\bf Amy Hee Kim}, Ph.D, {\sl Iridescent (science education nonprofit )} \\
732.~{\bf Bob Maltby}, President R and D Reflections Inc.c, {\sl R and D Reflections} \\
733.~{\bf Kieran Sweeney}, Student, {\sl The Pennsylvania State University, Aerospace Engineering Dept.} \\
734.~{\bf Rachel Williams}, Energy Specialist, {\sl The Weidt Group} \\
735.~{\bf David Bruhwiler}, PhD, {\sl RadiaSoft LLC} \\
736.~{\bf Mary Ahmann}, Faculty, {\sl Minneapolis Community and Technical College} \\
737.~{\bf Erin Quealy}, PhD, {\sl Napa Valley College} \\
738.~{\bf Worokya Duncan, EdD}, Director of Diversity, {\sl The Cathedral School of St. John the Divine} \\
739.~{\bf Merry Dearmon-Moore}, R.N.M.S., {\sl AZ Community Physicisns} \\
740.~{\bf Christopher M. Nakamura}, Assistant of Professor of Physics, {\sl Saginaw Valley State University} \\
741.~{\bf Jenna Ross}, Human \\
742.~{\bf Kelsey Hallinen}, PhD student, {\sl University of Michigan} \\
743.~{\bf Maria C Simani}, PhD, {\sl University of California Riverside} \\
744.~{\bf Elizabeth Wehner}, PhD \\
745.~{\bf Matthew C. Patterson}, PhD, {\sl Louisiana State University} \\
746.~{\bf Ron Shufflebarger}, world citizen, {\sl retired furniture artisan} \\
747.~{\bf David Winogradoff}, PhD, {\sl University of Illinois at Urbana-Champaign} \\
748.~{\bf Ashoordin Ashoormaram}, MS, {\sl University of Arizona} \\
749.~{\bf Patrick Zabawa}, PhD \\
750.~{\bf Joseph Harder}, PhD Candidate, {\sl Columbia University} \\
751.~{\bf Brandon M. Peden}, Dr., {\sl Dept. of Physics and Astronomy, Western Washington University, Bellingham WA} \\
752.~{\bf Mary Robinson}, MS, {\sl Retired} \\
753.~{\bf Karin Sandstrom}, Assistant Professor, {\sl UC San Diego} \\
754.~{\bf Perry Rice}, Professor, {\sl Miami University} \\
755.~{\bf Shirley Ho}, Professor, {\sl Carnegie Mellon University} \\
756.~{\bf Brianna Nicole Thorpe}, {\sl Arizona State University} \\
757.~{\bf Max Genecov}, Grad Student, {\sl UC Berkeley Astro} \\
758.~{\bf Maurice Champagne}, PhD \\
759.~{\bf Urmila Chadayammuri}, Graduate Student, {\sl Yale University } \\
760.~{\bf Paul Eichhorn}, BS Economics, {\sl Retired.} \\
761.~{\bf Sarah Heim}, PhD, {\sl University of Pennsylvania} \\
762.~{\bf Brooke Haag}, PhD, {\sl Harvard Graduate School of Education} \\
763.~{\bf Jordan James Gosselin}, Graduate Student, {\sl UC San Diego} \\
764.~{\bf Adrien Thob}, MS, {\sl Liverpool John Moores University} \\
765.~{\bf Mark Silcox}, PhD., {\sl University of Central Oklahoma} \\
766.~{\bf Brea Duval}, PhD, {\sl Novozymes North America} \\
767.~{\bf Dr. Benjamin Weiner}, {\sl University of Arizona, Steward Observatory} \\
768.~{\bf Colin Fredericks}, PhD, {\sl Harvard} \\
769.~{\bf Ta-Shana Taylor}, MS in Geosciences, {\sl University of Miami} \\
770.~{\bf Donald Smith}, Associate Professor, {\sl Guilford College} \\
771.~{\bf Damien Benveniste}, PhD, {\sl EMC Corp} \\
772.~{\bf Jacques Lipetz}, Ph.D, {\sl TWC Psychologica Svcs} \\
773.~{\bf Andrew A Cooper}, Teaching Assistant Professor of Mathematics, {\sl North Carolina State University} \\
774.~{\bf Krishna Chowdary}, PhD, Member of the Faculty (Physics \& Math), {\sl The Evergreen State College} \\
775.~{\bf Marios Karouzos}, Postdoctoral Researcher, {\sl Seoul National University} \\
776.~{\bf Gustavo del Castillo}, Dr. Nuclear Physics, {\sl MURA} \\
777.~{\bf Cindy Ok}, High School Physics Teacher, {\sl Manual Arts (LAUSD)} \\
778.~{\bf Ehsan Khatami}, Assistant Professor, {\sl San Jose State University} \\
779.~{\bf Alexander Kemper}, Assistant Professor, {\sl North Carolina State University} \\
780.~{\bf Robert Phillips}, Dr., PE, PE, D.Eng, {\sl Retired} \\
781.~{\bf Nance Cedar}, BA, Mathematics, Western Michigan University, {\sl Albuquerque Public Schools} \\
782.~{\bf R. Adriel Vasquez}, MS, {\sl Project Syndicate} \\
783.~{\bf Christina Fitzsimmons}, PhD candidate, {\sl UCSF} \\
784.~{\bf Manuel Alejandro Olmedo Negrete}, PhD candidate, {\sl University of California Riverside } \\
785.~{\bf Rhiannon Meharchand}, PhD \\
786.~{\bf Lisa Storrie-Lombardi}, PhD, {\sl Caltech} \\
787.~{\bf Iris Shiver}, M.B.A. \\
788.~{\bf Catherine Bergeron}, High School Physics Teacher, {\sl Philadelphia} \\
789.~{\bf Marshall C. Johnson}, {\sl University of Texas at Austin} \\
790.~{\bf Sherman Ragland}, MBA, CCIM \\
791.~{\bf Sara Petty}, PhD, {\sl Virginia Tech} \\
792.~{\bf Ilana Percher}, PhD Candidate, {\sl University of Minnesota} \\
793.~{\bf Chris Reilly}, Graduate Student, Teaching Assistant, {\sl University of Texas at Austin} \\
794.~{\bf Joseph Chatelain}, {\sl Georgia State University} \\
795.~{\bf Marisa Litz}, MS, {\sl Oregon State University} \\
796.~{\bf Hugo Guerrero-Cazares}, MD PhD, {\sl Johns Hopkins University} \\
797.~{\bf Hyman Bass}, Distinguished Univ. Prof. of Mathematics \& Mathematics Education, {\sl University of Michigan} \\
798.~{\bf Robert Hobbs}, Tenured Faculty, {\sl Belleveue College} \\
799.~{\bf Leo Piilonen}, Professor, {\sl Virginia Tech} \\
800.~{\bf Brandon Hensley}, PhD, {\sl JPL/Caltech} \\
801.~{\bf Daniel Barringer}, MS, {\sl Penn State} \\
802.~{\bf William Wilkin}, {\sl Harvard University} \\
803.~{\bf Philippe Lewalle}, PhD Student (physics), {\sl University of Rochester} \\
804.~{\bf William J.Mullin}, Professor Emeritus, {\sl University of Massachusetts} \\
805.~{\bf Kerry Hensley}, Graduate student, {\sl Boston University} \\
806.~{\bf Donald Ford}, MD, MBA, {\sl Cleveland Clinic} \\
807.~{\bf Denise M. McLinden}, M. A., {\sl Retired} \\
808.~{\bf Sandra Simmons}, Licensed Clinical Social Worker, {\sl Oakland Unified School District} \\
809.~{\bf Larissa Rodriguez}, Professor, {\sl USC} \\
810.~{\bf Carole Mundell}, Professor, {\sl University of Bath} \\
811.~{\bf Yvette Cendes}, PhD candidate, {\sl University of Amsterdam} \\
812.~{\bf Tracy D. Berman}, PhD, {\sl University of Michigan} \\
813.~{\bf Ashley R. Carter}, Assistant Professor of Physics, {\sl Amherst College} \\
814.~{\bf Sara Callori}, Assistant Professor, {\sl California State University, San Bernardino} \\
815.~{\bf Jean Welsh}, MD, {\sl Private practice } \\
816.~{\bf Annette A. Angus}, PhD, {\sl UC Berkeley} \\
817.~{\bf Shawn Ward}, {\sl Rutgers University} \\
818.~{\bf Donald Terndrup}, Associate Professor, {\sl Ohio State University} \\
819.~{\bf Carl Benander}, President, {\sl Boston Atlantic Corp} \\
820.~{\bf Mechie Nkengla}, PhD \\
821.~{\bf Cady McElheny}, Student, {\sl Loyola University Chicago } \\
822.~{\bf Merideth Frey}, PhD, {\sl Princeton University} \\
823.~{\bf Danielle Sipple}, Student, {\sl Anoka-Ramsey Community College} \\
824.~{\bf Nicole Carlson}, PhD, {\sl Shiftgig} \\
825.~{\bf Marcus Aguilar}, BA, MS, MBA, {\sl MC2 Energy } \\
826.~{\bf Gregory Feiden}, PhD, {\sl Uppsala University} \\
827.~{\bf David Kastor}, Senior Lecturer, {\sl UMass, Amherst} \\
828.~{\bf Todd Coleman}, Professor, {\sl Saint Paul College} \\
829.~{\bf Sara Lucatello}, PhD, {\sl INAF Padova Astronomical Observatory, Italy} \\
830.~{\bf Frank J. Giuliano}, PhD, {\sl Westfield State University} \\
831.~{\bf David L. Graham}, PhD, Professor, Ret., {\sl Iowa State Univ., Cornell Univ., Texas A.\&M. Univ.} \\
832.~{\bf Kristen Engle}, MD, {\sl Self employed} \\
833.~{\bf Michele Thornley}, Professor, {\sl Bucknell University } \\
834.~{\bf Xiangcheng Ma}, {\sl Caltech} \\
835.~{\bf Carolyn Greenlee}, BS - UW- Madison, {\sl Retired} \\
836.~{\bf Edozie C. Edoga}, Robotics Instructor and FIRST Team Lead Mentor, {\sl High School} \\
837.~{\bf Neil Aaronson}, Ph.D., {\sl Stockton University} \\
838.~{\bf Michael Occhino}, Director of Science Education Outreach, {\sl University of Rochester} \\
839.~{\bf Dallas Trinkle}, Professor of materials science, {\sl University of Illinois, Urbana-Champaign } \\
840.~{\bf Anand Gnanadesikan}, Associate Professor of Earth and Planetary Sciences, {\sl Johns Hopkins University} \\
841.~{\bf Yuka Asada}, PhD, {\sl UIC} \\
842.~{\bf Onyinye I. Iweala}, MD PhD, {\sl UNC Hospitals Chapel Hill, NC} \\
843.~{\bf Corbette Doyle}, Lecturer, {\sl Vanderbilt University} \\
844.~{\bf Sami Atif}, PhD, {\sl Phillips Exeter Academy } \\
845.~{\bf Giordon Stark}, Physics PhD candidate, {\sl University of Chicago} \\
846.~{\bf John C. Forbes}, {\sl University of California, Santa Cruz} \\
847.~{\bf Charles D. Allen, Esq.}, Jurist Doctorate, {\sl Advocates for Social \& Economic Justice} \\
848.~{\bf Kevin P. Davis}, PhD \\
849.~{\bf Susan Patrick}, Retired R.N, {\sl Was employed @ UAB in Birmingham, Al.} \\
850.~{\bf J. Nathan Scott}, Ph.D., {\sl Saint Francis University} \\
851.~{\bf Shawn Staudaher}, PhD Candidate, {\sl University of Wyoming} \\
852.~{\bf Eric Brewe}, Associate Professor, {\sl Florida International University} \\
853.~{\bf Lila Roberts}, PhD, Professor and Dean of the College of Information and Mathematical Sciences, {\sl Clayton State University} \\
854.~{\bf Elias Burstein}, Professor of  Physics Emeritus, {\sl University of Pennsylvania} \\
855.~{\bf Elizabeth Blanton}, PhD, Associate Professor, {\sl Boston University} \\
856.~{\bf Gerrick Lindberg}, Assistant Professor, {\sl Northern Arizona University} \\
857.~{\bf Lori Allen}, PhD Astronomy, {\sl National Optical Astronomy Observatory} \\
858.~{\bf Elizabeth Tompkins-Lindauer}, BS, {\sl Schneider Electric} \\
859.~{\bf Okey Enyia}, Scholar-Activist, {\sl Capitol Hill} \\
860.~{\bf Daniel Akerib}, Professor, {\sl SLAC National Accelerator Lab, Stanford University } \\
861.~{\bf Vince Stricherz}, Science writer (retired), {\sl University of Washington, Seattle} \\
862.~{\bf Carlene Neal}, MSW, {\sl Columbia University- class of 75} \\
863.~{\bf Joseph Ribaudo}, Associate Professor, {\sl Utica College} \\
864.~{\bf Vladimir Airapetian}, PhD, {\sl NASA GSFC} \\
865.~{\bf Henry Ngo}, MS, {\sl Caltech} \\
866.~{\bf Eugenia Zacks-Carney}, JD, {\sl American Jet Brokers} \\
867.~{\bf Camille Avestruz}, PhD, {\sl University of Chicago} \\
868.~{\bf Soenke Moeller}, {\sl UC Berkeley} \\
869.~{\bf Steuard Jensen}, Associate Professor of Physics and Associate Provost, {\sl Alma College} \\
870.~{\bf Susan Korbel}, PhD, {\sl Core Research} \\
871.~{\bf Tony Sprinkle}, BS, Physics, {\sl WebAssign} \\
872.~{\bf Leland Muller}, MS, {\sl University of Pennsylvania } \\
873.~{\bf Melissa Gresalfi}, PhD, {\sl Vanderbilt University} \\
874.~{\bf Raymond Adams}, Assistant Professor, {\sl Southern Arkansas University } \\
875.~{\bf Elisabeth Krause}, PhD, {\sl Stanford} \\
876.~{\bf Annie Rak}, AB Candidate in Applied Mathematics, {\sl Harvard College} \\
877.~{\bf Marco Verzocchi}, Scientist II, {\sl Fermi National Accelerator Laboratory} \\
878.~{\bf Janet Steinberg}, MFA, {\sl retired} \\
879.~{\bf Charles Boyle}, Dip Arch - architect, {\sl Papua New Guinea University of Technology} \\
880.~{\bf Sarah Gallagher}, Professor, {\sl University of Western Ontario} \\
881.~{\bf Alex Barr}, Professor, {\sl Howard Community College} \\
882.~{\bf Blake C. Stacey}, PhD, {\sl University of Massachusetts Boston} \\
883.~{\bf Charles Xu}, PhD Student, {\sl Caltech} \\
884.~{\bf LaTonya Hall}, The Computer Scientists, {\sl Vahna, Inc.} \\
885.~{\bf Kate Shaw}, PhD, {\sl International Centre for Theoretical Physics (ICTP)} \\
886.~{\bf Meredith Hughes}, Professor, {\sl Wesleyan University} \\
887.~{\bf Robert Sparks}, M.S. Science Education Specialist, {\sl National Optical Astronomy Observatory} \\
888.~{\bf Michelle Higgins}, MS Physics, {\sl University of Arizona} \\
889.~{\bf N. Joiner}, {\sl Utah State University} \\
890.~{\bf Nick Raykovich}, BS Mathematics / ME Professional Development, {\sl Madison West HIgh School} \\
891.~{\bf Adrian Baur}, PhD \\
892.~{\bf Theron Carmichael}, PhD student, {\sl Harvard University} \\
893.~{\bf Ryan Reece}, PhD, {\sl University of California, Santa Cruz} \\
894.~{\bf Matthew Benacquista}, Professor, {\sl University of Texas Rio Grande Valley} \\
895.~{\bf Keith Hawkins}, Marshall Scholar, {\sl University of Cambridge} \\
896.~{\bf Claire Bellis}, PhD, {\sl GIS, Singapore} \\
897.~{\bf Vanessa Ferrel}, B.S.; M.D. Candidate, {\sl University of California San Diego} \\
898.~{\bf Molly Peeples}, PhD, {\sl Space Telescope Science Institute} \\
899.~{\bf Willa N. France}, Naval Arcitect and Marine Engineer (BSc); Attorney at Law (JD), {\sl Self-employed consultant } \\
900.~{\bf Danielle Orr}, MHA, {\sl Hospital} \\
901.~{\bf Richard Zallen}, Professor, {\sl Virginia Tech} \\
902.~{\bf Kimberly Seashore}, Assistant Professor, {\sl San Francisco State University} \\
903.~{\bf David Hanneke}, Assistant Professor of Physics, {\sl Amherst College} \\
904.~{\bf Kannan Jagannathan}, Professor, {\sl Amherst College} \\
905.~{\bf Michael Kemp}, BS Math, {\sl Owner of HCS Group, Inc} \\
906.~{\bf Kendall Mahn}, Assistant Professor, {\sl Michigan State University} \\
907.~{\bf Lew Riley}, Professor, {\sl Ursinus College} \\
908.~{\bf Micah Johnson}, PhD, Research Scientist, {\sl Lawrence Livermore National Laboratory } \\
909.~{\bf Roman Kossak}, Professor of Mathematics, {\sl City University of New York} \\
910.~{\bf Janine Pforr}, PhD, {\sl Laboratoire d'Astrophysique de Marseille, France} \\
911.~{\bf Arthur Russakoff}, {\sl Vanderbilt University} \\
912.~{\bf Stephen Kanim} \\
913.~{\bf Jessamyn Fairfield}, Research Fellow, {\sl Trinity College Dublin} \\
914.~{\bf Cynthia Plank}, MS, {\sl Beaumont Middle School} \\
915.~{\bf Alyssa Barlis}, Graduate Student, {\sl University of Pennsylvania} \\
916.~{\bf Stephanie Majkut}, PhD, {\sl Nath, Goldberg \& Meyer} \\
917.~{\bf Morgan MacLeod}, PhD candidate, {\sl University of California, Santa Cruz} \\
918.~{\bf Shubha Tewari}, Lecturer in Physics, {\sl University of Massachusetts, Amherst} \\
919.~{\bf Imani J. Walker}, Medical Director/Psychiatrist, {\sl Gateways Hospital and Mental Health Services} \\
920.~{\bf Joshua H. Shiode}, PhD \\
921.~{\bf Benjamin Williams}, MS Physics, {\sl M-A High School} \\
922.~{\bf Anna Rosen}, PhD candidate, {\sl UCSC} \\
923.~{\bf Cassandry Redmond Keys}, Former Physics Student, {\sl Christian Methodist Episcopal Church } \\
924.~{\bf Brandon Rodenburg}, PhD, {\sl Rochester Institute of Technology} \\
925.~{\bf Lindia Willies-Jacobo}, MD-Professor of Pediatrics, {\sl UCSD} \\
926.~{\bf Amanda Moffett}, PhD, {\sl University of Western Australia} \\
927.~{\bf Kevin Freymiller}, {\sl Reed College} \\
928.~{\bf Bara Safarova}, PhD Student in Architecture, {\sl Texas A\&M University} \\
929.~{\bf Kristiana Schneck}, PhD, {\sl Pandora (formerly SLAC National Accelerator Laboratory)} \\
930.~{\bf Thomas Strauss}, PhD, Associate Scientist, {\sl Fermi National Accelerator Laboratory } \\
931.~{\bf Kevin McLin}, PhD, {\sl Sonoma State University} \\
932.~{\bf Georgia Bracey}, Research Associate, {\sl Southern Illinois University Edwardsville} \\
933.~{\bf Janet Hecsh}, PhD and Professor of Education, {\sl California State University, Sacramento} \\
934.~{\bf Donnie Kost}, Secondary Special Education Math Teacher, {\sl Central High School} \\
935.~{\bf Ben Sugerman}, Associate Professor and Chair, {\sl Goucher College} \\
936.~{\bf Sean P. Robinson}, PhD, {\sl MIT} \\
937.~{\bf Duncan Watts}, MA, {\sl Johns Hopkins University} \\
938.~{\bf Kelle Cruz}, Asst Professor, PhD, {\sl CUNY Hunter College} \\
939.~{\bf Jill P. Naiman}, PhD, {\sl Harvard-Smithsonian Center for Astrophysics} \\
940.~{\bf Eric Quintero}, Graduate Student, {\sl CalTech} \\
941.~{\bf Michael Lee Miller}, PhD, Founder, General Partner, {\sl Liquid 2 Venture Capital} \\
942.~{\bf Marco Viero}, Postdoctoral Fellow, PhD, {\sl KIPAC} \\
943.~{\bf Diana Powell}, PhD Student, {\sl UC Santa Cruz} \\
944.~{\bf Ann Heinson}, PhD, {\sl formerly at the University of California, Riverside} \\
945.~{\bf Yingyue Boretz}, PhD, {\sl The University of Texas at Austin } \\
946.~{\bf Nancy Rivera}, MS, MD Candidate, {\sl University of California, Davis School of Medicine } \\
947.~{\bf Adam Burgasser}, Professor of Physics, {\sl UC San Diego} \\
948.~{\bf Scott Pakudaitis}, Institutional Research Analyst, {\sl St. Catherine University} \\
949.~{\bf Jason Hancock}, Professor, {\sl University of Connecticut} \\
950.~{\bf Jodi Schwarz}, PhD, Associate Professor of Biology, {\sl Vassar College} \\
951.~{\bf Yonatan Zunger}, PhD, {\sl Google, Inc.} \\
952.~{\bf Timothy A. D. Paglione}, Chair, Professor, PhD, {\sl CUNY York College} \\
953.~{\bf Jolyon Bloomfield}, Dr, {\sl MIT} \\
954.~{\bf Peter Frinchaboy}, Associate Professor, {\sl Texas Christian University } \\
955.~{\bf Narayanan Menon}, Professor, Physics Dept, {\sl University of Massachusetts} \\
956.~{\bf Jason Veal}, {\sl Siegel High School} \\
957.~{\bf Kevin Hardegree-Ullman}, PhD candidate, {\sl University of Toledo} \\
958.~{\bf Emily Strickland}, Student, {\sl University of Texas at Austin} \\
959.~{\bf James Cummings}, {\sl UrbanPromise Ministries} \\
960.~{\bf Eric Toberer}, Assistant Professor, {\sl Colorado School of Mines} \\
961.~{\bf William Parkin}, PhD Student, {\sl University of Pennsylvania} \\
962.~{\bf Rita Wells}, {\sl Iowa State University} \\
963.~{\bf Jake Turner}, MS, {\sl University of Virginia} \\
964.~{\bf Theresa Watson}, Analyst Administration, {\sl Field of Finance} \\
965.~{\bf Jane Burnett}, PhD, {\sl Victor Valley College} \\
966.~{\bf Dr. Tony Roark}, Dean of Arts and Sciences, {\sl Boise State University } \\
967.~{\bf Marlena Watson, MA}, Lecturer, {\sl Lehman College} \\
968.~{\bf David Green}, Doctoral Student, {\sl NASA-GSFC} \\
969.~{\bf Julie Rolla}, MS Physics, {\sl UC Berkeley} \\
970.~{\bf Trisha Vickrey}, PhD, {\sl University of Nebraska-Lincoln} \\
971.~{\bf Sedona Price}, Graduate Student, {\sl UC Berkeley} \\
972.~{\bf Jennifer Blue}, PhD, {\sl Miami University} \\
973.~{\bf Barry N Costanzi}, MS, {\sl University of Minnesota} \\
974.~{\bf Dimitri Dounas-Frazer}, PhD, {\sl University of Colorado Boulder} \\
975.~{\bf Tom Ricks}, General Contractor \\
976.~{\bf Aditya Sriram Tiwari}, Student, {\sl Oakland University} \\
977.~{\bf Geoff Cureton}, PhD, {\sl University of Wisconsin - Madison} \\
978.~{\bf Sarah Wesolowski}, Graduate Student, {\sl The Ohio State University} \\
979.~{\bf Shaojun Luo}, PhD, {\sl City College of New York} \\
980.~{\bf R. Frank Polk, OD}, Faculty Member, {\sl Texas Tech University Health Sciences Center} \\
981.~{\bf Nicole Acosta}, Cloud Computing Evangelist, {\sl Cisco} \\
982.~{\bf Rayco Branch}, MSW, M.Ed \\
983.~{\bf Alison Sweeney}, assistant professor, {\sl University of Pennsylvania} \\
984.~{\bf Stuart A. Rice}, Frank P. Hixon Distinguished Service Professor, Emeritus, {\sl University of Chicago} \\
985.~{\bf Paul C. Bloom}, Associate Professor of Physics, {\sl North Central College} \\
986.~{\bf Jonathan McCoy}, PhD, Assistant Professor, {\sl Colby College} \\
987.~{\bf Sasha Leaman}, DPT graduate student, {\sl University of Puget Sound} \\
988.~{\bf Bryan Mendez}, PhD, {\sl University of California at Berkeley } \\
989.~{\bf Scott Paulson}, Associate Professor of Physics and Astronomy, {\sl James Madison University} \\
990.~{\bf Chandra Turpen}, PhD, Research Associate, {\sl University of Maryland, College Prk} \\
991.~{\bf Beverly Bell}, Dr., {\sl Small animal private practice} \\
992.~{\bf Daniel Huber}, PhD, {\sl University of Sydney} \\
993.~{\bf Yuhfen Lin}, PhD \\
994.~{\bf Leslie Upton}, PhD \\
995.~{\bf Valerie S. Tuck}, Senior Coordinator, Academic Rigor, {\sl Norfolk Public Schools} \\
996.~{\bf Dawn Buchan}, Teacher of Physics, {\sl Governor Livingston High School, NJ} \\
997.~{\bf Marc A. Murison}, PhD Astronomer \\
998.~{\bf Samuel Bloom}, PhD candidate, mathematics, {\sl University of Maryland, College Park} \\
999.~{\bf Jacquelyn Noronha-Hostler}, PhD, {\sl University of Houston} \\
1000.~{\bf Elizabeth Grace}, {\sl Reed College} \\
1001.~{\bf Ben Van Dusen}, Professor, {\sl California State University, Chico} \\
1002.~{\bf May Palace}, Student \\
1003.~{\bf Marissa B. Adams}, Graduate Student, {\sl University of Rochester} \\
1004.~{\bf Daniel Fernandez}, {\sl Radiation Monitoring Devices Inc.} \\
1005.~{\bf Marty L. Williams}, PhD, {\sl Valdosta State University} \\
1006.~{\bf James Peacock}, PhD, {\sl Duke University Medical Center} \\
1007.~{\bf David M. Gill}, Ed.D., {\sl UNC Wilmington} \\
1008.~{\bf Michael Lomnitz}, PhD candidate in Physics, {\sl Kent state University and Lawrence Berkeley National Lab} \\
1009.~{\bf Zin Lin}, MS, PhD candidate, {\sl Harvard University} \\
1010.~{\bf Fatma Mili}, Professor, {\sl Purdue University} \\
1011.~{\bf Katrina Hay}, PhD, Associate Professor of Physics, {\sl Pacific Lutheran University} \\
1012.~{\bf Teresa Barnes}, PhD, {\sl National Renewable Energy Lab} \\
1013.~{\bf Virginia Player}, Secondary Physics Teacher, {\sl Discovery College } \\
1014.~{\bf Randi Nelson}, MD \\
1015.~{\bf Rellen Hardtke}, PhD, Professor, {\sl University of Wisconsin-River Falls} \\
1016.~{\bf Martha Gross}, PhD candidate, {\sl University of Texas at Austin} \\
1017.~{\bf Jeffrey A Rich}, PhD, {\sl Caltech} \\
1018.~{\bf Bradford Benson}, Professor, {\sl University of Chicago} \\
1019.~{\bf Elizabeth Wilkes}, Mrs \\
1020.~{\bf Anne M. Brennan}, PhD, retired, {\sl Columbia University} \\
1021.~{\bf J. Allyn Smith}, PhD, {\sl Austin Peay State University} \\
1022.~{\bf Oluwatoyin Asojo}, PhD, {\sl Houston} \\
1023.~{\bf Van Savage}, Associate Professor, {\sl UCLA} \\
1024.~{\bf Natasha Holmes}, PhD, {\sl Stanford University} \\
1025.~{\bf Deborah Teramis Christian}, BA, Sociology, {\sl Independent Scholar} \\
1026.~{\bf Peter Galison}, Professor, {\sl Harvard University } \\
1027.~{\bf Philip F Meads Jr}, PhD, {\sl retired} \\
1028.~{\bf Kyle Willett}, PhD, {\sl University of Minnesota} \\
1029.~{\bf Alex McCormick}, PhD, Instructor, {\sl University of South Florida} \\
1030.~{\bf Rebekah Dawson}, PhD, Assistant Professor, {\sl Penn State} \\
1031.~{\bf Rebecca Harbison}, PhD, {\sl California Polytechnic State University} \\
1032.~{\bf Dillon Dong}, MPhil candidate, {\sl University of Cambridge} \\
1033.~{\bf Yafis Barlas}, PhD, {\sl University of California, Riverside} \\
1034.~{\bf Andy Johnson}, PhD, {\sl Black Hills State University} \\
1035.~{\bf Kenny Chowdhary}, PhD, Applied Mathematics, {\sl Sandia National Laboratories } \\
1036.~{\bf Deborah Levine}, PhD, {\sl Glendale Community College} \\
1037.~{\bf Matthew Buckley}, Professor, {\sl Rutgers University} \\
1038.~{\bf Matthew Lightman}, PhD \\
1039.~{\bf Elizabeth McGrath}, Professor, {\sl Colby College} \\
1040.~{\bf Rachel Theios}, PhD Student, {\sl Caltech} \\
1041.~{\bf Erik Daehler}, MS Space Systems, BA Physics, {\sl The Boeing Company} \\
1042.~{\bf Kemal Sobotkiewich}, Grad Student, {\sl Ut Austin} \\
1043.~{\bf Joshua Pepper}, Assistant Professor, {\sl Lehigh University} \\
1044.~{\bf Mark Palmer}, {\sl It's All Live} \\
1045.~{\bf Jessie Runnoe}, PhD, {\sl The Pennsylvania State University} \\
1046.~{\bf William Crawford}, MS in Physics, {\sl Los Angeles Harbor College} \\
1047.~{\bf Jacqueline Leonard}, Director and Professor, {\sl University of Wyoming} \\
1048.~{\bf Michaele Kashgarian}, PhD, {\sl LLNL} \\
1049.~{\bf Gregory Mace}, PhD, {\sl University of Texas at Austin} \\
1050.~{\bf Marco Pedulli}, PhD, {\sl University} \\
1051.~{\bf Thomas A. Moore}, PhD, {\sl Pomona College} \\
1052.~{\bf Luke C Flores}, PhD, {\sl Northwestern University} \\
1053.~{\bf Alex Krolewski}, Grad Student, {\sl UC Berkeley} \\
1054.~{\bf Andrew J. Davis}, PhD, {\sl Priory School of St .Louis} \\
1055.~{\bf Beth Stetzler}, MS, {\sl Goddard Space Flight Center} \\
1056.~{\bf Shankar Kumar}, PhD, {\sl Biotech Startup} \\
1057.~{\bf Alexander Madden}, {\sl Rochester Institute of Technology} \\
1058.~{\bf Mary Kate Dean}, MA in English, {\sl retired} \\
1059.~{\bf Jennifer L. Hoffman}, PhD, Associate Professor of Physics \& Astronomy, {\sl University of Denver} \\
1060.~{\bf Travis Rector}, PhD, {\sl University of Alaska} \\
1061.~{\bf Kingsley Durant, Jr.}, PhD, {\sl Durant \& Associates, LLC} \\
1062.~{\bf Dawn Yancy Elleby}, Mother of African American Children, {\sl Publishing} \\
1063.~{\bf Paolo Turri}, PhD student, {\sl University of Victoria} \\
1064.~{\bf Tom Foster}, Professor of Physics, {\sl Southern Illinois University Edwardsville} \\
1065.~{\bf Marissa L. Campbell, Esq.}, Attorney \\
1066.~{\bf Shannon Glavin}, {\sl University of Pennsylvania} \\
1067.~{\bf Johanna Teske}, PhD, {\sl Carnegie Institution of Science} \\
1068.~{\bf Craig DeForest}, PhD, {\sl Southwest Research Institute} \\
1069.~{\bf Samuel Cook}, Ph.D., {\sl Wheelock College} \\
1070.~{\bf Adric Riedel}, PhD, {\sl Caltech} \\
1071.~{\bf Michael Harold}, Inventor \\
1072.~{\bf Savannah Garmon}, Assistant Professor, {\sl Osaka Prefecture University} \\
1073.~{\bf Thaker Pinakin}, {\sl Darpana } \\
1074.~{\bf Owen Kenton}, {\sl University of Pennsylvania} \\
1075.~{\bf Charlotte Solarz}, MS, {\sl retired educator} \\
1076.~{\bf Jana Grcevich}, PhD, {\sl American Museum of Natural History} \\
1077.~{\bf Carol Guess}, Ph.D., {\sl Swarthmore College} \\
1078.~{\bf Marc Seigar}, Professor, {\sl University of Minnesota} \\
1079.~{\bf Kristine Washburn}, Professor of Physics, {\sl Everett Community College} \\
1080.~{\bf Kenyon Lee Whitman}, PhD Student, {\sl University of San Diego} \\
1081.~{\bf David Spergel}, Professor and Chair, {\sl Princeton University} \\
1082.~{\bf Dario Arena}, Associate Professor, {\sl University of South Florida} \\
1083.~{\bf Sara Anderson}, PhD, {\sl Georgetown} \\
1084.~{\bf Carlos Williams-Moreiras}, Student \\
1085.~{\bf Hilary Hurst}, Graduate Research Assistant, {\sl University of Maryland } \\
1086.~{\bf Nicole E. Cabrera Salazar}, M.Sc, {\sl Georgia State University} \\
1087.~{\bf Meleata Pinto}, MBA, {\sl Lenovo} \\
1088.~{\bf Britt Lundgren}, PhD, {\sl American Association for the Advancement of Science} \\
1089.~{\bf Louis J. Jisonna Jr., PhD}, Professor, Physical Sciences, {\sl Lone Star College - North Harris} \\
1090.~{\bf Justin Steinfadt}, PhD \\
1091.~{\bf Veronica Claxton} \\
1092.~{\bf Brandon Allgood}, PhD, CTO, {\sl Numerate, Inc} \\
1093.~{\bf Nancy Wolk}, MS Astronomy and Planetary Sciences, {\sl Solidus Techical Solutions} \\
1094.~{\bf Spencer Gessner}, PhD, {\sl SLAC National Accelerator Laboratory} \\
1095.~{\bf Benjamin Geller}, PhD, {\sl Swarthmore College} \\
1096.~{\bf Leigh Schaefer}, PhD student, {\sl University of Pennsylvania} \\
1097.~{\bf Erin Gauger}, PhD Student, {\sl University of Texas at Austin} \\
1098.~{\bf Samuel Homiller}, BS, Physics, {\sl Stony Brook University} \\
1099.~{\bf Zachary Gaber}, PhD, {\sl Francis Crick Institute} \\
1100.~{\bf Philip Plait}, PhD Astronomer \\
1101.~{\bf Ralph Romero}, Undergrad, {\sl Virginia Tech} \\
1102.~{\bf Thomas M. Caruso} \\
1103.~{\bf Charles A. Wood}, PhD, {\sl Planetary Science Institute} \\
1104.~{\bf Saad Zaheer}, PhD \\
1105.~{\bf Thereza C L Paiva}, Professor, {\sl Universidade Federal do Rio de Janeiro } \\
1106.~{\bf Bobby Rolison}, Medical Doctor, Mathematician UT Austin '98, {\sl Princess Alexandra Hospital Brisbane QLD Australia} \\
1107.~{\bf Katarina Cicak}, Physicist, {\sl National Institute of Science and Technology} \\
1108.~{\bf Cacey Stevens}, PhD, {\sl Duke University} \\
1109.~{\bf Ariel Weaver}, Graduate Research Assistant, {\sl University of Louisville} \\
1110.~{\bf Tammy Smecker-Hane}, Associate Professor, {\sl University of California, Irvine} \\
1111.~{\bf Christopher A. Hoffmann}, MS, Manager of Educational Laboratories, {\sl Western Michigan University} \\
1112.~{\bf Jason E. Ybarra}, PhD, Visiting Assistant Professor, {\sl Bridgewater College} \\
1113.~{\bf Brian Shuve}, Postdoctoral researcher, {\sl SLAC National Accelerator Laboratory} \\
1114.~{\bf Adam Becker}, PhD, {\sl Freelance} \\
1115.~{\bf Geoffrey Mathews}, PhD, {\sl University of Hawaii} \\
1116.~{\bf Taviare Hawkins}, PhD, {\sl University of Wisconsin La Crosse} \\
1117.~{\bf Albion Lawrence}, Associate Professor of Physics, {\sl Brandeis University} \\
1118.~{\bf Mayur Mudigonda}, PhD Student, {\sl UC Berkeley} \\
1119.~{\bf Anna Carter}, PhD, {\sl Science in Society Group, Victoria University of Wellington, NZ} \\
1120.~{\bf Jacqueline Monkiewicz}, PhD candidate, {\sl Arizona State University} \\
1121.~{\bf Hans Robinson}, Associate Professor, {\sl Virginia Tech} \\
1122.~{\bf Anne Medling}, PhD, {\sl Australian National University} \\
1123.~{\bf Janice Hudgings}, Seeley H. Mudd Professor of Physics, {\sl Pomona College} \\
1124.~{\bf Pedro Espino}, Graduate Student, {\sl University of Arizona} \\
1125.~{\bf Justin Shaw}, PhD, {\sl Emory University} \\
1126.~{\bf Daniel Rokhsar}, Professor, {\sl University of California} \\
1127.~{\bf Joshua Montgomery}, MS, {\sl McGill University / University of Chicago} \\
1128.~{\bf Magali Billen}, Professor, {\sl UC Davis} \\
1129.~{\bf Nicole Gugliucci}, Assistant Professor, {\sl Saint Anselm College} \\
1130.~{\bf Jami Valentine}, PhD \\
1131.~{\bf Sarah}, MS, {\sl University of Arizona} \\
1132.~{\bf Michael W. Busch}, Research Scientist, {\sl SETI Institute} \\
1133.~{\bf Kevin Tian}, PhD Student, {\sl Harvard University} \\
1134.~{\bf Peter Rosenthal}, PhD Stanford University \\
1135.~{\bf Barbara Fishman}, PH.D, {\sl I am a therapist and a meditation teacher} \\
1136.~{\bf Patrick McMaster}, Mr, {\sl DePalmas Italian Cafes} \\
1137.~{\bf Gopal Narayanan}, PhD, {\sl University of Massachusetts, Amherst} \\
1138.~{\bf Marc M. Mahan}, Photographer, {\sl Retired} \\
1139.~{\bf Anthony L. Hutcheson}, PhD, {\sl NRL} \\
1140.~{\bf John Zinke, MD}, MD, {\sl Retired} \\
1141.~{\bf Tony Zito}, MS \\
1142.~{\bf Joey Lee}, MS, {\sl University of British Columbia} \\
1143.~{\bf Natalie Ferraiolo}, 4th Year Medical Student, {\sl UC San Diego School of Medicine} \\
1144.~{\bf Annie Selden}, PhD, {\sl New Mexico State University} \\
1145.~{\bf James Guillochon}, PhD, {\sl Harvard} \\
1146.~{\bf Wynwood N Curry}, Professor, {\sl Coppin State University} \\
1147.~{\bf Linda Xu}, Graduate Student, {\sl Harvard University} \\
1148.~{\bf Diana Valencia}, Professor, {\sl University of Toronto} \\
1149.~{\bf Thomas Barclay}, PhD, {\sl Bay Area Environmental Research Institute} \\
1150.~{\bf William C. Keel}, Professor, {\sl University of Alabama} \\
1151.~{\bf Timothy Brandt}, PhD, {\sl Institute for Advanced Study} \\
1152.~{\bf Sharon K. Goldblatt}, Ms., {\sl Former founder/owner of Frontier Geosciences Inc. and Studio Geochimica Ltd.} \\
1153.~{\bf Miles Hogan}, B.A. Music Performance, {\sl Parris Island Marine Band} \\
1154.~{\bf Patrice Moorer}, Assistant Dean, College of Science \& Engineering, {\sl University of West Florida} \\
1155.~{\bf Matthew Morris}, MS Astrophysics, {\sl Johns Hopkins University} \\
1156.~{\bf Steven Finkelstein}, Professor, {\sl The University of Texas at Austin} \\
1157.~{\bf Nathan Joel Goldschmidt}, PhD, {\sl Stay True Philadelphia} \\
1158.~{\bf Shyamoli Chaudhuri Plassmann}, PhD, {\sl Princeton University Libraries} \\
1159.~{\bf Margaret A Charous}, PhD, {\sl Retired} \\
1160.~{\bf Liz Pollard}, Realtor, {\sl Keller Williams, San Carlos} \\
1161.~{\bf Natalia Antrobus}, {\sl NYC legal fund} \\
1162.~{\bf Kelley M. Hess}, PhD, {\sl University of Groningen} \\
1163.~{\bf John Morrison Galbraith}, Associate Professor, {\sl Marist College} \\
1164.~{\bf Britt Scharringhausen}, Associate Professor of Physics and Astronomy, {\sl Beloit College} \\
1165.~{\bf Elizabeth Hanna}, Developmental Psychology, PhD, {\sl Retired} \\
1166.~{\bf Sabrina Stierwalt}, PhD, {\sl NRAO} \\
1167.~{\bf Vivian Houng}, MD- family medicine \\
1168.~{\bf Roni Teich}, Graduate Student, {\sl Boston University} \\
1169.~{\bf Karen Hill}, PT, {\sl Medstar Rehabilitation} \\
1170.~{\bf Benjamin Tippett}, PhD, {\sl UBC} \\
1171.~{\bf Eric Hazlett}, Assistant professor, {\sl Carleton College} \\
1172.~{\bf Gwen N. Westerman}, PhD, Distinguished Faculty Scholar, {\sl Minnesota State University, Mankato } \\
1173.~{\bf Matt Tilley}, PhD Candidate, {\sl University of Washington} \\
1174.~{\bf Aniketa Shinde}, PhD, {\sl California Institute of Technology} \\
1175.~{\bf Peter Nguyen}, {\sl Arizona Science Center} \\
1176.~{\bf Susan Haase}, Special Education Teacher, {\sl Teaneck Public Schools} \\
1177.~{\bf Catherine Grant}, PhD, {\sl MIT Kavli Institute for Astrophysics and Space Research} \\
1178.~{\bf Raymond Hart}, PhD, {\sl Council of the Great City Schools} \\
1179.~{\bf Willie S. Rockward}, PhD, Associate Professor of Physics \& Dept. Chair, {\sl Morehouse College, Atlanta, GA} \\
1180.~{\bf David Targan}, PhD, {\sl Brown University} \\
1181.~{\bf Benjamin Stillwell}, Engineering physicist \\
1182.~{\bf Kenneth L. Johnson}, Statistician, {\sl NASA (representing self and not the U.S. Government))} \\
1183.~{\bf Emily Quinn Finney}, Graduate Student, Astrophysics, {\sl University of California, Davis} \\
1184.~{\bf Desika Narayanan}, Professor, {\sl Haverford College} \\
1185.~{\bf Murray Silverstone}, Professor, {\sl University of Alabama} \\
1186.~{\bf Richard R. Barnett}, Science Educator \\
1187.~{\bf Lafe Spietz}, PhD, {\sl Boulder Applied Physics, inc.} \\
1188.~{\bf LaNell Williams}, Masters Student, {\sl Fisk-Vanderbilt Masters to Ph.D} \\
1189.~{\bf Luiz de Viveiros}, PhD, {\sl University of California Santa Barbara} \\
1190.~{\bf Julie Shoemaker}, Asst. Professor of Earth Science, {\sl Lesley University} \\
1191.~{\bf Eric Johnson}, {\sl FIU} \\
1192.~{\bf Andrea Liu}, Hepburn Professor of Physics, {\sl University of Pennsylvania} \\
1193.~{\bf Daniel Mortlock}, lecturer, {\sl Imperial College London} \\
1194.~{\bf Eleanor Close}, Assistant Professor of Physics, {\sl Texas State University} \\
1195.~{\bf Jo Pitesky}, PhD, {\sl Jet Propulsion Laboratory} \\
1196.~{\bf Jackie Meyer}, MS in Applied Physics, {\sl Schlumberger} \\
1197.~{\bf Julie N. Skinner}, PhD, {\sl Boston University} \\
1198.~{\bf Jason L. James Jr.}, EdD, {\sl Wilmington University } \\
1199.~{\bf Gina Passante}, Assistant Professor, {\sl California State University Fullerton} \\
1200.~{\bf Ernesto Calleros}, Mathematics PhD Student, {\sl Rice University Mathematics Department } \\
1201.~{\bf Herbert Watson}, Professor, {\sl New York University} \\
1202.~{\bf Jason Hempstead}, {\sl University of Washington} \\
1203.~{\bf JoEllen McBride}, PhD candidate, {\sl University of North Carolina  at Chapel Hill } \\
1204.~{\bf Vergil L. Daughtery,iii}, BS, MS, {\sl Alumni, Georgia Institute of Technology} \\
1205.~{\bf Andrew Blain}, Professor, {\sl University of Leicester} \\
1206.~{\bf Jessica Muir}, PhD candidate, {\sl University of Michigan} \\
1207.~{\bf Lisa Tran}, MS, {\sl University of Pennsylvania} \\
1208.~{\bf Quyen Nguyen Hart}, PhD, Assistant Professor of Physics \& Astronomy, {\sl Regis University} \\
1209.~{\bf Jennifer Ross}, Associate Professor of Physics, {\sl University of Massachusetts Amherst} \\
1210.~{\bf Michael Regan}, PhD, {\sl Georgia State University} \\
1211.~{\bf Monika Lynker}, PhD, {\sl IU South Bend} \\
1212.~{\bf Erick J. Guerra}, Chairperson, Physics \& Astronomy, {\sl Rowan University} \\
1213.~{\bf Rebecca Roycroft}, Graduate Student, {\sl University of Texas Austin} \\
1214.~{\bf Andy Anderson, Ph.D.}, Academic Technology Specialist, {\sl Amherst College} \\
1215.~{\bf Morgan C. Benton}, PhD. Information Systems, {\sl James Madison University} \\
1216.~{\bf Phil Marshall}, PhD, {\sl Stanford University} \\
1217.~{\bf Benjamin Vollmayr-Lee}, Associate Professor of Physics, {\sl Bucknell University} \\
1218.~{\bf Nicholas Hunt-Walker}, MS, {\sl University of Washington} \\
1219.~{\bf Michael Blanton}, Associated Professor, {\sl New York University} \\
1220.~{\bf Ramsey Badawi}, Medical physicist; professor of radiology, {\sl UC Davis} \\
1221.~{\bf Kate Thedell}, Parent and paraeducator, {\sl Avani High School} \\
1222.~{\bf Onyema Osuagwu}, PhD, {\sl Johns Hopkins University Applied Physics Laboratory} \\
1223.~{\bf Dylan Gorman}, PhD candidate, {\sl UC Berkeley, Dept of Physics} \\
1224.~{\bf Mary Clapp}, PhD Student, {\sl University of California, Davis} \\
1225.~{\bf Marva Berry}, STEM Coordinator, {\sl The Learning Tree, Inc.} \\
1226.~{\bf Shea Cheatham}, {\sl Principia College} \\
1227.~{\bf Richard Shagam}, PhD, {\sl Sandia National Laboratories (retired)} \\
1228.~{\bf Mark Schueler}, PhD \\
1229.~{\bf Paul Scowen}, Professor, {\sl Arizona State University} \\
1230.~{\bf Jennifer Cash}, Professor of Physics, {\sl South Carolina State University} \\
1231.~{\bf Margaret Edelman}, MS \\
1232.~{\bf Angela N Fioretti}, Ms., {\sl National Renewable Energy Laboratory} \\
1233.~{\bf Kira Gould} \\
1234.~{\bf Krishna Myneni}, M.Sc. Physics \\
1235.~{\bf Cyrus Behroozi}, {\sl X, Alphabet Inc.} \\
1236.~{\bf Kay Kinoshita}, PhD, Professor and Head, Department of Physics, {\sl University of Cincinnati} \\
1237.~{\bf Thomas Hill}, Professor Emeritus, {\sl Rice University} \\
1238.~{\bf Adria C. Updike}, Assistant Professor of Physics, {\sl Roger Williams University} \\
1239.~{\bf Kailyn Elliott}, RN, {\sl University of Washington } \\
1240.~{\bf Meghan S Anzelc}, PhD, {\sl Zurich} \\
1241.~{\bf Simon Mochrie}, Professor, {\sl Yale University} \\
1242.~{\bf Danish Haque}, MD, {\sl Midwestern University} \\
1243.~{\bf Angelle Tanner}, PhD, {\sl Mississippi State University} \\
1244.~{\bf Reba Bandyopadhyay}, PhD, {\sl American Association for the Advancement of Science} \\
1245.~{\bf Maria Womack}, Professor, {\sl University of South Florida} \\
1246.~{\bf Kyron M. Williams}, PhD, {\sl Nyota Research} \\
1247.~{\bf Eileen Gonzales}, Graduate Student, {\sl CUNY GC} \\
1248.~{\bf Katherine J. Mack}, PhD, {\sl University of Melbourne} \\
1249.~{\bf Duncan Brown}, Charles Brightman Professor of Physics, {\sl Syracuse University} \\
1250.~{\bf B. Kevin Edgar}, PhD, {\sl University of Minnesota} \\
1251.~{\bf Carol Cihara}, Ph.D, {\sl USF (ret)} \\
1252.~{\bf Evan Thomas}, MSc, {\sl Univeersity of British Columbia} \\
1253.~{\bf S\'{e}bastien Cormier}, PhD, {\sl Grossmont College} \\
1254.~{\bf Shulamith Freedman}, MSEd, {\sl CUNY Lehman College} \\
1255.~{\bf Beverly Stewart}, MS Education, Kindergarten Teacher, {\sl Alexandria City Public Schools, Alexandria, VA} \\
1256.~{\bf Michael Rutkowski}, PhD, {\sl University of Minnesota} \\
1257.~{\bf Megan Comins}, PhD, {\sl Industry} \\
1258.~{\bf Breylon Riley}, {\sl Albert Einstein College of Medicine} \\
1259.~{\bf Kenna D. S. Lehmann}, PhD candidate, {\sl Mighican State University} \\
1260.~{\bf Kathryn Meehan}, MS, {\sl UC Davis} \\
1261.~{\bf Ann Nelson}, Professor of Physics, {\sl University of Washington} \\
1262.~{\bf Donna Y. Ford}, Professor, {\sl Vanderbilt University} \\
1263.~{\bf Suzanne Casement}, PhD Astrophysics, {\sl Northrop Grumman Corporation} \\
1264.~{\bf Razi Abdur-Rahman}, MS.Ed, {\sl Brooklyn Friends School} \\
1265.~{\bf Ryan Toole}, MS, {\sl University of Georgia} \\
1266.~{\bf Janet Glocker}, 4th grade Public education teacher, {\sl Minneapolis } \\
1267.~{\bf Lisa Hiton}, Professor, {\sl Harvard University} \\
1268.~{\bf Duane M. Jackson}, PhD., {\sl Morehouse College} \\
1269.~{\bf Ashley}, Student, {\sl Agnes Scott College} \\
1270.~{\bf Brian Metzger}, Professor, {\sl Columbia University} \\
1271.~{\bf Massimo Marengo}, Associate Professor, Physics and Astronomy, {\sl Iowa State University} \\
1272.~{\bf Brianna M. Smart}, MS, {\sl University of Wisconsin-Madison} \\
1273.~{\bf Evan Schneider}, MS, {\sl University of Arizona} \\
1274.~{\bf Hiranya Peiris}, Professor, {\sl University College London} \\
1275.~{\bf Paloma Orozco Scott}, Student, {\sl Brown University} \\
1276.~{\bf Michael Shaw}, PhD, {\sl University of Minnesota Duluth} \\
1277.~{\bf Daniel Lecoanet}, PhD candidate, {\sl UC-Berkeley} \\
1278.~{\bf Rachel Worth}, MS, {\sl Penn State} \\
1279.~{\bf Francisco Garcia}, PhD Candidate, {\sl University of Massachusetts - Amherst} \\
1280.~{\bf Joseph D. Lopes}, Graduate Student, {\sl University of California, Merced} \\
1281.~{\bf Carol Thornber}, Associate Professor, {\sl U. Rhode Island} \\
1282.~{\bf Tricia M. Tynan}, PhD, {\sl Arizona Oncology Associates} \\
1283.~{\bf Elliot Douglas}, Engineering Professor, {\sl University of Florida} \\
1284.~{\bf Stephan Frank}, Postdoctoral Researcher, {\sl The Ohio State University} \\
1285.~{\bf Geoff Lawrence}, MS Lecturer, {\sl North Hennepin Community College} \\
1286.~{\bf Therese Jones}, MS, {\sl RAND} \\
1287.~{\bf Hugh Lippincott}, Associate Scientist, {\sl Fermilab} \\
1288.~{\bf Ethan T. Vishniac}, Professor, {\sl Johns Hopkins University} \\
1289.~{\bf Colin Patterson}, MS, MPA, PhD candidate, Adjunct Instructor, {\sl University of Maryland University College European Divison} \\
1290.~{\bf Robert Tate}, Project Manager, {\sl AT\&T} \\
1291.~{\bf Dr. Dagny Looper}, PhD, {\sl NYU} \\
1292.~{\bf Daniel Alt}, PhD candidate, {\sl National Superconducting Cyclotron Laboratory, Michigan State University} \\
1293.~{\bf Kitchka Petrova}, Graduate Student, {\sl FSU} \\
1294.~{\bf Craig Keemer}, BS Chemistry, {\sl Silberline Mfg.} \\
1295.~{\bf Bethany Reilly}, PhD, {\sl University of Wisconsin Fox Valley} \\
1296.~{\bf Suzanne M. Cohen}, Adjunct Professor, {\sl Seattle University} \\
1297.~{\bf Eva Peterson}, Ed.D., {\sl retired} \\
1298.~{\bf Susan Tappero}, PhD, Mathematics, {\sl Cabrillo College} \\
1299.~{\bf Jessica Ennis}, MS, {\sl Augsburg College} \\
1300.~{\bf Cindy Harnett}, PhD, {\sl University of Louisville} \\
1301.~{\bf Stamatis Vokos}, PhD, Professor of physics, {\sl Seattle Pacific University} \\
1302.~{\bf Kristina Looper}, MS, {\sl The Ohio State University} \\
1303.~{\bf Denis Erkal}, Phd, {\sl University of Cambridge} \\
1304.~{\bf Camellia Moses Okpodu}, PhD Plant Physiology and Biochemistry, {\sl Norfolk State University, Norfolk VA} \\
1305.~{\bf Kevin McChesney}, Masters of Science in Physics, {\sl Pickerington High School Central} \\
1306.~{\bf Kevin R. Covey}, Asst. Prof., {\sl Western Washington Univ.} \\
1307.~{\bf Tammy Walton}, PhD, {\sl Fermilab} \\
1308.~{\bf Sean Chilelli}, Engineering BS, {\sl UC Berkeley} \\
1309.~{\bf Tiffany Cochran}, BS, MHA, and current MD student, {\sl Morehouse School of Medicine} \\
1310.~{\bf Oba William King}, ED.Ent  91' ... dec2015 sats 2p, {\sl Chicago History Museum } \\
1311.~{\bf Charles D. Hoyle}, Professor of Physics \& Astronomy, {\sl Humboldt State University} \\
1312.~{\bf Roberto Vega}, PhD \\
1313.~{\bf Mary Spraggs}, Student, {\sl WKU} \\
1314.~{\bf Hao Shi}, PhD student, {\sl Cornell University} \\
1315.~{\bf Aparna Baskaran}, Assistant Professor, {\sl Brandeis University} \\
1316.~{\bf Sara A. Solla}, Professor, {\sl Northwestern University} \\
1317.~{\bf Charles McIntyre}, Supervisor of Training Programs, {\sl Reed College Research Reactor} \\
1318.~{\bf Richard Brockie}, PhD, {\sl HGST, a Western Digital company} \\
1319.~{\bf Sheila Hazel}, RN, {\sl Retired} \\
1320.~{\bf Robert McNees}, PhD, Associate Professor, {\sl Loyola University Chicago} \\
1321.~{\bf Soneyet Muhammad}, Financial Educator, {\sl Nonprofit, Philadelphia} \\
1322.~{\bf Aaron Jacks}, MS, {\sl Houston Heights High School} \\
1323.~{\bf Lauren Aycock}, {\sl Joint Quantum Institute} \\
1324.~{\bf Laura Woodney}, Professor of Physics, {\sl California State Universtiy, San Bernardino} \\
1325.~{\bf Gabriel Dima}, MS, {\sl University of Hawaii} \\
1326.~{\bf Alison Coil}, Professor of Physics, {\sl University of California, San Diego} \\
1327.~{\bf Jeffrey Berger}, Journalist \\
1328.~{\bf Natina Brown}, {\sl DCS} \\
1329.~{\bf Robert A Nendorf}, PhD Mathematics, {\sl Allstate} \\
1330.~{\bf Faye Hendley Elgart}, student, {\sl Cornell University} \\
1331.~{\bf Rome Mubarak}, Professor of Political Science, {\sl Contra Costa College } \\
1332.~{\bf Djordje Minic}, Professor, {\sl Department of Physics, Virginia Tech} \\
1333.~{\bf Jessica Neubauer}, Ms., {\sl Grant County High School} \\
1334.~{\bf Meredith Danowski}, {\sl Boston University} \\
1335.~{\bf Joshua Hignight}, PhD, {\sl Michigan State University} \\
1336.~{\bf Heather N. Richardson}, PHD, {\sl University of Massachusetts Amherst} \\
1337.~{\bf Teresa Bixby}, Assistant Professor - Physical Chemistry, {\sl Lewis University} \\
1338.~{\bf Richard Wedeen}, Student, {\sl Stanford University} \\
1339.~{\bf Erika Houtz}, PhD, {\sl California DTSC Environmental Chemistry Lab} \\
1340.~{\bf Leslie Atkins}, Associate Professor, {\sl Boise State University} \\
1341.~{\bf Elizabeth Eeosti Jacobson}, Retired Juvinile Probation Officer, {\sl Sonoma County, CA.} \\
1342.~{\bf John C Martin}, Associate Professor of Atronomy/Physics, {\sl University of Illinois Springfield} \\
1343.~{\bf Kurt Truong}, DDS, MSD, {\sl Pediatric Dentist } \\
1344.~{\bf Bobbie DeCuir}, PhD, {\sl University of Louisiana} \\
1345.~{\bf Joel C. Corbo}, PhD, Postdoctoral Researcher, {\sl University of Colorado Boulder} \\
1346.~{\bf Reina Maruyama}, Assistant Professor of Physics, {\sl Yale University} \\
1347.~{\bf Barbara A. Walker}, MD, MPH \\
1348.~{\bf John Morgan}, PhD \\
1349.~{\bf Sara Daise} \\
1350.~{\bf Nicholas Pik}, PhD, {\sl University of Liege} \\
1351.~{\bf Omer Blaes}, Professor, {\sl UCSB} \\
1352.~{\bf Gonzalo Gonzalez-Del Pino}, PhD Candidate, {\sl Harvard University} \\
1353.~{\bf Karla S Hamblin}, Master of Urban and Regional Planning, {\sl BP Exploration \& Production } \\
1354.~{\bf Stephon Alexander}, Royce Professor of Physics, {\sl Brown University } \\
1355.~{\bf Mikki Fuller}, MS, {\sl Vocational Rehabilitation  } \\
1356.~{\bf Chieze Ibeneche-Nnewihe}, PhD \\
1357.~{\bf Matt Mechtley}, PhD, {\sl Arizona State University} \\
1358.~{\bf Shan Swaminathan} \\
1359.~{\bf Brad Knockel}, MS, {\sl CNM Community College} \\
1360.~{\bf Patricia Rolen}, MA, {\sl MUMC} \\
1361.~{\bf Donghui Jeong}, Professor, {\sl Penn State} \\
1362.~{\bf Paul A Williams}, Mr., {\sl Northwestern University } \\
1363.~{\bf Paul Goudfrooij}, PhD, {\sl Space Telescope Science Institute } \\
1364.~{\bf Charles Nelson}, PhD, {\sl Drake University } \\
1365.~{\bf Laura Lembke}, 6th grade Science/STEM Teacher, {\sl Augusta County Schools, Staunton VA} \\
1366.~{\bf Richard Reitman}, Mit graduate and retired engineering manager, {\sl Retired } \\
1367.~{\bf Travis Miles}, Professor, {\sl Rutgers University} \\
1368.~{\bf Juliana Bol}, PhD candidate, {\sl Johns Hopkins} \\
1369.~{\bf Eric Suchyta}, PhD, {\sl University of Pennsylvania} \\
1370.~{\bf Jannel Banks}, student, {\sl University of Washington} \\
1371.~{\bf Jacob Barandes}, Associate Director of Graduate Studies, Lecturer on Physics, {\sl Harvard University} \\
1372.~{\bf Tien-Tien Yu}, {\sl Stony Brook University} \\
1373.~{\bf Evra Baldinger}, {\sl UC Berkeley} \\
1374.~{\bf K. E. Saavik Ford}, Professor, {\sl CUNY BMCC} \\
1375.~{\bf Joanne Hughes Clark}, PhD, {\sl Seattle University} \\
1376.~{\bf Laura Daniela Vega}, NASA Graduate Fellow, {\sl Vanderbilt University} \\
1377.~{\bf Keivan Stassun}, Professor of Physics \& Astronomy, {\sl Vanderbilt University} \\
1378.~{\bf Andreas Amann}, PhD, {\sl Quantum Design, Inc.} \\
1379.~{\bf Mylo Egipciaco}, Professor, {\sl LACCD/Southwest \& Pierce} \\
1380.~{\bf Tabetha Hole}, Professor, {\sl Norwich University} \\
1381.~{\bf Crystal L. Pope}, {\sl Georgia State University} \\
1382.~{\bf Luke Mastalli-Kelly}, PhD candidate and College Trustee, {\sl University of Houston, Harvey Mudd College} \\
1383.~{\bf Nicholas Pierce}, BS Data Science, {\sl University of Rochester} \\
1384.~{\bf Kevin Rathunde}, Shoku-cho (manager), {\sl Aisin ELectronic of Illinois LLC} \\
1385.~{\bf Anna Crowe Dewart}, Ass't Prof., English, retired, {\sl College of Coastal Georgia} \\
1386.~{\bf Oscar Ozuna}, Physicist, {\sl Rogers Corporation} \\
1387.~{\bf Patricia Juarez}, Professor / Analyst, {\sl Calpulli Huey Papalotl / UC Berkeley} \\
1388.~{\bf Bodhitha Jayatilaka}, PhD, {\sl Fermi National Accelerator Laboratory} \\
1389.~{\bf Kim Coble}, Professor of Physics, {\sl Chicago State University} \\
1390.~{\bf Sarah D. Johnson}, PhD, {\sl Simon Fraser University} \\
1391.~{\bf Amalia Castonguay}, {\sl Oak Crest Institute of Science } \\
1392.~{\bf Daniel Aharoni}, Postdoctoral Fellow, {\sl UCLA} \\
1393.~{\bf James Lowenthal}, Professor of Astronomy, {\sl Smith College} \\
1394.~{\bf David Cauffman}, PhD, {\sl retired industry chief scientist} \\
1395.~{\bf Karri Folan DiPetrillo}, Physics PhD Student, {\sl Harvard University } \\
1396.~{\bf Dan Spiegel}, Professor, {\sl Trinity University} \\
1397.~{\bf Robyn Levine}, PhD, {\sl Exeter, UK} \\
1398.~{\bf Stuart Kerridge}, PhD, {\sl Cal Tech (ret'd)} \\
1399.~{\bf Stephanie Walker}, Medical Student, 1st year, {\sl UCSD School of Medicine} \\
1400.~{\bf Adrian B. Lucy}, Graduate Fellow, {\sl Dept. of Astronomy, Columbia University} \\
1401.~{\bf Larry Seiple}, BS Botany Research Assoc., {\sl U.C. Berkeley, retired} \\
1402.~{\bf Gardner Swan}, MS, {\sl United States Patent and Trademark Office} \\
1403.~{\bf Josh Borja}, Global Academic Fellow in Physics and Writing, {\sl New York University Shanghai (NYU Shanghai)} \\
1404.~{\bf Evalyn Gates}, PhD, Executive Director and CEO, {\sl Cleveland Museum of Natural History} \\
1405.~{\bf Dawn Vincent Dumas}, RN/BSN \\
1406.~{\bf Lamiya Mowla}, Graduate Student in Astronomy, {\sl Yale University} \\
1407.~{\bf Kjiersten Fagnan}, PhD, {\sl Lawrence Berkeley National Laboratory} \\
1408.~{\bf Frances Richardson}, MS, {\sl Prince George's County Memorial Library System} \\
1409.~{\bf Allen H. Williams}, {\sl Retired} \\
1410.~{\bf Michael Lerner}, Ed.M., {\sl Beachwood High School} \\
1411.~{\bf Anton Baleato Lizancos}, Student, {\sl Columbia University} \\
1412.~{\bf Kenneth Salins}, {\sl Digital Infuzion} \\
1413.~{\bf Cecilia Tung}, High School Physics Teacher, {\sl The Northwest School} \\
1414.~{\bf Amy Graves}, Professor, {\sl Swarthmore College} \\
1415.~{\bf Julie Benton}, PhD \\
1416.~{\bf David Rosario}, PhD, {\sl Durham University} \\
1417.~{\bf William Carter}, PhD, {\sl HRL Laboratories, LLC} \\
1418.~{\bf Richa Batra}, Mechanical Engineering Doctoral Student, {\sl Columbia University} \\
1419.~{\bf Tia C. Madkins}, PhD candidate, STEM Instructor, {\sl UC Berkeley} \\
1420.~{\bf Adelita Mendoza}, PhD candidate, {\sl Northwestern University } \\
1421.~{\bf Anne Goodsell}, PhD, {\sl Middlebury College} \\
1422.~{\bf Ansel Neunzert}, PhD student, {\sl University of Michigan} \\
1423.~{\bf Duane L. Bindschadler}, PhD, {\sl Jet Propulsion Laboratory, California Institute of Technology} \\
1424.~{\bf Tyler J Hayden}, PhD candidate, {\sl University of Arizona} \\
1425.~{\bf La Bien Nance}, Professional homemaker \\
1426.~{\bf Jesse Smith}, Physics Student, {\sl San Francisco State University} \\
1427.~{\bf Emily Pritchett}, PhD, {\sl HRL Laboratories} \\
1428.~{\bf Joel Weisberg}, Prof of Phys \& Astronomy, {\sl Carleton College} \\
1429.~{\bf Nathalie Brunet}, M.Sc.(Chemistry) - Science Teacher, {\sl Macdonald Drive Junior High School} \\
1430.~{\bf Willie Rockward}, National President, {\sl Sigma Pi Sigma Physics Honor Society} \\
1431.~{\bf Russell Scarola}, MS, MA, Professor, {\sl Sonoma State University} \\
1432.~{\bf Ruth Ludwin}, Seismologist, MA in Geophysics, {\sl University of Washington (retired)} \\
1433.~{\bf Jose Cuellar}, PhD, {\sl SFSU} \\
1434.~{\bf Peter Edmonds}, PhD, {\sl Harvard-Smithsonian Center for Astrophysics} \\
1435.~{\bf Sam Whitehead}, PhD Student, {\sl Cornell University} \\
1436.~{\bf Michael Johnson}, Applied \& Engineering Physics, Cornell; MBA, Stanford \\
1437.~{\bf Erin Wolf}, System Engineer, {\sl NASA} \\
1438.~{\bf Adam James Archibald}, PhD candidate, graduate student, {\sl Washington University in Saint Louis} \\
1439.~{\bf Raymond Brock}, University Distinguished Professor, {\sl Michigan State University} \\
1440.~{\bf Breanna Binder}, PhD, {\sl University of Washington} \\
1441.~{\bf JoAnne Hewett}, Professor of Physics, {\sl Stanford University/SLAC} \\
1442.~{\bf Owen Anderson}, Retired Professor, {\sl Bucknell University} \\
1443.~{\bf Douglas Durian}, Professor, {\sl University of Pennsylvania} \\
1444.~{\bf Melanie Butler}, MS Chemistry, {\sl UIC} \\
1445.~{\bf Timothy J. Thompson}, MS, {\sl NASA/JPL Science Division, retired} \\
1446.~{\bf Eugene Beier}, Professor of Physics, {\sl University of Pennsylvania} \\
1447.~{\bf Erik Shirokoff}, Asst. Professor, {\sl University of Chicago} \\
1448.~{\bf Kebra Ward}, PhD, {\sl Massachusetts College of Liberal Arts} \\
1449.~{\bf David Whelan}, Professor, {\sl Austin College} \\
1450.~{\bf Craig Brashear}, BA Physics, Haverford College, {\sl Appleby Foundation} \\
1451.~{\bf Sean Taylor}, PhD, {\sl Minnesota State University Moorhead} \\
1452.~{\bf Maya Price}, {\sl Student at Michigan Technological University } \\
1453.~{\bf Matthew Enjalran}, PhD, Professor and Chair,, {\sl Physics, Southern CT State University} \\
1454.~{\bf Helen Jackson}, {\sl University of Pennsylvania} \\
1455.~{\bf Abra Alahouzos}, Montessori Early Childhood Educator, {\sl Oneness Family School} \\
1456.~{\bf Zachary Pace}, PhD Student, {\sl University of Wisconsin-Madison} \\
1457.~{\bf Paul Mirel}, {\sl NASA Goddard/ Wyle STE} \\
1458.~{\bf Edwin Ladd}, Professor, {\sl Bucknell University} \\
1459.~{\bf James F Prewitt}, PhD, Theoretical Nuclear Physics \\
1460.~{\bf Neill Reid}, Dr., {\sl STScI} \\
1461.~{\bf Melissa Hirsch}, MD/MPH Student, Previous High School Chemistry Teacher, {\sl SUNY Downstate College of Medicine} \\
1462.~{\bf Danielle Gulick}, PhD, {\sl University of South Florida} \\
1463.~{\bf Veronica Policht}, PhD candidate, {\sl University of Michigan} \\
1464.~{\bf Mary Arnold}, Ms, {\sl retired  case manager } \\
1465.~{\bf Yvette Aguilar}, Attorney, JD, {\sl City of Corpus Christi} \\
1466.~{\bf Jacqueline Townsend}, BS Physics, University of Maryland \\
1467.~{\bf Dwight Whitaker}, PhD Associate Professor of Physics, {\sl Pomona College} \\
1468.~{\bf Stephen S. Eikenberry}, Professor, {\sl University of Florida} \\
1469.~{\bf Ryan Trainor}, PhD, {\sl UC Berkeley / Miller Institute} \\
1470.~{\bf Nicole Ackerman}, Assistant Professor of Physics, {\sl Agnes Scott College} \\
1471.~{\bf Barok Yemane}, Student, {\sl University of Washington} \\
1472.~{\bf Ryan P. Norris}, Graduate Student, {\sl Georgia State University} \\
1473.~{\bf Elliot Richmond, PhD}, Professor, {\sl Austin Community College} \\
1474.~{\bf Sean Decker}, Mr, {\sl Chevron Stations Inc.} \\
1475.~{\bf David Sheffield}, PhD candidate, {\sl Rutgers University} \\
1476.~{\bf Shandre Delaney}, Grassroots Activist, {\sl Abolitionist Law Center} \\
1477.~{\bf Fawn Huisman}, PhD, {\sl Intel Corp} \\
1478.~{\bf Alexis Knaub}, Postdoc, {\sl Western Michigan University} \\
1479.~{\bf Ivelina Momcheva}, PhD, {\sl Space Telescope Science Institute} \\
1480.~{\bf Eleanor Sayre}, PhD, {\sl Kansas State University} \\
1481.~{\bf Sandi Clement}, PhD, Assistant Professor, {\sl California Polytechnic University San Luis Obispo} \\
1482.~{\bf Trish Millines Dziko}, Executive Director, {\sl Technology Access Foundation} \\
1483.~{\bf Kevin Black}, Professor, {\sl Boston University} \\
1484.~{\bf Nicolas Rey-Le Lorier}, Graduate Student, {\sl Cornell University} \\
1485.~{\bf Jonathan Ouellet}, PhD, {\sl MIT} \\
1486.~{\bf Debra Murray}, PhD, {\sl Baylor College of Medicine} \\
1487.~{\bf Everardo Olide}, BS Student, {\sl University of California-Davis} \\
1488.~{\bf Badr Albanna}, PhD, {\sl Fordham University} \\
1489.~{\bf Stanlie M. James}, Ph.D, {\sl Arizona State University} \\
1490.~{\bf Lauren D. Thomas}, PhD, {\sl University of Washington} \\
1491.~{\bf Zach Berta-Thompson}, PhD, {\sl MIT Kavli Institute for Astrophysics and Space Research} \\
1492.~{\bf Beth Newton Watson}, Rev., {\sl Indiana University Health} \\
1493.~{\bf Lavi Blumberg}, BS. and PhD student, {\sl Stanford University } \\
1494.~{\bf Tim Coyle}, American Voter, Tax Payer, Citizen \\
1495.~{\bf Ravi Sheth}, Professor, {\sl University of Pennsylvania} \\
1496.~{\bf Jolie Glaser}, Physics \& Environmental Science Teacher, {\sl Green Valley High School, Henderson, Nevada} \\
1497.~{\bf Maryam Modjaz}, Professor, {\sl NYU} \\
1498.~{\bf Karlen Shahinyan}, Graduate Student, {\sl Minnesota Institute for Astrophysics, University of Minnesota} \\
1499.~{\bf Rami Vanguri}, PhD, {\sl Columbia University} \\
1500.~{\bf Jeremy Bailin}, Assistant Professor, {\sl University of Alabama} \\
1501.~{\bf Elizabeth Boulton}, Graudate Student, {\sl Yale University} \\
1502.~{\bf Daniel Kang}, Medical Student, {\sl Albert Einstein College of Medicine} \\
1503.~{\bf Brian Good}, PhD, {\sl NASA Glenn Research Center} \\
1504.~{\bf Carol Dorf}, Math Teacher, {\sl Berkeley High School} \\
1505.~{\bf Siabhan May-Washington}, Ed.D. (Assistant Principal), {\sl Pembroke Hill School-Kansas City, Missouri} \\
1506.~{\bf David Caudel}, PhD candidate, Physics, {\sl Vanderbilt University} \\
1507.~{\bf Janelle Leger}, Associate Professor, {\sl Western Washington University} \\
1508.~{\bf Weigang Liu}, BS in Physics, {\sl University of Michigan } \\
1509.~{\bf Maria Siliezar}, undergraduate student/researcher, {\sl UC Berkeley} \\
1510.~{\bf Socorro Rojas}, {\sl School teacher} \\
1511.~{\bf Scott Lindauer}, {\sl North Carolina State University } \\
1512.~{\bf Sean Gavin}, Professor, PhD, {\sl Wayne State University} \\
1513.~{\bf Eric Corwin}, Assistant Professor of Physics, {\sl University of Oregon} \\
1514.~{\bf Melissa Munroe}, MD, PhD, {\sl Oklahoma Medical Research Foundation } \\
1515.~{\bf Derek Stein}, Professor, {\sl Brown University} \\
1516.~{\bf Cristina I. Moody}, MS \\
1517.~{\bf Walter Allen Jr}, PhD, {\sl Primavera Online High School - The American Virtual Academy} \\
1518.~{\bf Silviu Pufu}, Assistant Professor, {\sl Princeton University} \\
1519.~{\bf Nicolas Flagey}, PhD, {\sl Canada-France-Hawaii Telescope} \\
1520.~{\bf Anastasia Karabina}, PhD, {\sl University of Massachusetts Lowell} \\
1521.~{\bf Martin Bergoffen}, MS Physics, University of Colorado, {\sl Wheaton High School Physics Teacher} \\
1522.~{\bf Pamela Lange}, Professor, {\sl SMU} \\
1523.~{\bf Mitchell Struble}, Lecturer, {\sl University of Pennsylvania} \\
1524.~{\bf Kalle Christiansen}, Professor, {\sl O'Connell Preparatory Academy} \\
1525.~{\bf Steven Villanueva Jr}, {\sl The Ohio State University } \\
1526.~{\bf Markus Diefenthaler}, PhD, {\sl Thomas Jefferson National Accelerator Facility} \\
1527.~{\bf Rajeswari Kolagani}, PhD, Professor of Physics, {\sl Towson University} \\
1528.~{\bf Chris Hillman}, MSEE, {\sl Teledyne Scientific} \\
1529.~{\bf Brian Crow}, Physics Student, {\sl Middle Tennessee State University} \\
1530.~{\bf Florence Ackridge}, Msw, MA, {\sl Retired } \\
1531.~{\bf David Spiegel}, PhD, {\sl Stitch Fix} \\
1532.~{\bf Edward Bernstein}, Professor BU School of Medicine, {\sl Boston  University } \\
1533.~{\bf Maximilian H Abitbol}, Graduate Student, {\sl Columbia University} \\
1534.~{\bf Christina Krawiec}, {\sl University of Pennsylvania} \\
1535.~{\bf Kate Burleson}, {\sl City University of New York - City College} \\
1536.~{\bf Patrick M. McCall}, PhD Candidate in Physics, {\sl University of Chicago} \\
1537.~{\bf Sharon Persinger}, Dr., {\sl Bronx Community College of the City University of New York} \\
1538.~{\bf Craig Wiegert}, Associate Professor, {\sl University of Georgia} \\
1539.~{\bf Gail Zuniga} \\
1540.~{\bf Antonino Cucchiara}, PhD, {\sl Space Telescope Science Institute} \\
1541.~{\bf Michael Kaufman}, Professor and Chair, Dept. of Physics \& Astronomy, {\sl San Jose State University} \\
1542.~{\bf Jennifer Strauss}, PhD Physics, {\sl UC Berkeley} \\
1543.~{\bf Elizabeth J. Hines}, PhD, {\sl University of Pennsylvania} \\
1544.~{\bf Terri Major-Kincade}, MD, {\sl THR Resources/MEDNAX} \\
1545.~{\bf James V. Earley}, MS Astronomy, {\sl Excel Academy, Denver Public Schools} \\
1546.~{\bf Suzanne Willis}, Professor Emerita, {\sl Northern Illinois University} \\
1547.~{\bf Dan Baxter}, graduate student, {\sl Northwestern University} \\
1548.~{\bf Gary Felder}, Professor, {\sl Smith College} \\
1549.~{\bf Michael W. Vanni}, Chairman Emeritus, {\sl JCM Partners, LLC } \\
1550.~{\bf Roxanne Moore}, PhD Student, {\sl Washington State University} \\
1551.~{\bf Bruce Ballard}, Dr., {\sl Bronx Charter School for Better Learning} \\
1552.~{\bf Gergely Zimanyi}, Professor of Physics, {\sl University of California, Davis} \\
1553.~{\bf Jesse Livezey}, Graduate Student, {\sl UC Berkeley } \\
1554.~{\bf M. Renee Bellinger}, PhD, {\sl University of Hawaii} \\
1555.~{\bf Cherno Kah}, PhD candidate, {\sl University of louisville } \\
1556.~{\bf Njema J. Frazier}, Nuclear Physicist, {\sl US Department of Energy, National Nuclear Security Administration} \\
1557.~{\bf Steven Derek Rountree}, PhD, {\sl Virginia Tech} \\
1558.~{\bf Eric Sheets}, Director, {\sl Kickapoo Tribe in Kansas Environmental Office} \\
1559.~{\bf Joseph Kimball}, PhD, {\sl TCU} \\
1560.~{\bf Elisabeth Berger Bolaza}, MPH and PhD student, {\sl California Institute of Integral Studies} \\
1561.~{\bf Donna Kountoupes}, teacher, {\sl retired} \\
1562.~{\bf T. Brian Bunton, PhD}, Associate Professor of Physics, {\sl Coastal Carolina University} \\
1563.~{\bf Frank X. V\'{a}zquez}, PhD, {\sl IBM} \\
1564.~{\bf Dr. Chris Purcell}, {\sl West Virginia University} \\
1565.~{\bf Steve Stepanski}, Geologist \\
1566.~{\bf Deborah Jackson}, PhD \\
1567.~{\bf Robert Dipert}, Particle physics PhD candidate, {\sl Arizona State University} \\
1568.~{\bf Natalie Gosnell}, PhD, {\sl University of Texas at Austin} \\
1569.~{\bf Dr. Javier del Castillo}, Dr. Gynological Medicine, {\sl Harlingen, Tx} \\
1570.~{\bf Jose Vithayathil}, PhD, {\sl University of Pennsylvania} \\
1571.~{\bf William Detmold}, {\sl MIT} \\
1572.~{\bf Steven Torrisi}, Undergraduate Physics/Math Student, {\sl University of Rochester} \\
1573.~{\bf Dan Weisz}, PhD, {\sl University of California, Berkeley} \\
1574.~{\bf Hector Acaron}, MS, PhD candidate, {\sl University of Chicago} \\
1575.~{\bf Peter B. Boyce}, Past Executive Officer, {\sl American  Astronomical Society} \\
1576.~{\bf Arriety Lowell}, MS, {\sl Unviersity of Wisconsin-River Falls} \\
1577.~{\bf Farah Tavakoli} \\
1578.~{\bf Ayush Gupta}, Assistant Research Professor, {\sl University of Maryland} \\
1579.~{\bf Pete McCabe}, Resident Dramaturg, {\sl HERE Arts Center} \\
1580.~{\bf Erica Price}, MEd, {\sl Cherry Hill Public Schools} \\
1581.~{\bf William Thomas Machi}, Teacher, {\sl Apollo Education, Hai Phong, Vietnam} \\
1582.~{\bf Joshua McNeur}, PhD, {\sl Friedrich Alexander University} \\
1583.~{\bf Ken Shen}, PhD, {\sl UC Berkeley} \\
1584.~{\bf Kris McDaniel-Miccio}, Professor, {\sl Univerisity of Denver, Sturm College of Law} \\
1585.~{\bf Jacob Johnson}, Undergraduate, {\sl University of Washington} \\
1586.~{\bf Laura Blecha}, PhD, {\sl University of Maryland} \\
1587.~{\bf Glenn Lopez}, PhD, {\sl CIT Bank} \\
1588.~{\bf Emilie Huffman}, PhD Student, {\sl Duke University} \\
1589.~{\bf Yvette G Flores}, PhD, {\sl UC Davis} \\
1590.~{\bf Phil Brady}, BS, Physics, {\sl Washington State Government} \\
1591.~{\bf Steven Gourrier}, Principal, {\sl South Early College High School} \\
1592.~{\bf Kalee Tock}, MS, {\sl Stanford University} \\
1593.~{\bf Kate Follette}, PhD, {\sl Stanford University} \\
1594.~{\bf Lynne Raschke}, PhD, {\sl The College of St. Scholastica} \\
1595.~{\bf Eric Knuth}, Professor, {\sl University of Wisconsin-Madison} \\
1596.~{\bf Thomas Crawford}, PhD, {\sl University of Chicago} \\
1597.~{\bf Brendan P. Bowler}, PhD, {\sl University of Texas at Austin} \\
1598.~{\bf William DeGraffenreid}, PhD, Professor of Physics and Department Chair, {\sl California State University, Sacramento} \\
1599.~{\bf Kayla Furukawa}, Female BS in Physics graduate, {\sl Blueprint Consulting Services} \\
1600.~{\bf Jessica Esquivel}, Graduate Student, PhD Physics, {\sl Syracuse University } \\
1601.~{\bf Arlene Maclin}, PhD, Adjunct Professor of Physics and Exec. Director, MACCAE Program, {\sl Morgan State University} \\
1602.~{\bf Michael Page}, PhD, Associate Professor, {\sl Cal Poly Pomona University} \\
1603.~{\bf James Hesser}, Director Emeritus, {\sl NRC Canada} \\
1604.~{\bf Jamil Alexis}, Medical student, {\sl Hofstra NS-LIJ School of Medicine } \\
1605.~{\bf Enrique Gamez}, PhD Student, {\sl University of Michigan} \\
1606.~{\bf Troy Messina}, PhD, {\sl Berea College} \\
1607.~{\bf Meredith Durbin}, Research and Instrument Analyst, {\sl STScI} \\
1608.~{\bf Miguel Centellas}, PhD, {\sl University of Mississippi } \\
1609.~{\bf Maria Falbo}, PhD, {\sl Elon University} \\
1610.~{\bf Kareem Kazkaz}, PhD, {\sl Lawrence Livermore National Laboratory} \\
1611.~{\bf Othell Begay}, BA Biology, {\sl N/A} \\
1612.~{\bf Christopher M. Hirata}, Professor of Physics and Astronomy, {\sl The Ohio State University} \\
1613.~{\bf Louis-Francois Arsenault}, postdoctoral research scientist, {\sl Columbia University} \\
1614.~{\bf Nebiyou Samuel}, 1st Year Medical Student, {\sl UC Sandiego School of Medicine} \\
1615.~{\bf Shyamala Ratnayeke}, Associate Professor of Biology, {\sl Sunway University, Malaysia} \\
1616.~{\bf Olgy Gary}, M.A., PhD candidate, {\sl doctoral student at Regent University} \\
1617.~{\bf Andre de Gouvea}, Professor, {\sl Northwestern University} \\
1618.~{\bf Timothy B. Hayward}, PhD candidate, {\sl William and Mary} \\
1619.~{\bf Aaron M. Datesman}, Ph.D., {\sl NASA Goddard Space Flight Center} \\
1620.~{\bf Leslie Rogers}, PhD, {\sl UC Berkeley/University of Chicago} \\
1621.~{\bf Roger Gonnering}, MSEE, MSCS, MBA \\
1622.~{\bf Mable Anthony}, Political Activist, {\sl Retired} \\
1623.~{\bf Rodney Sadler}, Professor, {\sl Union Presbyterian Seminary} \\
1624.~{\bf Kenneth Tyler}, PhD, {\sl University of Kentucky } \\
1625.~{\bf Tony Li}, Graduate Student, {\sl Stanford University} \\
1626.~{\bf MacKenzie Warren}, PhD Candidate, {\sl University of Notre Dame} \\
1627.~{\bf Lynn Rathbun}, PhD, {\sl Cornell University} \\
1628.~{\bf William Dirienzo}, Professor, {\sl University of Wisconsin Colleges} \\
1629.~{\bf Thomas Leith}, AB Candidate in Astrophysics, {\sl Harvard College} \\
1630.~{\bf David Drucker}, BS, {\sl IATSE local 720} \\
1631.~{\bf Anna Phillips}, PhD student, {\sl Tufts} \\
1632.~{\bf Brent Barker}, Professor, {\sl Roosevelt University} \\
1633.~{\bf Shawn Westerdale}, {\sl Princeton University} \\
1634.~{\bf Sara Freedman}, Ed.D. \\
1635.~{\bf Nirupama Shrudhar}, PhD, {\sl Lake Washington Technical Institute} \\
1636.~{\bf Dina Willner} \\
1637.~{\bf Daryl Haggard}, Professor, {\sl McGill University} \\
1638.~{\bf Jennifer Scott}, PhD, {\sl Towson University} \\
1639.~{\bf Nicholas G. Heavens}, Research Assistant Professor of Planetary Science, {\sl Hampton University} \\
1640.~{\bf Kenneth L. Graham}, PhD, Senior Research Engineer, {\sl Digital Reasoning Systems} \\
1641.~{\bf Ernest Beaulieu}, History Dept. Chair, {\sl Chase Collegiate School} \\
1642.~{\bf Brian Colding}, Educator \\
1643.~{\bf Zachary A. LaBry}, PhD \\
1644.~{\bf Katherine Zaunbrecher}, PhD, {\sl National Renewable Energy Laboratory} \\
1645.~{\bf Adrienne Leonard}, PhD, Marie Curie Fellow, {\sl University College London} \\
1646.~{\bf Amanda Walker}, Aerospace Engineering Student, {\sl University of Texas at Austin} \\
1647.~{\bf Sarah Jane Schmidt}, PhD, {\sl Leibniz Institute for Astrophysics, Potsdam} \\
1648.~{\bf Ana Larson}, PhD, {\sl University of Washington} \\
1649.~{\bf Dan Schwartz}, PhD, {\sl SAO} \\
1650.~{\bf Steven Dale Theiss}, PhD, {\sl 3M Company} \\
1651.~{\bf Sahar Allam}, PhD \\
1652.~{\bf Bradly Alicea}, PhD, {\sl University of Illinois  Urbana-Champaign} \\
1653.~{\bf Cynthia Keeler}, PhD, {\sl Niels Bohr Institute} \\
1654.~{\bf John Konopak}, PhD, {\sl (Retired: Formerly LSU and UofOkla)} \\
1655.~{\bf Erin T Woodds}, JD \\
1656.~{\bf Jacob Disston}, MS, Math/Science Teacher Credential Program Director, {\sl Graduate School of Education, U.C. Berkeley} \\
1657.~{\bf Tom Pundsack}, PhD, {\sl Pace Analytical} \\
1658.~{\bf Kevin McDermott}, PhD student, {\sl Cornell University} \\
1659.~{\bf Arthur A. Evans}, PhD, {\sl U Wisconsin, Madison} \\
1660.~{\bf Carolyn Atkinson}, student, {\sl Pomona College} \\
1661.~{\bf Douglas Caldwell}, PhD Research Astrophysicist, {\sl SETI Institute} \\
1662.~{\bf Charles Stephens}, MA, {\sl TAMUC, CCCC} \\
1663.~{\bf Mathew Madhavacheril}, {\sl Stony Brook University} \\
1664.~{\bf Dorothy}, instructor, {\sl retired} \\
1665.~{\bf John Shaw}, PhD, {\sl Louisiana Tech University} \\
1666.~{\bf Melinda Andrews}, Professor, {\sl Hamilton College} \\
1667.~{\bf Edward S. Grood}, Emeritus Professor, Biomedical Engineering, {\sl University of Cincinnati} \\
1668.~{\bf Emily A. Kraus}, PhD Student, {\sl University of Pennsylvania} \\
1669.~{\bf Robert S. Mahurin}, PhD, {\sl Middle Tennessee State University} \\
1670.~{\bf Adam Patch}, Graduate Fellow, {\sl Syracuse University} \\
1671.~{\bf Ramgopal Mettu}, Professor, {\sl Tulane University} \\
1672.~{\bf James Rovira}, PhD, English, {\sl Tiffin University} \\
1673.~{\bf Rodney Hopson}, Professor, {\sl George Mason University} \\
1674.~{\bf Lionel Johnnes}, B.A., Physics, {\sl Washington University in St. Louis} \\
1675.~{\bf Daniel Gershman}, PhD, {\sl University of Maryland, College Park} \\
1676.~{\bf Howard Haber}, Professor of Physics, {\sl University of California, Santa Cruz} \\
1677.~{\bf Elliot Kaplan}, PhD, {\sl Institut des Sciences de la Terre} \\
1678.~{\bf Martijn Mulders}, PhD, {\sl CERN} \\
1679.~{\bf Molly Shea}, PhD, {\sl Exploratorium} \\
1680.~{\bf Chris Zin}, PhD Student, {\sl Wayne State University} \\
1681.~{\bf Judith DeBose}, MD \\
1682.~{\bf Mohan Srinivasarao}, Professor, {\sl Georgia Insitute of Technology} \\
1683.~{\bf Rebecca Farber} \\
1684.~{\bf Nicholas McConnell}, PhD \\
1685.~{\bf Rosalie M Romano}, PhD, {\sl Western Washington University} \\
1686.~{\bf Joe Milazzo}, MLS, MFA, {\sl Southern Methodist University} \\
1687.~{\bf Giada Arney}, MS, {\sl University of Washington } \\
1688.~{\bf Artur Marek Ankowski}, PhD, {\sl Virginia Tech} \\
1689.~{\bf Khin Maung Maung}, Professor, {\sl The University of Southern Mississippi} \\
1690.~{\bf Davida Brown}, PhD, {\sl George Fox University } \\
1691.~{\bf Nathan Nakatsuka}, MDPhD student at Harvard Medical School, {\sl Systems Biology PhD program} \\
1692.~{\bf Robert Blum}, PhD, Deputy Director, {\sl NOAO} \\
1693.~{\bf Holger Meyer}, Professor and Director of Physics, {\sl Wichita State University} \\
1694.~{\bf Paul DeCunzo}, Science Educator; NGSS Teacher facilitator, {\sl California State University, Northridge (CSUN)} \\
1695.~{\bf Thomas R. Lightfoot}, EdD, Associate Professor, {\sl Rochester Institute of Technology} \\
1696.~{\bf Cassie Reuter}, Ph.D. Candidate, {\sl Purdue University} \\
1697.~{\bf Julia Aguirre}, Associate Professor, {\sl University of Washington Tacoma} \\
1698.~{\bf Matthew R Hermes}, PhD in physical chemistry, {\sl University of Illinois at Urbana-Champaign} \\
1699.~{\bf Emmet Golden-Marx}, Graduate Student, {\sl BU} \\
1700.~{\bf Donnell Nichols}, Consultant EMI Research, {\sl EMI SIG} \\
1701.~{\bf Sharon Cheryl Onga Nana}, Law student, {\sl Roger Williams University School of Law} \\
1702.~{\bf Cathleen Fry}, PhD candidate, {\sl Michigan State University} \\
1703.~{\bf Lynford Goddard}, Associate Professor of Electrical and Computer Engineering, {\sl University of Illinois at Urbana-Champaign} \\
1704.~{\bf Clifford V. Johnson}, Professor of Physics, {\sl University of Southern California} \\
1705.~{\bf Tara Javidi}, Professor of Engineering, {\sl UC San Diego} \\
1706.~{\bf Jonathan Williams}, Professor, {\sl University of Hawaii} \\
1707.~{\bf Chi-Tai Tang}, PhD, {\sl Utah Bureau of Forensic Services} \\
1708.~{\bf Amelious Whyte}, Ph.D., {\sl University of Minnesota} \\
1709.~{\bf Edmund Bertschinger}, Professor of Physics, {\sl MIT} \\
1710.~{\bf Ethan A. Romans}, Engineer, {\sl Aerospace} \\
1711.~{\bf Thomas F. Haff}, AAPT Fellow, {\sl Ballard High School} \\
1712.~{\bf Linda Samelson}, Ms, {\sl Retired educator} \\
1713.~{\bf Eddie M Abbott}, MD, {\sl retired} \\
1714.~{\bf Martha-Elizabeth Baylor}, PhD, Assistant Professor of Physics, {\sl Carleton College} \\
1715.~{\bf Brittany Burton}, MS II, {\sl UCSD School of Medicine} \\
1716.~{\bf Eric D. Miller}, PhD, {\sl MIT Kavli Institute for Astrophysics and Space Research} \\
1717.~{\bf Grant Larsen}, Professor, {\sl Principia College} \\
1718.~{\bf Wendi Wampler}, PhD, Assistant Professor of Physics and Engineering, {\sl Central Oregon Community College} \\
1719.~{\bf Lee May}, Clergy, {\sl Associate Minister} \\
1720.~{\bf Josh Veazey}, PhD, {\sl Grand Valley State University} \\
1721.~{\bf Nicholas Murphy}, Dr., {\sl Smithsonian Astrophysical Observatory} \\
1722.~{\bf Roy Kilgard}, Research Associate Professor, {\sl Wesleyan University} \\
1723.~{\bf Matthew Titsworth}, PhD, {\sl University of Texas at Dallas} \\
1724.~{\bf Mirah Gary}, PhD, {\sl Institute for Theoretical Physics, TU Wien} \\
1725.~{\bf Ralph Wijers}, Professor, {\sl University of Amsterdam} \\
1726.~{\bf Steve Bryson}, PhD, {\sl NASA Ames Research Center} \\
1727.~{\bf Jay Banks} \\
1728.~{\bf Erin Lee Ryan}, PhD, {\sl University of Maryland} \\
1729.~{\bf Laura Leg\'{e}}, Professor, {\sl California State University Sacramento } \\
1730.~{\bf Laura Zschaechner}, PhD, {\sl Max Planck Institute for Astronomy} \\
1731.~{\bf Keely Finkelstein}, PhD, {\sl University of Texas at Austin} \\
1732.~{\bf Roy Gal}, Associate Astronomer, {\sl University of Hawaii} \\
1733.~{\bf Brian Holton}, Ed.D., {\sl Middlesex County College} \\
1734.~{\bf Alejandro L. Garcia}, Professor, {\sl San Jose State University} \\
1735.~{\bf Andrew Duncan}, Associate Professor, {\sl Willamette University} \\
1736.~{\bf Deborah Appel Harris}, PhD, {\sl Fermi National Accelerator Laboratory} \\
1737.~{\bf Jai Salzwedel}, MS, {\sl The Ohio State University} \\
1738.~{\bf Eric Jensen}, Professor of Astronomy, {\sl Swarthmore College} \\
1739.~{\bf Bradley M. Peterson}, Professor Emeritus, {\sl The Ohio State University} \\
1740.~{\bf Ellen Rose Robinson}, MS \\
1741.~{\bf Tashalee Billings}, {\sl University of Pennsylvania } \\
1742.~{\bf Andrea Guajardo}, MPH, {\sl University of the Incarnate Word} \\
1743.~{\bf Rachel Reddick}, PhD, {\sl Robert Bosch LLC} \\
1744.~{\bf Christian Santangelo}, Associate Professor, {\sl University of Massachusetts Amherst} \\
1745.~{\bf Sandy Grande}, Chair, Education Department \\
1746.~{\bf Javier Garcia}, PhD, {\sl Harvard-Smithsonian Center for Astrophysics} \\
1747.~{\bf Bill Kelly}, {\sl Fine art image} \\
1748.~{\bf Carlo Dallapiccola}, Professor, {\sl University of Massachusetts, Amherst} \\
1749.~{\bf Mariana Guerrero}, PhD, Clinical Associate Professor, {\sl University of Maryland} \\
1750.~{\bf Adrian Meza}, BS Aerospace Engineering, {\sl Synergy Academies: K-12 Teacher} \\
1751.~{\bf Victor Chudnovsky}, PhD, {\sl Google} \\
1752.~{\bf Enrique Rueda} \\
1753.~{\bf Kristine Callan}, PhD, {\sl Colorado School of Mines} \\
1754.~{\bf Ursula Copeland}, MSW \\
1755.~{\bf Walter Sedriks}, PhD, BSc, ACGI, C.Eng, {\sl Stanford Research Institute (Retired)} \\
1756.~{\bf Dedra Demaree}, PhD, {\sl Freelance} \\
1757.~{\bf Nina Trivedi}, PhD, {\sl Royal College of Art } \\
1758.~{\bf Arthur Lipstein}, PhD \\
1759.~{\bf Luis G. Rosa}, PhD, Professor, {\sl University of Puerto Rico - Humacao} \\
1760.~{\bf Quinton L. Williams}, Professor and Chair, {\sl Howard University} \\
1761.~{\bf Chanda Prescod-Weinstein}, PhD, {\sl MIT} \\
1762.~{\bf Keaton J. Bell}, Astronomer, {\sl The University of Texas at Austin} \\
1763.~{\bf Kevin Moore}, Visiting Assistant Professor, {\sl W.M. Keck Science Dept., Claremont McKenna, Pitzer, and Scripps Colleges} \\
1764.~{\bf Deborah Grojean}, ms, {\sl retired} \\
1765.~{\bf Crystal Noel}, PhD Student, {\sl University of California Berkeley } \\
1766.~{\bf Alice Olmstead}, MS, {\sl University of Maryland, Department of Astronomy} \\
1767.~{\bf Jacob Stanley}, PhD in Physics, {\sl CU Boulder} \\
1768.~{\bf Deatrick Foster}, PhD, {\sl Space Telescope Science Institute} \\
1769.~{\bf R Sekhar Chivukula}, Professor of Physics, {\sl Michigan State University} \\
1770.~{\bf Gyorgyi Voros}, Dr., {\sl Virginia Tech} \\
1771.~{\bf Jacob A. Ketchum}, PhD \\
1772.~{\bf Craig Hetherington}, PhD, {\sl Lawrence Berkeley National Laboratory} \\
1773.~{\bf Lou Nigra}, PhD, {\sl Adler Planetarium} \\
1774.~{\bf David Lerner}, Professor Emeritus, {\sl University of Kansas } \\
1775.~{\bf Trevor LaMountain}, PhD Student, {\sl Northwestern University} \\
1776.~{\bf Joshua Burrow}, Electro-Optics MS Candidate, {\sl University of Dayton} \\
1777.~{\bf Katherine Silkaitis}, {\sl Harvard School of Public Health} \\
1778.~{\bf Martin E. Cobern}, VP, R\&D - retired, {\sl APS Technology, Inc.} \\
1779.~{\bf Justin T. Schultz}, PhD candidate, {\sl University of Rochester} \\
1780.~{\bf Ali Bramson}, {\sl University of Arizona} \\
1781.~{\bf Candice M. Etson}, Assistant Professor, {\sl Wesleyan University} \\
1782.~{\bf Michael A. Miller}, PhD, {\sl Philips} \\
1783.~{\bf Bruce Farah}, Mr., {\sl Springfield Public Schools } \\
1784.~{\bf Peter Teuben}, PhD, {\sl U of Maryland} \\
1785.~{\bf Leif Segen}, Physicist and physics teacher, {\sl John F. Kennedy Memorial High School} \\
1786.~{\bf Marissa Bellino}, Science educator, {\sl Brooklyn College } \\
1787.~{\bf Chin-Hao Chen}, PhD \\
1788.~{\bf Talea L Mayo}, PhD, Assistant Professor, {\sl University of Central Florida} \\
1789.~{\bf Jeffrey Silverman}, PhD, {\sl University of TX at Austin} \\
1790.~{\bf Lauren Alsberg}, {\sl SLAC National Accelerator Laboratory} \\
1791.~{\bf Thomas H. Appleton}, {\sl Alta Bates Summit Medical Center} \\
1792.~{\bf Azure Hansen}, PhD, {\sl University of Rochester} \\
1793.~{\bf Bart M. Bartlett}, Associate Professor of Chemistry, {\sl University of Michigan} \\
1794.~{\bf Joshua E. Schlieder}, PhD, {\sl NASA Ames Research Center} \\
1795.~{\bf Elana Voigt}, Scientific Instructional Technician, Physics Labs, {\sl University of Washington } \\
1796.~{\bf Kristine Rezai}, Graduate student, {\sl Harvard university} \\
1797.~{\bf Molly Martin}, PhD in progress, {\sl University of Miami, RSMAS} \\
1798.~{\bf Andreea Petric}, PhD, {\sl Gemini Observatory} \\
1799.~{\bf Stephanie Barr}, Senior Project Engineer, BS Engineering Physics \\
1800.~{\bf Andrew Baden}, Professor and Chair, {\sl University of Maryland, College Park} \\
1801.~{\bf Sean P. Hendrick}, Associate Professor and Chair of Physics, {\sl Millersville University} \\
1802.~{\bf Racquel Sherwood}, PhD, {\sl Yale University } \\
1803.~{\bf Claire M Watts}, PhD, {\sl Post-Doctoral Researcher, University of Washington} \\
1804.~{\bf Maryam Hazeghazam}, MD PhD, {\sl Hospital} \\
1805.~{\bf Brendan Casey}, PhD, {\sl Fermilab} \\
1806.~{\bf Marie Blatnik}, Graduate Student, {\sl CALTECH} \\
1807.~{\bf Nicolas B Cowan}, Professor, {\sl McGill University} \\
1808.~{\bf Wesley T Fuhrman}, PhD candidate, {\sl Johns Hopkins University} \\
1809.~{\bf Arnulfo Gonzalez}, MS, {\sl Texas A\&M University } \\
1810.~{\bf Jeanne A. Hardy}, PhD, Associate Professor, {\sl UMass Amherst} \\
1811.~{\bf Arthur Vandenberg}, MS, MVA, {\sl Artist} \\
1812.~{\bf David Tolchin}, Attorney, {\sl Jaroslawicz \& Jaros PLLC, New York} \\
1813.~{\bf Daniel Cebra}, Professor, {\sl University of California, Davis} \\
1814.~{\bf Sarah H Smith}, PhD, {\sl Providence College} \\
1815.~{\bf Patricia Carroll}, PhD candidate, {\sl University of Washington} \\
1816.~{\bf Gregory F. Snyder}, PhD \\
1817.~{\bf Kimberly A Shaw}, Full Professor of Physics, {\sl Columbus State University} \\
1818.~{\bf Aaron J Redd}, PhD, {\sl Philips Healthcare} \\
1819.~{\bf Anna Redden}, Associate Professor, {\sl Acadia University} \\
1820.~{\bf Daniel Nagasawa}, Graduate Student, {\sl Texas A\&M University} \\
1821.~{\bf Karen Blitz Shabbir}, DO, {\sl Holy Name Hospital } \\
1822.~{\bf Scott Collier}, PhD student, {\sl Harvard University} \\
1823.~{\bf Paul S. Nerenberg}, Research Assistant Professor, {\sl Caltech} \\
1824.~{\bf Megan Donahue}, Professor, {\sl Michigan State u} \\
1825.~{\bf Gregory A Good}, PhD, {\sl American Institute of Physics} \\
1826.~{\bf Jina Mendoza} \\
1827.~{\bf Sibyl R. Beaulieu}, Asst. Professor and Director of Field Education, {\sl Brandman University} \\
1828.~{\bf Katherine Campbell}, BA Physics, PhD Cardiac Electrophysiology, {\sl University of Michigan} \\
1829.~{\bf Lauranne Lanz}, PhD, {\sl California Institute of Technology} \\
1830.~{\bf Amy Rouinfar}, PhD, Physics, {\sl University of Colorado Boulder} \\
1831.~{\bf Christine J Benally}, PhD, {\sl IHS} \\
1832.~{\bf Lisa M. Will}, PhD \\
1833.~{\bf Sara Breslow}, PhD \\
1834.~{\bf Paul J. Heafner}, MS, Astronomy/Physics Instructor, {\sl Catawba Valley Community College} \\
1835.~{\bf Murray L Brown}, Associate Professor, {\sl Georgia State University} \\
1836.~{\bf Andrea Prestwich}, PhD, {\sl Smithsonian Astrophysical  Observatory} \\
1837.~{\bf Theresa Brennan-Hochstetler}, LMT, MA-C, {\sl Seattle Children's Hospital} \\
1838.~{\bf Stephanie Hamilton}, Graduate Student, {\sl University of Michigan} \\
1839.~{\bf Margaret Piper McNulty}, Adjunct Lecturer, {\sl De Anza College, Cupertino, CA} \\
1840.~{\bf Mark Richardson}, PhD, {\sl University of Oxford} \\
1841.~{\bf Nirav P. Mehta}, Assistant Professor, {\sl Trinity University, San Antonio Texas} \\
1842.~{\bf Kara Kundert}, {\sl UC Berkeley} \\
1843.~{\bf Lara Kassab}, Ed.D., Adjunct Professor of Teacher Education, {\sl San Jose State University, San Jose, California} \\
1844.~{\bf Fernanda Foertter}, {\sl ORNL} \\
1845.~{\bf Jim Schewe}, PhD, Principal Scientist, {\sl Philips } \\
1846.~{\bf Bruce Umbaugh}, Professor of Philosophy, {\sl Webster University } \\
1847.~{\bf George Gale}, Prof. of Philosophy \& Physical Science (emeritus), {\sl Univ. of Missouri-Kansas City} \\
1848.~{\bf David Kawall}, Associate Professor of Physics, {\sl University of Massachusetts Amherst} \\
1849.~{\bf Patrick J. Tool}, Student, {\sl George Washington University} \\
1850.~{\bf Jonathan Curtis}, Mr., {\sl University of Rochester } \\
1851.~{\bf Chelsea Bartram}, graduate student, {\sl UNC Chapel Hill} \\
1852.~{\bf Adam Johnston}, Professor, {\sl Weber State University} \\
1853.~{\bf Jacob Searcy}, PhD, {\sl University of Michigan} \\
1854.~{\bf Muhed Rana}, Graduate Student, {\sl University of Arizona} \\
1855.~{\bf Rachel Winter}, Undergraduate Student, {\sl Hamline University} \\
1856.~{\bf Timothy Stiles}, PhD Medical Physics, Associate Professor, {\sl Monmouth College} \\
1857.~{\bf Michelle Dolinski}, Assistant Professor, {\sl Drexel University} \\
1858.~{\bf Guy Ron}, Professor, {\sl Hebrew University of Jerusalem and The George Washington University} \\
1859.~{\bf Brad Ambrose}, PhD, {\sl Grand Valley State University} \\
1860.~{\bf Terence L Watts}, Professor of Physics \& Astronomy Emeritus, {\sl Rutgers University} \\
1861.~{\bf Charles Coleman}, PhD, {\sl Colorado Academy (retired)} \\
1862.~{\bf Jay Baltisberger}, Professor of Physical Chemistry, {\sl Berea College} \\
1863.~{\bf Bronwen Cohn-Cort}, BSc, {\sl American University graduate} \\
1864.~{\bf Johanna-Laina Fischer}, MS (soon to be PhD), {\sl University of Pennsylvania} \\
1865.~{\bf Meredith L. Rawls}, PhD candidate, {\sl New Mexico State University } \\
1866.~{\bf Kathy Opachich}, PhD Principal Scientist, {\sl National Security Technologies } \\
1867.~{\bf Brian W. Mulligan}, PhD candidate in Astrophysics, {\sl University of Texas at Austin} \\
1868.~{\bf Jennifer Steele}, PhD, {\sl Trinity University } \\
1869.~{\bf Tahir Yusufaly}, Postdoc, {\sl USC} \\
1870.~{\bf Matthew D. Multach}, {\sl Lifespan/Rhode Island Hospital} \\
1871.~{\bf Madeline Elkins}, PhD, {\sl University of California, Berkeley } \\
1872.~{\bf Elizabeth Gire}, Assistant Professor, {\sl Oregon State University} \\
1873.~{\bf Christopher Bennett}, phd, {\sl Georgia Institute of Technology} \\
1874.~{\bf Ami Radunskaya}, Professor of Mathematics, {\sl Pomona College} \\
1875.~{\bf Lilah G}, {\sl Casa Myrna Vazquez, Inc.} \\
1876.~{\bf Grant R. Tremblay}, Dr., {\sl Yale University} \\
1877.~{\bf Robert Blum}, PhD candidate, {\sl Yale University} \\
1878.~{\bf Cyrus Taylor}, Albert A. Michelson Professor in Physics \& Dean, College of Arts and Sciences, {\sl Case Western Reserve University} \\
1879.~{\bf Brian W Taylor}, M.S. Physics, {\sl Boston University} \\
1880.~{\bf Meryl Spencer}, PhD candidate, {\sl University of Michigan} \\
1881.~{\bf James Antonaglia}, BS, {\sl University of Michigan} \\
1882.~{\bf Richard E Ward}, Emeritus Professor, {\sl Indiana University} \\
1883.~{\bf Kelly Price}, BS Computer Engineering, MS Computer Science and Software Engineering, {\sl Auburn University} \\
1884.~{\bf Ted Fujimoto}, {\sl University of Pennsylvania} \\
1885.~{\bf Patrick P. Murphy}, PhD \\
1886.~{\bf Julia Fisher}, PhD \\
1887.~{\bf Noemi Waight, PhD}, Associate Professor of Science Education, {\sl University at Buffalo} \\
1888.~{\bf Laura Prichard}, PhD candidate, {\sl University of Oxford, UK} \\
1889.~{\bf Charles Buchanan}, Professor Emeritus, {\sl UCLA} \\
1890.~{\bf Michael Jewell}, Physics Major, {\sl Wayne State University} \\
1891.~{\bf Sandra B Hadley}, Adjunct professor, {\sl College of Southern Maryland} \\
1892.~{\bf Katelin Schutz}, PhD Student, {\sl Berkeley} \\
1893.~{\bf Jennifer Rittenhouse West}, MS, PhD student, {\sl University of California Irvine} \\
1894.~{\bf Kolby Weisenburger}, PhD Student, {\sl University of Washington} \\
1895.~{\bf Kathleen Padilla}, PhD, {\sl New Mexico VAHCS} \\
1896.~{\bf Emily Thompson}, PhD, {\sl Insight Data Science} \\
1897.~{\bf Alejandro Sanchez Alvarado}, PhD, {\sl Howard Hughes Medical Institute} \\
1898.~{\bf Rispba McCray-Garrison}, MD, MS, {\sl Houston Family Physicians } \\
1899.~{\bf David M. French}, Graduate Student, {\sl University of Wisconsin - Madison} \\
1900.~{\bf Ilana Ambrogi}, MD, {\sl FQHC} \\
1901.~{\bf Javad Shabani}, Professor, {\sl City College of New York} \\
1902.~{\bf Tina Cheuk}, Graduate Student, {\sl Stanford Graduate School of Education} \\
1903.~{\bf Rashid V. Williams-Garcia}, PhD candidate, {\sl Indiana University Bloomington} \\
1904.~{\bf Alejandra Le\'{o}n}, PhD, {\sl Osu alumni} \\
1905.~{\bf Vipul Gupta}, Mr., {\sl UC Berkeley} \\
1906.~{\bf Victor Madge}, Architect \\
1907.~{\bf Fletcher White}, Professor, {\sl Indiana University} \\
1908.~{\bf Jonathan Fraine}, PhD, {\sl University of Arizona} \\
1909.~{\bf Al Thompson}, PhD, {\sl retired} \\
1910.~{\bf Catherine Reed}, MA in Secondary Physical Science Education, {\sl Friends Academy} \\
1911.~{\bf Danielle Pahud}, PhD candidate, {\sl Boston University} \\
1912.~{\bf Jarle Brinchmann}, Associate Professor, {\sl Leiden University} \\
1913.~{\bf Alain Brizard}, Professor of Physics, {\sl Saint Michael's College} \\
1914.~{\bf Marinos Vouvakis}, Professor, {\sl University of Massachusetts Amherst} \\
1915.~{\bf Nestor Guillen}, PhD (University of Texas), {\sl University of Massachusetts} \\
1916.~{\bf Thaddeus Dobey}, MSHA, {\sl FDA} \\
1917.~{\bf James Bullock}, Professor, {\sl University of California, Irvine} \\
1918.~{\bf Baruti N. Kopano}, Ph.D., {\sl Morgan State University} \\
1919.~{\bf Jan Cami}, Professor, {\sl The University of Western Ontario} \\
1920.~{\bf Jure Zupan}, Prof. Dr., {\sl U. Cincinnati} \\
1921.~{\bf Kenneth Stephen Birch}, Associate Professor, {\sl Boston College} \\
1922.~{\bf Carol Hood}, Assistant Professor, {\sl California State University, San Bernardino} \\
1923.~{\bf Devin Silvia}, NSF Astronomy and Astrophysics Postdoctoral Fellow, {\sl Michigan State University} \\
1924.~{\bf Andrea Chen}, MEd, {\sl Propeller: a force for social innovation } \\
1925.~{\bf Parthiban Santhanam}, Founder and CEO, PhD, {\sl Exergy Dynamics, Inc.} \\
1926.~{\bf Mary McDonald}, MS, {\sl High School High School Mathematics and FIRST Robotics Coach} \\
1927.~{\bf Alex S. Hill}, Dr, {\sl Haverford College} \\
1928.~{\bf Antonio Maceo}, MSEE, {\sl Capitol Technology University} \\
1929.~{\bf John M. Perkins} \\
1930.~{\bf Caitlin Casey}, Assistant Professor, {\sl The University of Texas at Austin} \\
1931.~{\bf Gabrielle Mehta}, Student, {\sl Pomona College} \\
1932.~{\bf Martin Elvis}, PhD, {\sl Smithsonian Astrophysical Observatory} \\
1933.~{\bf Kristine Larsen}, PhD, {\sl Central Connecticut State University} \\
1934.~{\bf Jay Bowman-Kirigin}, MD/PhD student, {\sl Washington University in St. Louis Medical Scientist Training Program} \\
1935.~{\bf Jason Dexter}, Dr., {\sl Max Planck Institute for Extraterrestrial Physics} \\
1936.~{\bf Adam M. Jacobs}, PhD candidate, {\sl Stony Brook University} \\
1937.~{\bf Judith DeGroat}, PhD, {\sl St. Lawrence University} \\
1938.~{\bf Karen Masters}, PhD in Astronomy, {\sl Institute of Cosmology and Gravitation, University of Portsmouth} \\
1939.~{\bf Joyce Lang}, Educator, {\sl Blair Academy} \\
1940.~{\bf Alma Barranco}, PhD, {\sl Canadian Space Society} \\
1941.~{\bf Peter Behroozi}, PhD, {\sl Space Telescope Science Institute} \\
1942.~{\bf Molly Wassermann}, Senior Research Associate, {\sl The Metropolitan Opera} \\
1943.~{\bf Imran Khundkar}, J.D. Candidate, {\sl UC Irvine School of Law} \\
1944.~{\bf Kira Grogg}, PhD, {\sl Harvard Medical School } \\
1945.~{\bf Joshua Teves}, B.S. Expected May 2016, {\sl College of Charleston} \\
1946.~{\bf Stephanie Castillo}, JD Student, {\sl Santa Clara University School of Law} \\
1947.~{\bf Graeme Gossel}, PhD, {\sl New York} \\
1948.~{\bf Melody Maxson}, Graduate Student, {\sl Case Western Reserve University} \\
1949.~{\bf Janice Nelson}, {\sl SLAC national accelerator laboratory} \\
1950.~{\bf Katherine Stumpo}, Assistant Professor of Chemistry, {\sl University of Scranton} \\
1951.~{\bf Marc Weinberg}, PhD, {\sl Florida State University} \\
1952.~{\bf Teppei Katori}, PhD, {\sl Queen Mary University of London} \\
1953.~{\bf Bebo White}, Professor/Department Associate, {\sl SLAC National Accelerator Laboratory} \\
1954.~{\bf Lee Anne Willson}, PhD and University Professor Emerita, {\sl Iowa State University} \\
1955.~{\bf Scott Menor}, PhD, {\sl Roambotics, Inc.} \\
1956.~{\bf Anna Watts}, Associate Professor, {\sl University of Amsterdam } \\
1957.~{\bf Eric Perlman}, Professor, {\sl Florida Institute of Technology} \\
1958.~{\bf Surajit Sen}, PhD, {\sl SUNY Buffalo} \\
1959.~{\bf George Vejar}, Graduate Student, {\sl Vanderbilt University} \\
1960.~{\bf David Reiss}, PhD, {\sl Scientific Arts} \\
1961.~{\bf Jennifer Klay}, Associate Professor, {\sl California Polytechnic State University, San Luis Obispo} \\
1962.~{\bf Daniel Evans}, PhD, {\sl Antelope Valley College} \\
1963.~{\bf Lucy Archer}, BA, {\sl MIT} \\
1964.~{\bf Teresa Mayer}, MD, {\sl Th medical center of Auora, Colorado } \\
1965.~{\bf Alexander Gil}, PhD, {\sl Columbia University } \\
1966.~{\bf John Lucido}, PhD \\
1967.~{\bf Nathan Cook}, PhD, {\sl RadiaSoft LLC} \\
1968.~{\bf Ana Romero}, Electronic Engineer/interpreter/translator, {\sl ExactLingua, LLC} \\
1969.~{\bf Karen Coulter}, MS, {\sl University of Michigan } \\
1970.~{\bf Michael Maseda}, PhD, {\sl Leiden University} \\
1971.~{\bf Evelyn Van Til}, Ms, {\sl Ohio Department of Education} \\
1972.~{\bf Chris Mihos}, Professor, {\sl Case Western Reserve University} \\
1973.~{\bf Maureen Foss}, MS, {\sl Private Practice} \\
1974.~{\bf Roger A. Freedman}, Lecturer in Physics, {\sl University of California, Santa Barbara} \\
1975.~{\bf Ineke Abunawass}, {\sl University of West Georgia } \\
1976.~{\bf Olivia Mello}, Graduate Student, {\sl Harvard University} \\
1977.~{\bf Aaron Hagerstrom}, PhD, {\sl University of Maryland} \\
1978.~{\bf Elena Sabbi}, PhD, {\sl STScI} \\
1979.~{\bf Sharon Xuesong Wang}, {\sl Penn State University} \\
1980.~{\bf Stephanie Strazisar}, PhD, {\sl Covestro} \\
1981.~{\bf Micha Kilburn}, PhD, {\sl University of Notre Dame} \\
1982.~{\bf Rachel Craigmile}, M.S., {\sl University of Texas at Austin} \\
1983.~{\bf Ximena Fern\'{a}ndez}, PhD, {\sl Rutgers University} \\
1984.~{\bf Sarah Church}, Professor, {\sl Stanford University} \\
1985.~{\bf Juliana N.Anyanwu}, MD, MPH \\
1986.~{\bf Heshan "Grasshopper" Illangkoon}, PhD, {\sl Blue Marble Space Institute of Science \& The Grasshopper Group} \\
1987.~{\bf Sebastian Fischetti}, PhD, {\sl Imperial College London} \\
1988.~{\bf Amihan Huesmann}, PhD, {\sl University of Wisconsin-Madison} \\
1989.~{\bf Alan Jackson}, MS physics, {\sl Shell Oil Company  (retired ) } \\
1990.~{\bf Caroline E. Simpson}, PhD; Professor of Physics, {\sl Florida International University} \\
1991.~{\bf Todd Spindler}, PhD, {\sl NOAA National Weather Service} \\
1992.~{\bf Manuel Paul Pe\~{n}a}, Ed.D, Science Special Edication Resource Teacher, {\sl Minneapolis Public Schools} \\
1993.~{\bf Shermane Benjamin}, PhD candidate (Physics), {\sl Florida State University/ National High Magnetic Field Laboratory} \\
1994.~{\bf Armin Rest}, PhD, {\sl Space Telescope Science Institute} \\
1995.~{\bf Nicholas E. Thornburg}, PhD of Chemical Engineering, {\sl Northwestern University} \\
1996.~{\bf Guy Indebetouw}, Prof. Emeritus, {\sl Virginia Tech} \\
1997.~{\bf Karen Collins}, PhD, {\sl Vanderbilt University} \\
1998.~{\bf Ian T. Durham}, PhD, Professor and Chair of Physics, {\sl Saint Anselm College} \\
1999.~{\bf Kenneth Varner}, PhD, {\sl Louisiana State University} \\
2000.~{\bf Vann Priest}, PhD, {\sl Rio Hondo College} \\
2001.~{\bf William Wootters}, Professor of Physics, {\sl Williams College} \\
2002.~{\bf Jessie Shelton}, Assistant Professor, {\sl U. Illinois Urbana-Champaign} \\
2003.~{\bf Daniel Cox}, Distinguished Professor, {\sl University of California, Davis} \\
2004.~{\bf Daniel Wolf}, BSc \\
2005.~{\bf Roslyn Shapiro} \\
2006.~{\bf Cheryl B.}, average citizen, {\sl planet Earth} \\
2007.~{\bf Lauren M Taylor}, Director of Educator Associates, {\sl American Inst of Aeronautics and Astronautics, Bay Area Chapter} \\
2008.~{\bf Bjorg Larson}, PhD, {\sl Drew University } \\
2009.~{\bf Kimberly Lokovic}, MS \\
2010.~{\bf Regina Caputo}, PhD, {\sl UCSC} \\
2011.~{\bf Aaron Pearlman}, PhD \\
2012.~{\bf Linda Green}, PhD Biology, {\sl Georgia Institute of Technology} \\
2013.~{\bf Peter Madigan}, BS Physics, {\sl University of Colorado Boulder} \\
2014.~{\bf Mark Gonzalez}, MS \\
2015.~{\bf Ronda Wery}, PhD, {\sl Klamath Community College} \\
2016.~{\bf Diana Parno}, Acting Assistant Professor, {\sl University of Washington} \\
2017.~{\bf Kalina Slavkova}, Student, {\sl University of Pennsylvania} \\
2018.~{\bf Sarah Martell}, Dr, {\sl University of New South Wales, Sydney, Australia} \\
2019.~{\bf Kathryn Spitzer Kim}, Certified Genetic Counselor, {\sl Stanford University} \\
2020.~{\bf Joshua Paul Tan}, PhD, {\sl Pontificia Universidad Cat\'{o}lica de Chile} \\
2021.~{\bf Martin Fraas}, PhD \\
2022.~{\bf Naomi A. Ridge}, PhD, Assistant Professor of Physics, {\sl Wentworth Institute of Technology} \\
2023.~{\bf Macheo Payne}, Assistant Professor, {\sl Cal State University East Bay} \\
2024.~{\bf Andr\'{e}s L\'{o}pez-Sepulcre}, PhD, {\sl CNRS} \\
2025.~{\bf Benjamin Richardson}, PhD Student, {\sl UIC} \\
2026.~{\bf Sam Towers}, MS, {\sl Northern Michigan University } \\
2027.~{\bf Glenda Howell}, PhD candidate, {\sl Don Bosco School of English} \\
2028.~{\bf Matthew Drake}, High School Physics Teacher, {\sl Palmer High School, Massachusetts} \\
2029.~{\bf William R. McGehee}, PhD \\
2030.~{\bf Daniel Castro}, PhD, {\sl NASA Goddard Space Flight Center} \\
2031.~{\bf Terra Caldwell}, MA, teacher, {\sl Edendale Middle School, CA} \\
2032.~{\bf David Rubin} \\
2033.~{\bf Erin Cox}, Graduate Fellow, {\sl University of Illinois} \\
2034.~{\bf Simon Kheifets}, PhD, {\sl Harvard University} \\
2035.~{\bf M. Virginia McSwain}, PhD, {\sl Lehigh University} \\
2036.~{\bf Tessa Johnson}, PhD, {\sl University of California Davis} \\
2037.~{\bf Concepcion del Castillo}, Dr. Anthropologists, {\sl World bank} \\
2038.~{\bf J. Sebastian Pineda}, PhD Student, {\sl Caltech} \\
2039.~{\bf Patrick Charbonneau}, Associate Professor of Chemistry and Physics, {\sl Duke University} \\
2040.~{\bf Rosalie McGurk}, PhD Astrophysics, {\sl Max Planck Institute for Astronomy, Germany} \\
2041.~{\bf Eric Rowley}, Ph.D., {\sl Wright State University} \\
2042.~{\bf Carolina Alvarado}, PhD, {\sl University of Maine} \\
2043.~{\bf Adrian Liu}, PhD, {\sl University of California Berkeley} \\
2044.~{\bf Darryl L. Corey}, PhD, {\sl Radford University } \\
2045.~{\bf Eugenia Etkina}, Professor, {\sl Rutgers} \\
2046.~{\bf Benjamin Dreyfus}, PhD, {\sl University of Maryland} \\
2047.~{\bf Tom Burtonwood}, Assistant Professor, {\sl The School of the Art Institute of Chicago} \\
2048.~{\bf Protik Majumder}, Professor of Physics, {\sl Williams College} \\
2049.~{\bf Tiara Diamond}, PhD, Postdoctoral Fellow, {\sl NASA Goddard Space Flight Center} \\
2050.~{\bf Justin Blair}, {\sl University of Texas at Austin} \\
2051.~{\bf H. Lee Sawyer}, Professor and Program Director, Physics, {\sl Louisiana Tech University} \\
2052.~{\bf Lia Corrales}, PhD, Postdoctoral Associate, {\sl MIT Kavli Institute for Astrophysics and Space Research} \\
2053.~{\bf Michael C. Weston JD}, Vice President and General Counsel (Retired), {\sl Northwestern University} \\
2054.~{\bf Ravi Gupta}, PhD, {\sl Argonne National Laboratory} \\
2055.~{\bf Francis Ge}, Biology and studio art student \\
2056.~{\bf Lauren Chai}, {\sl Massachusetts Institute of Technology} \\
2057.~{\bf Dennis Shields}, Chancellor, {\sl University of Wisconsin-Platteville} \\
2058.~{\bf Gaurav Chaudhary}, PhD student, {\sl University of Texas, Austin} \\
2059.~{\bf James Formato} \\
2060.~{\bf David Sliski}, {\sl University of Pennsylvania } \\
2061.~{\bf Susan Hans}, Ph.D., {\sl Self-employed} \\
2062.~{\bf Sarah N. Soisson}, PhD, {\sl Sandia National Laboratories} \\
2063.~{\bf Josh Taton}, PhD cand, math education; BA, math, {\sl UPenn} \\
2064.~{\bf Seyda Ipek}, Postdoc, {\sl Fermi National Laboratory} \\
2065.~{\bf S. Douglas Marcum}, Emeritus Professor of Physics, {\sl Miami University} \\
2066.~{\bf Emily Kading}, Grad Student, {\sl UConn} \\
2067.~{\bf Rev. Dr. Georgiana Kugle}, {\sl Retired Clergy} \\
2068.~{\bf Brian Kruse}, MS, {\sl Astronomical Society of the Pacific} \\
2069.~{\bf Frederick Aldama}, Professor, {\sl The Ohio State University} \\
2070.~{\bf Emmanuel Grotheer}, MS, {\sl University of Texas at San Antonio} \\
2071.~{\bf Randolf Klein}, PhD, {\sl NASA Ames} \\
2072.~{\bf Phyllis Terry Friedman, PhD}, Clinical Professor, {\sl Saint Louis University} \\
2073.~{\bf Michaela Kleinert}, Associate Professor of Physics, {\sl Willamette University, Salem, OR} \\
2074.~{\bf Francis W. Starr}, Professor of Physics; Director of the College of Integrative Sciences, {\sl Wesleyan University} \\
2075.~{\bf Dr. Lourdes del Castillo}, Dr. Social work, {\sl Univ. of SC} \\
2076.~{\bf Jane Rigby}, Ph.D., {\sl (Personal Capacity)} \\
2077.~{\bf Karen Daniels}, Professor of Physics, {\sl North Carolina State University} \\
2078.~{\bf Anashe Bandari}, MS, {\sl College of William \& Mary} \\
2079.~{\bf Mark Sarisky}, Professor, {\sl Guilford Technical College} \\
2080.~{\bf Jon Saderholm}, PhD, {\sl Berea College} \\
2081.~{\bf Larissa J. Estes}, DrPH, {\sl Walden University; Texas Tech University School of Nursing; Prevention Institute } \\
2082.~{\bf Sarah Rugheimer}, PhD \\
2083.~{\bf Jielai Zhang}, Ms, {\sl University of Toronto} \\
2084.~{\bf James D. Oliver III}, MD, PhD \\
2085.~{\bf Hume A. Feldman}, Professor and Chair, {\sl University of Kansas, Department of Physics \& Astronomy} \\
2086.~{\bf Jayadev Athreya}, Associate Professor of Mathematics, {\sl University of Washington} \\
2087.~{\bf Jaime Gonzalez}, Pharm.D, {\sl Hospital} \\
2088.~{\bf Adam Zakheim}, MSc. Microbiology, {\sl The William Penn Charter School} \\
2089.~{\bf Eric C. Martell}, Associate Professor of Physics, {\sl Millikin University } \\
2090.~{\bf Jolene Johnson}, PhD, {\sl St. Catherine University} \\
2091.~{\bf Emily McLinden}, PhD, {\sl University of Texas at Austin } \\
2092.~{\bf Matt Thibault}, PMP, {\sl Fred Hutch Cancer Research Center} \\
2093.~{\bf Ranjan Dharmapalan}, {\sl Argonne National Laboratory} \\
2094.~{\bf Jennifer Whalen}, MS, Physics \\
2095.~{\bf Reinaldo Mario Machado}, PhD Chemical Engineering, {\sl Air Products and Chemicals, Inc.} \\
2096.~{\bf Myriam P. Sarachik}, Distinguished Prof of Physics, {\sl City College of New York, CUNY} \\
2097.~{\bf John Caraher}, Associate Professor of Physics \& Astronomy, {\sl DePauw University} \\
2098.~{\bf Charlie Doret}, Assistant Professor, {\sl Williams College} \\
2099.~{\bf Andrew Yu}, BS, {\sl UC San Francisco} \\
2100.~{\bf Kevin Sapp}, PhD, {\sl University of Pittsburgh} \\
2101.~{\bf Anna Marburger}, {\sl eSalon} \\
2102.~{\bf Emily Quinty}, Physics Education Research Faculty, {\sl University of Colorado at Boulder} \\
2103.~{\bf Rachel A Rosen}, Assistant Professor, {\sl Columbia University} \\
2104.~{\bf Reginald T. Bailey}, Corporate Recruiting Manager, MS, {\sl ABM Government Services, LLC } \\
2105.~{\bf Gordon Berman}, Assistant Professor, {\sl Emory University} \\
2106.~{\bf Vinita Ruth Brocko}, MS \\
2107.~{\bf H. Daniel Ou-Yang}, Professor, {\sl Lehigh University} \\
2108.~{\bf Julia K Olsen}, PhD, {\sl University of Arizona} \\
2109.~{\bf Sarah "Sam" McKagan}, PhD in physics, {\sl American Association of Physics Teachers} \\
2110.~{\bf Sarah Tuttle}, Researcher, {\sl University of Texas at Austin} \\
2111.~{\bf Juan Manfredi}, PhD candidate, {\sl Michigan State University} \\
2112.~{\bf Jennifer Schenk Sacco}, PhD, Assoc. prof. of political science, {\sl Quinnipiac University} \\
2113.~{\bf Mark Allen Moore}, MS, {\sl Indiana University} \\
2114.~{\bf Monika Kress}, Professor of Physics \& Astronomy, {\sl San Jose State University} \\
2115.~{\bf Michele Stark}, PhD, {\sl University of Michigan-Flint} \\
2116.~{\bf John N Nishio}, Professor, {\sl California State University, Chico } \\
2117.~{\bf Kristal Shelvin}, PhD \\
2118.~{\bf Lucas Macri}, Associate Professor of Physics and Astronomy, {\sl Texas A\&M University} \\
2119.~{\bf Indara Suarez}, PhD, {\sl UCSB} \\
2120.~{\bf Curtis Bennett}, Professor of Mathematics, {\sl Loyola Marymount University} \\
2121.~{\bf Karl Mamola}, Professor Emeritus, {\sl Appalachian State Univ.} \\
2122.~{\bf Anna Williams}, PhD Student, {\sl University of Wisconsin-Madison} \\
2123.~{\bf Tiffany Glassman}, PhD \\
2124.~{\bf Bethany R. Wilcox}, PhD, {\sl University of Colorado Boulder} \\
2125.~{\bf Wendell D. Hall}, PhD, University of Maryland, College Park, {\sl The College Board} \\
2126.~{\bf Katherine Alatalo}, PhD, {\sl The Carnegie Observatories} \\
2127.~{\bf Jennifer Novotney}, PhD, {\sl MIT} \\
2128.~{\bf Paul D'Alessandris}, Professor of Physics and Engineering Science, {\sl Monroe Community College} \\
2129.~{\bf August Muench}, PhD, {\sl American Astronomical Society} \\
2130.~{\bf Cynthia Correa}, PhD, {\sl University of Texas at Austin} \\
2131.~{\bf Paul J. Camp}, Dr., {\sl Georgia Gwinnett College} \\
2132.~{\bf Amy Dae Nagy}, VMD, MS, {\sl AAAS Fellow at USAID} \\
2133.~{\bf Khalida S. Hendricks}, Graduate Fellow, Department of Physics, {\sl The Ohio State University} \\
2134.~{\bf Joseph T. Schick}, Associate Professor, {\sl Villanova University} \\
2135.~{\bf Annette Suggs}, TESL Educator, {\sl Kanto Kokusai High School } \\
2136.~{\bf Mark G. Vasquez}, PhD, {\sl Schwartz Law Firm PLLC} \\
2137.~{\bf Kevin Long}, Professor, {\sl Texas Tech University} \\
2138.~{\bf Pablo Bianucci}, Assistant Professor, {\sl Department of Physics, Concordia University, Montreal, Quebec, Canada} \\
2139.~{\bf Jennifer Delgado}, PhD, {\sl University of Kansas} \\
2140.~{\bf Heather Lewis}, B.S. Physics, Stanford University \\
2141.~{\bf Manuel Calderon de la Barca Sanchez}, Professor of Physics, {\sl University of California, Davis.} \\
2142.~{\bf Cristian Heredia}, PhD candidate, {\sl University of California, Davis} \\
2143.~{\bf Orkan Mehmet Umurhan}, PhD, {\sl NASA Ames Research Center} \\
2144.~{\bf Ed Romano}, Scientist/Engineer, {\sl Bio-Rad} \\
2145.~{\bf Jeyhan Kartaltepe}, Assistant Professor, {\sl Rochester Institute of Technology} \\
2146.~{\bf Jeremy Faludi}, Adjunt Professor, {\sl Minneapolis College of Art and Design} \\
2147.~{\bf Erika Peters}, Physics Professor, {\sl MiraCosta College} \\
2148.~{\bf Marcoantonio Arellano}, {\sl writer} \\
2149.~{\bf Neal N. Oza}, PhD, {\sl US Army Research Lab} \\
2150.~{\bf Sudip Chakravarty}, Professor, {\sl UCLA} \\
2151.~{\bf Michael Donnay}, Student, {\sl Georgetown University} \\
2152.~{\bf Gavin Oliver}, Chemistry/Physics Teacher/Tutor, {\sl Lane Communtiy College} \\
2153.~{\bf Joy Lawson Davis}, PhD, {\sl Virginia Union University} \\
2154.~{\bf William Rodney}, MBA, {\sl City of Chicago} \\
2155.~{\bf Rod Morison}, CTO, Slated Inc.; BS Caltech, {\sl Slated Inc} \\
2156.~{\bf Sidney Nagel}, Professor, {\sl University of Chicago} \\
2157.~{\bf Sara Campbell}, PhD Student, {\sl University of Colorado at Boulder} \\
2158.~{\bf Siddheshwari Advani}, Doctoral candidate, {\sl UMass Amherst} \\
2159.~{\bf Kathleen Tatem}, graduate student, physics PhD candidate, {\sl University of Hawaii} \\
2160.~{\bf Janaki Krishnamoorthy}, PhD, {\sl NY State DOH} \\
2161.~{\bf Nancy Rose}, Retired Professor \\
2162.~{\bf Bryan Terrazas}, MS, {\sl University of Michigan} \\
2163.~{\bf Albert K Henning}, PhD, {\sl Fraunhofer Institute for Photonic Microsystems} \\
2164.~{\bf R K P Zia}, Professor Emeritus, {\sl Virginia Tech} \\
2165.~{\bf Katy Garmany}, PhD, {\sl NOAO} \\
2166.~{\bf Jedidah Isler}, PhD, {\sl Vanderbilt University} \\
2167.~{\bf Ronald J. Sheehy}, PhD, {\sl Retired molecular biologist} \\
2168.~{\bf St\'{e}phane Test\'{e}}, Student, {\sl Johns Hopkins University} \\
2169.~{\bf Taryn Heilman}, Graduate student, {\sl University of Minnesota} \\
2170.~{\bf Ramon Thomas}, MD, {\sl The Lexington Clinic} \\
2171.~{\bf Rachael Livermore}, PhD, {\sl University of Texas at Austin} \\
2172.~{\bf John M. Brewer}, PhD Candidate, {\sl Yale University, Astronomy} \\
2173.~{\bf Eilat Glikman}, PhD \& Professor, {\sl Middlebury College} \\
2174.~{\bf Vishwanath Mohan}, B.S. Mech.Eng., JD, {\sl UC Irvine} \\
2175.~{\bf Andre Loose}, PhD \\
2176.~{\bf Robin B. Goodloe},  PhD, {\sl US Fish and Wildlife Service} \\
2177.~{\bf Dr. Gustavo del Castillo}, Dr. Political Science, {\sl Univ of San Diego} \\
2178.~{\bf Aram Harrow}, Assistant Professor, {\sl MIT} \\
2179.~{\bf Hayya Mintz}, Counseling Psychologist, {\sl Lehman Alternative Community School} \\
2180.~{\bf Tracy Hodge}, Associate Professor, {\sl Berea College} \\
2181.~{\bf Sarah Mullins}, PhD Chemistry, {\sl 3M Company} \\
2182.~{\bf Toyia Younger}, PhD \\
2183.~{\bf Paul Syers}, PhD, {\sl Potomac Institute for Policy Studies } \\
2184.~{\bf David Pontes}, MD candidate, {\sl University of Wisconsin-Madison} \\
2185.~{\bf Sean Downes}, PhD \\
2186.~{\bf James MacNaughton}, MS, RPA, {\sl Independent Contractor} \\
2187.~{\bf Nichole Harris}, MS, Medical Physicist \\
2188.~{\bf Justin Dressel}, Assistant Professor, {\sl Chapman University} \\
2189.~{\bf Oliver Elbert}, Graduate Astrophysicst, {\sl University of California, Irvine} \\
2190.~{\bf Jose Negrete JR}, Dr. Rer. Nat (PhD), {\sl Max Planck Institute for the Physics of Complex Systems} \\
2191.~{\bf Adam Martin}, Assistant Professor, {\sl University of Notre Dame} \\
2192.~{\bf Paul Zweifel}, Professor Emeritus, {\sl Virginia Tech} \\
2193.~{\bf Glenda Emanuelson}, Mrs., {\sl Retired educator} \\
2194.~{\bf James Gunton}, Professor of Physics, {\sl Lehigh University} \\
2195.~{\bf Elizabeth Behrman}, professor of mathematics and physics, {\sl Wichita State University} \\
2196.~{\bf Mart\'{i}n Hoecker-Mart\'{i}nez}, PhD, {\sl University of Michigan} \\
2197.~{\bf Cheryl Nahmias}, Ms., {\sl Decatur High School} \\
2198.~{\bf Kyle Sheppard}, Mr., {\sl University of Maryland} \\
2199.~{\bf Mark Helmlinger}, Spectral Remote Sensing Calibration Specialist, BS Physics 1991, Cal Poly Pomona, {\sl Jet Propulsion Laboratory} \\
2200.~{\bf Arlo Weil}, Professor of Geology, {\sl Bryn Mawr College } \\
2201.~{\bf Jack Holdford}, MSN, {\sl The Ohio State University College of Public Health} \\
2202.~{\bf Jacob Hummel}, PhD Student, {\sl The University of Texas at Austin} \\
2203.~{\bf Lining Wang}, Student, {\sl Yale University } \\
2204.~{\bf Elisabeth Mills}, Dr, {\sl University of Arizona / NRAO} \\
2205.~{\bf Thomas Gomez}, Graduate Student, {\sl University of Texas} \\
2206.~{\bf Brandon Swift}, MS, {\sl Steward Observatory, Univ. of AZ} \\
2207.~{\bf Stephanie LaMassa}, PhD, {\sl NASA GSFC} \\
2208.~{\bf Vinod Menon}, Professor, {\sl CCNY} \\
2209.~{\bf Jorge Moreno}, Professor, {\sl Cal Poly Pomona} \\
2210.~{\bf Aomawa Shields}, NSF Astronomy and Astrophysics Postdoctoral Fellow, UC President's Postdoctoral Fellow, {\sl UCLA/Harvard} \\
2211.~{\bf Brett K. Sandercock}, Professor, {\sl Kansas State University} \\
2212.~{\bf Mihir K Bhaskar}, Graduate Student, {\sl Harvard University} \\
2213.~{\bf Rachel Philippone}, MA, LPC, {\sl Self-employed} \\
2214.~{\bf Alexander J. Smith}, PhD candidate, {\sl Northwestern University} \\
2215.~{\bf Lasse Pommerenke}, {\sl Institut für angewandte Futuristik} \\
2216.~{\bf Salvador Adame-Zambrano}, MD, {\sl Memorial Medical Center, Las Cruces, NM} \\
2217.~{\bf Wouter Deconinck}, Assistant Professor, {\sl College of William \& Mary} \\
2218.~{\bf Benjamin L. Stottrup}, Associate Professor, {\sl Augsburg College} \\
2219.~{\bf Gabriel Dorfsman-Hopkins}, PhD candidate, {\sl University of Washington} \\
2220.~{\bf Rochell Isaac}, Ph.D., {\sl LaGuardia community College} \\
2221.~{\bf Colby Childress}, Optical Engineer, {\sl Varroc Lighting Systems} \\
2222.~{\bf Philipp Moesta}, Dr., {\sl UC Berkeley} \\
2223.~{\bf Marguerite Madden},  Professor and Director Ctr. for Geospatial Research, {\sl University of Georgia} \\
2224.~{\bf Eva de la Riva}, PhD (Chair and Professor of Psychology), {\sl Oakton Community College} \\
2225.~{\bf Margaret Moerchen}, PhD, {\sl Carnegie Institution for Science} \\
2226.~{\bf Darryl Cannady II}, Medical student, {\sl University of Miami Miller school of medicine } \\
2227.~{\bf Stephanie Sevigny}, Adjunct Instructor, Physics, {\sl College of Western Idaho} \\
2228.~{\bf Jessica L Ware}, Assistant Professor, PhD, {\sl Rutgers University} \\
2229.~{\bf Shane Mecklenburger}, Professor, {\sl University of Massachusetts Amherst} \\
2230.~{\bf Zachary Glassman}, Graduate Student, {\sl Joint Quantum Institute} \\
2231.~{\bf Akiba Kiiesmira}, Student, age 66, {\sl Southern New Hampshire University} \\
2232.~{\bf Nuala McCullagh}, PhD, {\sl Durham University} \\
2233.~{\bf Wanda Ashman} \\
2234.~{\bf Eli Levenson-Falk}, PhD, {\sl Stanford University} \\
2235.~{\bf Alex Storrs}, Prof., {\sl Towson Univ.} \\
2236.~{\bf Kathleen L. Sannicks-Lerner}, MEd, {\sl School District of Philadelphia} \\
2237.~{\bf Matt Brueckmann}, Student, {\sl UC Berkeley} \\
2238.~{\bf Joel Green}, PhD, {\sl STScI} \\
2239.~{\bf Allen Rozelle}, {\sl Volunteer Center, Santa Cruz} \\
2240.~{\bf Lynne R Spencer, Ph.D.}, STEM Instructional Specialist-Chemist, {\sl College of Engineering, University of Washington, Seattle, WA} \\
2241.~{\bf Annika Peter}, Professor, {\sl The Ohio State University} \\
2242.~{\bf Natalie Santiago}, MD, FAAP (Board certified pediatrician) \\
2243.~{\bf Kristin M. Block}, MS in Planetary Science, Spacecraft Science Engineer, {\sl The University of Arizona Lunar and Planetary Laboratory} \\
2244.~{\bf Barbara McArthur}, Research Scientist, {\sl UT Austin} \\
2245.~{\bf Edward Thomas}, Professor of Physics, {\sl Auburn University } \\
2246.~{\bf Bill Taylor}, Physics Teacher, retired, {\sl Drew School, Westmont HS} \\
2247.~{\bf Joseph Grange}, PhD, {\sl Argonne National Laboratory} \\
2248.~{\bf Logan Henderson} \\
2249.~{\bf Alberto Conto}, PhD Astrophysics, {\sl Northrop Grumman Aerospace} \\
2250.~{\bf Zachary Sharfman}, MS, MD candidate \\
2251.~{\bf Nathan Waugh}, Biophysics Graduate Student, {\sl Oregon State University} \\
2252.~{\bf Craig Ogilvie}, Morrill Professor, {\sl Iowa State University} \\
2253.~{\bf Katja Poppenhaeger}, PhD, Assistant Professor, {\sl Queen's University Belfast} \\
2254.~{\bf Willow Wedemeyer}, PhD, Assistant Professor, {\sl Michigan State University/Schule Birklehof} \\
2255.~{\bf Elinor Gates}, PhD, {\sl UCO/Lick Observatory} \\
2256.~{\bf John Coxon}, Dr, {\sl University of Southampton, UK} \\
2257.~{\bf J. Peter Brosius}, Distinguished Research Professor, {\sl University of Georgia} \\
2258.~{\bf Kelly Brittain}, PhD, RN, {\sl Michigan State University } \\
2259.~{\bf Ryan Whitcomb}, Graduate Student, {\sl University of Michigan} \\
2260.~{\bf Kimberly Cartier}, MS in Astronomy and Astrophysics, {\sl Penn State University} \\
2261.~{\bf Sasha Shirman}, PhD student, {\sl UC San Diego} \\
2262.~{\bf Will Parker}, PhD, {\sl University of Maryland} \\
2263.~{\bf Alessondra Springmann}, MSc, {\sl Lunar \& Planetary Laboratory, Department of Planetary Sciences, University of Arizona} \\
2264.~{\bf Danielle Lucero}, PhD, {\sl NRAO} \\
2265.~{\bf Justin Spilker}, PhD candidate, {\sl University of Arizona} \\
2266.~{\bf Robert W. Seidel}, Professor Emeritus, {\sl University of Minnesota} \\
2267.~{\bf Suzanne Brahmia}, Professor, {\sl Rutgers University} \\
2268.~{\bf Km Saunders}, Computer Scientist, {\sl Microsoft } \\
2269.~{\bf Charlotte Christensen}, PhD, {\sl Grinnell College} \\
2270.~{\bf Rachel Olzer}, PhD, {\sl University of Minnesota-Twin Cities} \\
2271.~{\bf Russell Scott Day}, Founder of Transcendia, {\sl Transcendia} \\
2272.~{\bf David Garrison, PhD}, Associate Professor of Physics, {\sl University of Houston Clear Lake} \\
2273.~{\bf Lee Samuel Finn}, Professor, {\sl The Pennsylvania State University} \\
2274.~{\bf Tim M.P. Tait}, Professor, {\sl University of California, Irvine} \\
2275.~{\bf Jonathan Bougie}, Associate Professor, {\sl Loyola University Chicago} \\
2276.~{\bf Maha Marouan}, PhD, {\sl Penn State} \\
2277.~{\bf Fleda Mask Jackson}, PhD \\
2278.~{\bf John Burk}, BS, MS Physics, {\sl St. Andrew's Schol, Delaware} \\
2279.~{\bf Thomas E Browder}, Professor, {\sl University of Hawaii} \\
2280.~{\bf Lori Downen}, MS, PhD student, {\sl UNC-CH} \\
2281.~{\bf Carlene Turner}, PhD, {\sl Norfolk State University} \\
2282.~{\bf Sharon L. Shields, PhD}, Professor of Human and Organizational Development and Associate Dean of Professional Education, {\sl Vanderbilt University's Peabody College of Education and Human Development} \\
2283.~{\bf Adam Watts}, Engineering Physicist, {\sl Fermi National Accelerator Laboratory} \\
2284.~{\bf Benjamin FitzPatrick}, PhD, {\sl State of Minnesota Weights and Measures} \\
2285.~{\bf Barry E. Weingarten}, PhD, {\sl Johns Hopkins University} \\
2286.~{\bf Zoe Van Hoover}, {\sl SLAC National Accelerator Laboratory} \\
2287.~{\bf Joan Pachner}, PhD, {\sl Museum of Modern Art} \\
2288.~{\bf Alexis Jeanette} \\
2289.~{\bf Farzad Faramarzi}, BSc, {\sl San Francisco state university } \\
2290.~{\bf Raynald A. Blackwell}, Director, {\sl Capital Guardian Youth Challenge Academy } \\
2291.~{\bf Vanessa Bailey}, PhD, {\sl Stanford University} \\
2292.~{\bf Andrea Derdzinski}, Graduate Student, {\sl Columbia University } \\
2293.~{\bf Peter T. Geantil}, PhD, {\sl Flux Power} \\
2294.~{\bf Mordecai-Mark Mac Low}, Curator \& Professor, {\sl American Museum of Natural History} \\
2295.~{\bf Mona Leila Mays}, PhD, {\sl Catholic University of America/NASA Goddard Space Flight Center} \\
2296.~{\bf Matthew Hogan}, MS, {\sl Colorado State University} \\
2297.~{\bf Knut Olsen}, PhD, {\sl NOAO} \\
2298.~{\bf Sandip Medhe}, Doctor, {\sl India } \\
2299.~{\bf Connor Gorman}, MS, {\sl UC Davis} \\
2300.~{\bf James Holmes}, MS, Educational Technology, {\sl Stanford University} \\
2301.~{\bf Christian Wunsch}, LCDR (Retired), {\sl Pennsylvania State University Applied Research Lab} \\
2302.~{\bf Samantha Dixon}, Graduate student, {\sl UC Berkeley} \\
2303.~{\bf Jeremy Templeton}, Phd, {\sl Sandia labs} \\
2304.~{\bf Marco Bairy Jesi}, PhD, {\sl University of Pennsylvania} \\
2305.~{\bf Daniel A. Wubah}, Professor, {\sl Washington and Lee University} \\
2306.~{\bf Andrea Peterson}, PhD, physics postdoc, {\sl Carleton University, Ottawa (U. Wisconsin PhD)} \\
2307.~{\bf Brian Frank}, Assistant Professor, {\sl Middle Tennessee State University} \\
2308.~{\bf Joel Gordon}, Emeritus Professor of Physics, {\sl Amherst College} \\
2309.~{\bf Darya Filippova}, PhD, Principal Scientist, {\sl Roche Sequencing Unit} \\
2310.~{\bf Matthew Penny}, Postdoctoral Fellow, {\sl The Ohio State University} \\
2311.~{\bf Christopher Johnson}, Professor, {\sl California College of the Arts} \\
2312.~{\bf Jessie Broussard}, PhD, {\sl University of Louisiana at Lafayette} \\
2313.~{\bf Jacqueline Radigan}, PhD \\
2314.~{\bf Marivi Fernandez-Serra}, Associate Professor of Physics, {\sl Stony Brook University } \\
2315.~{\bf Robin Toney}, Undergraduate Administrative Coordinator, Dept. of Math, {\sl University of Pennsylvnai} \\
2316.~{\bf Jesse Stryker}, Graduate student, {\sl University of Washington} \\
2317.~{\bf Paul Hunter} \\
2318.~{\bf Tammye Mathews}, ND, {\sl Restored by Tammye Inc.} \\
2319.~{\bf Alexandra Abate}, {\sl University of Arizona} \\
2320.~{\bf Javier F. Boyas}, PhD, {\sl The University of Mississippi } \\
2321.~{\bf Rachel E. Sapiro}, PhD \\
2322.~{\bf Christine Black}, PhD, {\sl Dartmouth College} \\
2323.~{\bf Jeremy D. Secaur}, High School physics teacher \\
2324.~{\bf Anne Leary Skenzich}, MA, MBA, PhD Candidate, {\sl University of Minnesota, School of Public Health} \\
2325.~{\bf Camilla Schneier}, Student, {\sl University of Pennsylvania} \\
2326.~{\bf Aaron Godfrey}, MSc, {\sl Federal Defense Contractor} \\
2327.~{\bf Sarah Ballard}, PhD, {\sl MIT} \\
2328.~{\bf Cyndie Beacham}, IT Professional, {\sl IBM/ATT} \\
2329.~{\bf Brendan Folie}, Graduate Student, {\sl UC Berkeley} \\
2330.~{\bf Crystal Patteson}, student, {\sl Appalachian State University} \\
2331.~{\bf Ian D. Beatty}, Associate Professor of Physics, {\sl University of North Carolina Greensboro } \\
2332.~{\bf John Hoffman}, MS, {\sl Princeton University} \\
2333.~{\bf Lucianne Walkowicz}, PhD, {\sl The Adler Planetarium} \\
2334.~{\bf Jorge Jim\'{e}nez}, MS Physics, {\sl University of Michigan } \\
2335.~{\bf Amanda Heiderman}, PhD, {\sl NSF Astronomy \& Astrophysics Fellow} \\
2336.~{\bf Christopher Burns}, PhD, {\sl Carnegie Observatories} \\
2337.~{\bf Maraippan Jawaharlal}, PhD, {\sl California State Polytechnic University, Pomona} \\
2338.~{\bf Christopher G. De Pree}, PhD, {\sl Agnes Scott College} \\
2339.~{\bf Benjamin Nicholson}, PhD student, {\sl Cornell University} \\
2340.~{\bf DeWayne Allen}, {\sl United Technologies Corporation } \\
2341.~{\bf Alfred B. Bortz}, PhD, {\sl Self-employed science writer} \\
2342.~{\bf Karen Knierman}, PhD, {\sl Arizona State University} \\
2343.~{\bf Matthew Irvin Brookhart}, PhD, {\sl Intel Corp} \\
2344.~{\bf Sheldon Campbell}, PhD, {\sl University of California, Irvine} \\
2345.~{\bf William Carpenter}, PhD candidate, {\sl University of Chicago} \\
2346.~{\bf Sangeeta Malhotra}, Prof., {\sl Arizona State University} \\
2347.~{\bf John Ruan}, {\sl University of Washington} \\
2348.~{\bf Rachael Alexandroff}, Graduate student, {\sl Johns Hopkins University} \\
2349.~{\bf Ugochukwu Nze}, PhD Student, {\sl University of Utah} \\
2350.~{\bf Stephen Hampe}, PhD, {\sl Walden University} \\
2351.~{\bf James W. LaMance, Jr.}, PhD, Aerospace Engineering, {\sl Constell, Inc.} \\
2352.~{\bf Andrew Ferreira}, Teacher, {\sl Thomas Edison CTE High School} \\
2353.~{\bf David Ensminger}, PhD, {\sl Loyola University Chicago} \\
2354.~{\bf Stephanie Hunte}, Educator \& Student Advocate, {\sl Doctoral Student - Clark Atlanta Univeristy} \\
2355.~{\bf Lauren Corlies}, Graduate Student, {\sl Columbia University} \\
2356.~{\bf Ben Zurkow} \\
2357.~{\bf Christopher Jantzi}, PhD candidate, {\sl University of Virginia} \\
2358.~{\bf Paul M. Amy}, Graduate Research Fellow, {\sl Rensselaer Polytechnic Institute} \\
2359.~{\bf Wladimir Lyra}, Professor of Astrophysics, {\sl California State University, Northridge} \\
2360.~{\bf William Thornton}, MFA, {\sl JP Getty Museum} \\
2361.~{\bf Chelsea Sharon}, Research Associate, {\sl Cornell University} \\
2362.~{\bf Aurelia Williams}, PhD, {\sl Norfolk State University } \\
2363.~{\bf Garrett Somers} \\
2364.~{\bf Sukanya Chakrabarti}, Assistant Professor, {\sl RIT} \\
2365.~{\bf Stefano Meschiari}, PhD, {\sl UT Austin} \\
2366.~{\bf Sarah Brosnan}, PhD \\
2367.~{\bf Priscilla Canizares}, PhD, {\sl Cambridge University} \\
2368.~{\bf Vinay Hegde}, MS, {\sl Northwestern University} \\
2369.~{\bf John Weiss}, PhD, Assistant Professor of Physics, {\sl Saint Martin's University} \\
2370.~{\bf Andrew Gorzen}, MS in Applied Mathematics, {\sl Western Michigan University} \\
2371.~{\bf Mel Sabella}, PhD, {\sl Chicago State University} \\
2372.~{\bf Kirk A Nienaber}, MD, {\sl Nashville, TN} \\
2373.~{\bf Leslie Kerby}, Dr., {\sl Idaho State University} \\
2374.~{\bf Catherine Capello}, Professor, {\sl Salem State University} \\
2375.~{\bf Matt Block}, Professor, {\sl Sacramento State (CSUS)} \\
2376.~{\bf Asad Aboobaker}, PhD, {\sl NASA-JPL} \\
2377.~{\bf Gordon Watts}, Professor, {\sl University of Washington} \\
2378.~{\bf Ryan Cybulski}, PhD Student, {\sl University of Massachusetts, Amherst} \\
2379.~{\bf Erin D. Glover}, Doctoral Student, Mathematics Education, {\sl Oregon State University} \\
2380.~{\bf Reed Riddle}, PhD, {\sl Caltech} \\
2381.~{\bf Steven Greenstein}, PhD, {\sl Montclair State University} \\
2382.~{\bf Gail Olabisi}, MD \\
2383.~{\bf Marcos D Caballero}, Assistant Professor, {\sl Michigan State University } \\
2384.~{\bf Mary Lou Faherty}, Technician, {\sl IATSE Local \# 7} \\
2385.~{\bf Dylan Roderick}, {\sl The Ohio State University } \\
2386.~{\bf Michelle Collins}, Lecturer, {\sl University of Surrey} \\
2387.~{\bf Christopher Orita} \\
2388.~{\bf Michelle Schultz}, PhD, {\sl University of Illinois-Chicago} \\
2389.~{\bf Tiffany Simon}, PhD, {\sl Sam Houston State University} \\
2390.~{\bf Vivian U}, BS, Mechanical Engineering; JD, {\sl UC Riverside} \\
2391.~{\bf Kisha Delain}, M.S., {\sl University of St. Thomas} \\
2392.~{\bf Elena Long}, PhD, {\sl University of New Hampshire} \\
2393.~{\bf David Hopped}, {\sl University of Pennsylvania } \\
2394.~{\bf Erin Kraal}, PhD, {\sl Kutztown University} \\
2395.~{\bf Jonathan Pober}, PhD, {\sl Brown University} \\
2396.~{\bf Natoshia Anderson}, EdD, {\sl Georgia Piedmont Technical College} \\
2397.~{\bf Elizabeth Wright}, PhD, {\sl Penn State University} \\
2398.~{\bf Monika Alem}, MD Candidate, {\sl Keck School of Medicine at University of Southern California} \\
2399.~{\bf Anne-Marie Hoskinson}, PhD, ecology, {\sl Michigan State University} \\
2400.~{\bf Susan Shadle}, PhD, Professor, {\sl Boise State University} \\
2401.~{\bf Vicky}, Professor, {\sl Northwestern University} \\
2402.~{\bf Sara Watt}, MS, {\sl Glendale Community College, AZ} \\
2403.~{\bf Jessie MNG Lopez}, M.S., {\sl CSU, Northridge} \\
2404.~{\bf Tee S Emm}, PhD Physics, {\sl Institute of Science NALCS} \\
2405.~{\bf Delilah Gates}, {\sl Harvard University} \\
2406.~{\bf T. Elon Dancy}, PhD, {\sl University of Oklahoma} \\
2407.~{\bf Russell Herman}, Professor, Physics and Mathematics, {\sl UNC Wilmington} \\
2408.~{\bf Luca Bombelli}, PhD, {\sl University of Mississippi} \\
2409.~{\bf Mark Horvath}, Astronomer \\
2410.~{\bf Susan A. Rabe}, Dr., {\sl North Park University} \\
2411.~{\bf Krista Lewicki}, M.S., Wildlife Biology, {\sl U.S. Geological Survey } \\
2412.~{\bf David Mortimer}, Teacher of Mathematics and Science, {\sl Bank Street School for Children} \\
2413.~{\bf Markus Luty}, Professor, {\sl University of California, Davis} \\
2414.~{\bf Michael Goho}, Professor of Engineering Science/Physics, {\sl Monroe Community College} \\
2415.~{\bf Duncan Christie}, PhD, {\sl University of Florida} \\
2416.~{\bf Jude T. Socrates}, PhD Caltech;, {\sl  Pasadena City College} \\
2417.~{\bf Matthew Bayliss}, Professor, {\sl Colby College} \\
2418.~{\bf Brooke D. Simmons}, PhD; Einstein Fellow, {\sl University of California, San Diego} \\
2419.~{\bf Stacey Alberts}, PhD, {\sl Steward Observatory} \\


}
\end{multicols}

\end{document}